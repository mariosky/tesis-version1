\chapter{Case of study} \label{case}

\section{Experimental settings} 

In this particular experimental scenario, the basic guidelines 
proposed in Adomavicius \cite{adomavicius2011context} are followed,  
and they are briefly explained next. 
\begin{itemize} 
\item \textbf{Hypothesis:} before running the experiment we must form
an \textit{hypothesis}. It is important to be concise and restrictive
about this hypothesis, and design an experiment that tests the
hypothesis. For example, an hypothesis can be that  \textbf{algorithm
A} predicts better user ratings than  \textbf{algorithm B}.  In that
case, the experiment should test the prediction accuracy, and not
other factors.
\item \textbf{Controlling variables:} when comparing a few candidate
algorithms on a certain hypothesis, it is important that all
\textit{variables} that are not tested will stay fixed. For example,
suppose that we wish to compare the prediction accuracy of movie
ratings of  \textbf{algorithm A} and \textbf{algorithm B}, that both
use different collaborative filtering models.
\item \textbf{Generalization power:} when drawing conclusions from
experiments, we may desire that our conclusions generalize beyond the
immediate context of the experiments. When choosing an algorithm for a
real application, we may want our conclusions to hold on the deployed
system, and generalize beyond our experimental data set. Similarly,
when developing new algorithms, we want our conclusions to hold beyond
the scope of the specific application or data set that we experimented
with. It is important to understand the properties of the various data
sets that are used. Generally speaking, \textit{the more diverse the
data used, the more it can generalize the results}.
\end{itemize} 

\subsection{Off-line experiments} 

An off-line experiment is performed using a pre-collected dataset
of users choosing or ratings. Using this dataset tries to simulate
the behavior of users that interact with a recommender system. In
doing so, it assume that the user behavior when the data was collected
will be similar enough to the user behavior when the recommender
system is deployed, so that we can make reliable decisions based on
the simulation.  Off-line experiments are attractive because they
\textit{require no interaction with real users}, and thus it allows to compare
a wide range of candidate algorithms at a low cost. \\ The downside is
that it can answer a very narrow set of questions, typically questions
about the prediction of an algorithm. The goal of the off-line
experiments is to filter out inappropriate  approaches, leaving a
relatively small set of alternatives algorithms for subsequently to be
tested for the more costly user studies or on-line 
experiments\cite{adomavicius2011context}.\\ 

The next sections show a comprehensive description of the 
experimental setup for experiments, as well as results obtained 
in the experiments. Each method was tested using contextual 
datasets in the domain.  This chapter ends with the entire 
description of the functionality of the prototype used to 
validate the method utilized in the experiments.
%% Haz un puente, dices que vaz a hacer ciertos experimentos
%% menciona aquí que lo que sigue es el setup de tus parte experimental.
\section{Restaurants recommendations} \label{restaurants}.
%%%%%%%
Previous the implementation of context-aware recommender system  
some experiments were done to test the behaviour of algorithms 
and the performance in the proposed method. 
For the experiment the proposed hypothesis is \textit{The accuracy of 
recommendations is improved using the context in the 
recommendation process}. To validate the hypothesis the next test  
was realized. \\
A first experiment\cite{ramirez2013restaurant}  presents a  
contextual recommender system using the post-filtering approach and 
collaborative filtering technique in a restaurant domain. 
Collaborative filtering works to get a Top-N list utilized to adjust it in 
the context. 
Later, the Top-N list is obtained and the restaurants are adjusted to 
make ranking of restaurants  in the current context. Post-filtering is
based on the average  of ratings in a specific context, so prediction
is made with: 1) \textit{the average} that a restaurant has in the
current context (that is the  mean of user ratings) and 2) 
\textit{the rating} predicted by the collaborative filtering algorithm. 
The top-N list contains the restaurants with highest predictions, 
so each restaurant is adjusted for the user's context and listed in 
contextual recommendations; the process is depicted in figure
\ref{fig:postfiltering}.
\begin{figure*}
\centering
\captionsetup{font=footnotesize}
\setlength\fboxsep{0pt}
\includegraphics[width=0.50\textwidth]{img/posfil.png}
\caption{The post-filtering approach for Tijuana restaurants.}
\label{fig:postfiltering}     
\end{figure*}
%%%%%%%%%%%%
In order to validate the proposed approach,  data  about 
restaurant preferences of users in different contexts was used.
The study subjects were students  with a major in engineering,  
master's program and professors of the Tijuana Institute of
Technology. A total of \textit{50 users} answered a questionnaire; the
questions were about their preferences for nearby restaurants and the
technology most often used by them. The \textit{questionnaire} consisted 
of \textit{8 questions} and also they were asked to rate any number 
of restaurants from a list of 40 restaurants.
Each of the restaurants chosen, was rated 6 times one per proposed 
context, a matrix rating with \textit{1,422 ratings} was collected. The
questions are shown in table \ref{tab:questions}. The reason for allowing
users to chose what restaurants to rate it to give them the same liberty
that they have when visiting a web or mobile application. 
\begin{table}
\small
\captionsetup{font=footnotesize}
\caption{Questionnaire applied to collect contextual dataset.}
\label{tab:questions} 
\centering
\small
\begin{tabular}{p{7cm} p{5cm} }
\hline\noalign{\smallskip}
Question & Answers \\
\noalign{\smallskip}\hline\noalign{\smallskip}
\small{1.What is your occupation?} & \small{1. Student 2.Employee} \\ \hline  
\small{2.According your priority, order by importance the features 
you consider when you choose to visit a restaurant.} & 
\small{1.Installation/decoration 2.Prices 3.Service 4.Dishes
5.Atmosphere 6.Location} \\ \hline  
\small{3.What are the devices that you used
utilizes?} & \small{1.Smartphone 2.Tablet 3.Laptop 4.PC} \\ \hline   
\small{4.What Operating System do you used?} & 
\small{1.Android 2.Windows 3.iOS 4.Symbian 5.Blackberry 6.Other}
\\ \hline  
\small{5.Did you use an application to search restaurants in Tijuana?} &
\small{1.Yes 2.No 3.Which one?} \\ \hline   
\small{6.Would you like to use an application of
recommender systems of Tijuana?} & \small{1.Yes 2.No} \\ \hline  
\small{7.Please, rates your favorites restaurants(without context).} & 
\small{Restaurant list} \\ \hline
\small{8.Please, rates your favorites restaurants in contextual situations.} & 
\small{Restaurant list} \\
\noalign{\smallskip}\hline
\end{tabular}
\end{table}
The user's answers from question 1 to question 6 are represented in
the figure \ref{fig:cakeschart}. \textit{Figure \ref{fig:cakeschart}a}
represents the percentage of surveyed students and teachers;
\textit{figure \ref{fig:cakeschart}b}  the percentage of the element
that users consider the most important to visit a restaurant;
\textit{figure \ref{fig:cakeschart}c} represents the preferences of
devices when are using Internet for restaurant recommendations;
\textit{figure \ref{fig:cakeschart}d} represents the percentage of
operating system that often used, \textit{figure
\ref{fig:cakeschart}e} shows the percentage of users that use the
Internet to search restaurants in Tijuana; and \textit{figure
\ref{fig:cakeschart}f}, shows the percentage of users that would like
using a restaurant recommender system of Tijuana.
\begin{figure*}
\captionsetup{justification=centering,margin=2cm,font=footnotesize}
\centering
\setlength\fboxsep{0pt}
\includegraphics[width=0.8\textwidth]{img/cakes.png}
\caption{The chart cakes show the users preferences for questions from 1 to 6.}
\label{fig:cakeschart}     
\end{figure*}
For questions 7 and 8 only the top-ten restaurants are shown,
without/with the contextual situation. In figure \ref{fig:barschart}a,
the favorite restaurant is \textbf{Daruma}(178 votes),  whereas in
figure \ref{fig:barschart}b, \textbf{Daruma} does not appear in the
top-ten. When considering the context \textit{midweek}, the favorite
restaurant was \textbf{Carl's Jr.}, which appears in both graphs; this
restaurant was also the most voted in the different contexts.
\begin{figure*}
\captionsetup{justification=centering,margin=2cm,font=footnotesize}
\centering
\setlength\fboxsep{0pt}
%\setlength\fboxrule{0.7pt}
\includegraphics[width=0.55\textwidth]{img/bars.png}
\caption{The chart shows the users preferences for questions 7 and 8.}
\label{fig:barschart}     
\end{figure*}
Contextual recommendations of post-filtering approach depends of
context \textit{midweek} or \textit{weekend}, which is the day when
the restaurants were rated. Subsequently, the result of the query is
refined according to the user context; the 6 contexts mentioned
correspond to combinations of contextual factors shown in table
\ref{tab:contextstijuana}.
\begin{table}
\small
\captionsetup{font=footnotesize}
\caption{Contextual factors considered in the questionnaire.}
\label{tab:contextstijuana} 
\centering
\begin{tabular}{p{2.5cm} p{7cm} }
\hline\noalign{\smallskip}
Contextual Factor & Context \\
\noalign{\smallskip}\hline\noalign{\smallskip}
\small{Day} & \small{1.Midweek(Monday, Thuesday,Wednesday and Thursday)
2.Weekend(Friday,Saturday and Sunday)}  \\ \hline 
\small{Place} & \small{1.School 2. Home 3.Work} \\ 
\noalign{\smallskip}\hline
\end{tabular}
\end{table}
The  mean absolute error obtained was \textbf{0.5859} 
in contextual recommendations. 
The observation for this result is that using a small
dataset the performance of the method proposed is limited, the cold-start 
problem affects the accuracy because of the data scarcity.

\section{Hotels recommendations} \label{hotels}

A second experiment using TripAdvisor dataset was used. 
For this case, the proposed method consists of three algorithms 
to recommend:  \textit{fuzzy inference system}, \textit{collaborative 
filtering} and \textit{content-based}. Each one uses the ratings 
matrix to get recommendations.\\    
The contextual recommender system uses  \textit{post-filtering}
approach\cite{adomavicius2011context} for adjust recommendations in
context such as restaurants. The recommendation by popularity is 
through the fuzzy inference system depicted in figure \ref{fig:fis}, 
the fuzzy inference
system contains the variables that are involved in the process to
recommend in a human interaction, this process is the same that the
recommender system does. \\The output represents how matter each item
into the users community, i.e. if it is a popular item between users. \\
The dataset used to evaluate the algorithm was TripAdvisor in two
versions downloaded\cite{linkzeng}, this datasets was used in
\cite{zheng2014context} and \cite{zheng2012differential} to  evaluate the
performance of context-aware recommender systems. \\The first
dataset contains 4669 contextual ratings, 1202 users and 1890 hotels;
the second dataset contains 14175 contextual ratings, 2731 users and
2269 hotels. Data were collected of reviews online in tripadvisor.com.
There is only one context: \textit{type of trip} (family, friends, bussines,
romantic and relax).\\ 
The FIS has Gaussians membership functions and are depicted in figure
\ref{fig:mffis}.
\begin{figure}[ht!]
   \captionsetup{font=footnotesize}
   \centering
   %%----primera subfigura----
   \subfloat[]{
        \label{fig:1a}
        \includegraphics[width=0.42\textwidth]{img/ratingaverage.png}}
   \hspace{0.1\linewidth}
   %%----segunda subfigura----
   \subfloat[]{
        \label{fig:1b} 
        \includegraphics[width=0.42\textwidth]{img/userparticipation.png}}\\[20pt]
   %%----tercera subfigura----
    \subfloat[]{
        \label{fig:1c} 
        \includegraphics[width=0.42\textwidth]{img/recommendation.png}}
   \caption{Gaussian Membership functions in the input are: a) RatingAverage, 
   b) UserParticipation, and an output: c) Recommendation.}
   \label{fig:mffis} 
\end{figure}
The fuzzy inference system uses fuzzy rules to infer the inputs and 
output(a crisp value) that represents the weight of the recommendation. 
The rules are following: 
\begin{enumerate}
\item \textit{If \textbf{RatingAverage} is low and 
\textbf{UserParticipation} is insufficient then \textbf{recommendation} is low.}
\item \textit{If \textbf{RatingAverage} is low and 
\textbf{UserParticipation} is sufficient then \textbf{recommendation} is high.}
\item \textit{If \textbf{RatingAverage} is high and 
\textbf{UserParticipation} is insufficient then \textbf{recommendation} is low.}
\item \textit{If \textbf{RatingAverage} is high and 
\textbf{UserParticipation} is sufficient then \textbf{recommendation} is high.}
\end{enumerate}
\begin{figure*}
\captionsetup{justification=centering,margin=2cm,font=footnotesize}
\centering
\setlength\fboxsep{0pt}
\setlength\fboxrule{0.7pt}
\includegraphics[width=0.75\textwidth]{img/fis.png}
\caption{Fuzzy inference system.}
\label{fig:fis}   
\end{figure*}
Content-based uses cosine similarity to compare the binary
vectors representing the profile of each item, thereby obtaining a
numerical value that determines similarity, based on a threshold. \\   
In other words, it makes a comparison of profiles of each item to
determine the most similar to items the user has rated with highest
score, context-aware recommender system proposed has a scale 
from 1 to 5. 
\begin{table}[htb]
\small
\centering
\captionsetup{font=footnotesize}
\caption{Example of contextual ratings in the user profile.}
\label{tab:2}
\small
\begin{tabular}{lll}
\hline
\multicolumn{3}{c}{\textbf{User profile}} \\ \hline
Item & Rating & Context \\ \hline
La Casa del Mole & 5.0 & Midweek \\ 
Daruma           & 4.0 & Weekend \\ 
Daruma           & 5.0 & Midweek \\ 
Carl's Jr.       & 3.0 & Weekend \\ \hline
\end{tabular}
\end{table}
Next, the outputs of every recommender technique is represented by a
list of recommended items. Subsequently applies the context filter and
context-aware recommender system gets the final contextual
recommendations. Context-aware recommender system identifies
contextual data of the user profile (see table \ref{tab:2}), and
compares recommended items to filter those items that are adjusted to
the user context.  The context filtering is the last step before to
get the recommended items. The schema of architecture for context-
aware recommender system is depicted in figure \ref{fig:architecture}.
\begin{figure*}
\captionsetup{font=footnotesize}
%\captionsetup{justification=centering,margin=2cm}
\centering
\includegraphics[width=0.80\textwidth]{img/archit-ta.png}
\caption{Recommender system methodology.}
\label{fig:architecture}   
\end{figure*}
Two tests were performed using TripAdvisor dataset, table
\ref{tab:3} describes the data sets and the scarcity percentage of the
specified data. Scarcity of 99\% mean that there are problems to
recommend items because the information is not enought to get 
good recommendations.\\  By other side, in table \ref{tab:4} the comparison
shows that the algorithm has a acceptable performance, i.e., the error
falls into the range of results obtained with others algorithms. Then,
contextual recommendations were evaluated with the Root Mean Square
Error in order to compare the results with context relaxation
algorithm\cite{zheng2012differential} that is evaluated with the same
dataset.
\begin{table}
\centering
\small
\captionsetup{font=footnotesize}
\caption{Datasets description.}
\label{tab:3}      
\begin{tabular}{lllll}
\hline\noalign{\smallskip}
Dataset & Users & Items & Ratings & Scarcity (percent) \\
\noalign{\smallskip}\hline\noalign{\smallskip}
TripAdvisor v1 & 1202 & 1890 & 4669 & 99.79 \\
TripAdvisor v2 & 2731 & 2269 & 14175 & 99.77 \\
\noalign{\smallskip}\hline
\end{tabular}
\end{table}
\begin{table}
\centering
\small
\captionsetup{font=footnotesize}
\caption{Comparison of RMSE.}
\label{tab:4}  
\small   
\begin{tabular}{lll}
\hline\noalign{\smallskip}
Dataset & Algorithm & RMSE \\
\noalign{\smallskip}\hline\noalign{\smallskip}
TripAdvisor v2 & FC + Post-filtering  & 0.504  \\
               & FC          & 0.994  \\
               & Pre-filtering + Relaxation & 0.985  \\
\noalign{\smallskip}\hline
\end{tabular}
\end{table}
The cosine similarity plays an important role in content-based because
if similarity value among items is high, the recommendations will
improve the degree of user satisfaction. \\ This is observed when
calculating the similarity average in each dataset as shown in table
\ref{tab:5}.
\begin{table}
\centering
\small
\captionsetup{font=footnotesize}
\caption{Level of similarity among items in datasets. }
\label{tab:5}      
\begin{tabular}{lll}
\hline\noalign{\smallskip}
Dataset  & Similarity  & Avg.votes per user. \\
\noalign{\smallskip}\hline\noalign{\smallskip}
TripAdvisor v1 & 0.448  & 5  \\
TripAdvisor v2 & 0.508  & 8  \\
\noalign{\smallskip}\hline
\end{tabular}
\end{table}
Fuzzy inference system can provides a list of popular items for each dataset,
recommendations through averages obtained, and recommendations are
conditioned to show it when the collaborative filtering and content-
based are not delivering recommendations because of data scarcity.\\ 
However, the majority of popular items of dataset were rated in contexts: 
\textit{romantic, family and bussines}, that means that the dataset has
biases that affects the results.


\section{Context-aware recommender system prototype} 

This section presents a context-aware recommender system prototype.
The backend have been explained in chapter \ref{method},  in sections
\ref{restaurants} and \ref{hotels} talk about experiments realized
using the recommendation techniques proposed. \\ To develop the prototype
was used python language, technologies as Django Framework 1.7,
JavaScript, JQuery, Ajax, HTML5, Bootstrap 3.0  and PotsgreSQL for database.
Some dependencies and libraries were used also, it can review links of
downloads in appendix \ref{appendixc}.

\subsection{User Interfaces}

The system starts in a landing page, the user should do \textit{Sign in} or
\textit{Sign up} to create a new user to enter the home page. 
Landing page contains the \textit{Best restaurants} rated and 
\textit{Featured top list restaurants}
proposed by users, such as shows it the figure \ref{fig:landing}, 
the restaurants are updated while users add ratings, 
if the tendency is changed, the section displays the change 
in the tendency. The aim is to provide
information for old and new users,the prototype tries to keep updated
the current preferences of users constantly.
\begin{figure*}
\captionsetup{font=footnotesize}
\centering
\fbox{\includegraphics[width=0.70\textwidth]{img/landingpage.png}}
\caption{Landing page interface.}
\label{fig:landing}   
\end{figure*}
%%%%%%%%%%
\subsubsection{Home Page interface}
%%%%%%%%%%
\textit{Home Page} (figure \ref{fig:home-page}), shows the main 
menu, tags of user preferences,
all filters to start the search of restaurants an the popular
restaurants in the user community. Users can start exploring filters
to find restaurants under their own criteria, each restaurant has a
profile with complete information about the characteristics and
opinions of other users. When users click the restaurant picture the
system redirects to the profile, for instance figure \ref{fig:rest-profile2}  
shows the profile of Daruma restaurant, in this figure is showed the general
information of the restaurant, reviews or personal opinions of users
that visited the restaurant, details about ratings, the chart of
ratings and the user location using Google maps services. It is also
added the button to \textit{add wishlist}, this element will be explained in a
posterior section.
\begin{figure*}
\captionsetup{font=footnotesize}
\centering
\fbox{\includegraphics[width=0.70\textwidth]{img/homepage.png}}
\caption{Home page interface.}
\label{fig:home-page}   
\end{figure*}
%%%%%%%%%%%%%%%%%%%
\begin{figure*}
\captionsetup{font=footnotesize}
\centering
\fbox{\includegraphics[width=0.70\textwidth]{img/rest-profile.png}}
\caption{Restaurant profile interface.}
\label{fig:rest-profile2}   
\end{figure*}
%%%%%%%%%%%%
\subsubsection{My Recommendations interface}
%%%%%%%%%%%%%
\begin{figure*}
\captionsetup{font=footnotesize}
\centering
\fbox{\includegraphics[width=0.70\textwidth]{img/expert-recs.png}}
\caption{Expert recommendations interface.}
\label{fig:expert-recs}   
\end{figure*}
In recommendations interface users have the options to get  
recommendations such as was described in chapter \ref{method}, four 
recommendation techniques works to suggest restuarants for 
users. Here, users can choose the prefered option to get 
information in the screen.\\ 

For instance, figure \ref{fig:expert-recs} shows that the expert can 
not to recommend
whether users don’t have participations, because of expert's
recommendation is based in fuzzy rules that involves the popularity of
items (input variable), if the information is sparse the prototype can
not get results and only shows an alert message. On the contrary case,
it shows a list of popular restaurants.\\ 
%%%%%%%%%
\begin{figure*}
\captionsetup{font=footnotesize}
\centering
\fbox{\includegraphics[width=0.70\textwidth]{img/base-content.png}}
\caption{Content-based recommendations interface.}
\label{fig:base-content}   
\end{figure*}
%%%%%%%%%%%%%%%
\textit{Content-based recommendation} finds the similar restaurants to the
favorites of the user. In fact, makes a comparison of all restaurants
and gets the more similars to display in the screen. 
Figure \ref{fig:base-content}  
 shows the results of the search, large amount of restaurants are showed
because of the similarity among restaurants, this is the most
efficiently way to get recommendations when the system has not enough
information about user preferences and it the users community is
small. It is not functional whether the user has not at less one vote
with high rating (5 stars).\\ 
%%%%%%%%%%%%%%%
\textit{Collaborative filtering recommendation} is depicted in the 
figure \ref{fig:cf-recs}
this technique is based in users' opinions, if the system has not
information of users, or whether the user community is small, 
cold-start problem highlights in results. 
\begin{figure*}
\captionsetup{font=footnotesize}
\centering
\fbox{\includegraphics[width=0.70\textwidth]{img/cf-recs.png}}
\caption{Collaborative filtering recommendations interface.}
\label{fig:cf-recs}   
\end{figure*}
As in expert  recommendations if the users participations 
is low or null, the results of recommendations will be empty 
and the alert message appears in the
screen, else, the restaurants suggested will be displayed 
in the screen.\\
%%%%%%%%%
\textit{Nearby recommendations} (figure \ref{fig:nearby-recs}) 
provides recommendations when the
user previously specifies a geographical position. The distance to
considers restaurants from the current position to two kilometers
around. Then, recommendations are dis- played in the screen if the
location is provided in the context, else, an alert message  notifies
the lack of user current location.
\begin{figure*}
\captionsetup{font=footnotesize}
\centering
\fbox{\includegraphics[width=0.70\textwidth]{img/nearby-recs.png}}
\caption{Nearby recommendations interface.}
\label{fig:nearby-recs}   
\end{figure*}
%%%%%%%%%%%%%%
\subsubsection{My Profile interface}
%%%%%%%%

My Profile interface shows all the information of the user divided in
three sections: my preferences, my reviews and my favorite restaurants
as shows the figure \ref{fig:myprofile}. The preferences represent the basic
information and tastes of user, it contains the tastes and preferences
of the user, some factors as price, position, cuisine and attributes
of restaurants could be changing continuosly, it depends how the user
want manage its information. The prototype uses this information every
time that a request of recommendation is done.\\
My reviews section shows all the reviews that the user typed, each
review is stored in database and the user cannot delete it. The stars
represent the level of satisfaction of the user when he/she visits the
restaurant, each review highlights the good and the bad things that
the user perceives in the visit. This information is valuable but,
unfortunatly this prototype not uses it to infer possible tastes and
preferences and subsequently, recommendations. This process might be a
functionality integrated in the future.\\
My favorite restaurants section, displays restaurants rated with five
stars. It means that are the more important options of the user to
visit in this moment and in the future. These restaurants are related
to the content-based recommendation because are the restaurants with
high rating of this particular user.
%%%%%%%%%%%%%%%%%%%%%%
\begin{figure*}
\captionsetup{font=footnotesize}
\centering
\fbox{\includegraphics[width=0.70\textwidth]{img/myprofile.png}}
\caption{My Profile interface.}
\label{fig:myprofile}   
\end{figure*}
%%%%%%%%%%%%%%%%%%%%%
%%%%%%%%%%%%%%
\subsubsection{My Context interface}
%%%%%%%%
My Context interface is divided in two sections: contextual 
information and current location as in the figure \ref{fig:mycontext}.
%%%%%%%%%%%%%%%%%%%
\begin{figure*}
\captionsetup{font=footnotesize}
\centering
\fbox{\includegraphics[width=0.70\textwidth]{img/mycontext.png}}
\caption{My Context interface.}
\label{fig:mycontext}   
\end{figure*}
 %%%%%%%%%%%%%%%%%%%
Contextual information represents the user preferences, for instance
the price range that wants for a specific ocasion, the attributes are
the characteristics of restaurants, so it selects as many as the user
prefers, the limit is the number of characteristics displayed. \\ 
The same is for cuisines, a total of 30 type of cuisines were proposed
considering the list of restaurants in the prototype, it can select as
many as the user want.  All this information is stored and displayed
in home page also in order to help the user to remember which is the
information that the prototype is using to display recommendations.\\
%%%%%%%%%%%%%%%%%%%%%%%%%%%
 Current location section shows the Google map to display the 
 current location, previously the user ought to confirm that wants 
 to share the current location. When user click in the 
 \textit{Save My Location button}, the content in the box of latitude 
 and longitude will be stored in the database(alert message 
 confirms the action) and the status will be modified when 
 user location changes, then, all the locations stored in 
 database become user's historical information.
%%%%%%%%%%%%%%%%%%%
\subsubsection{My Wishlist interface}
%%%%%%%%%%%%%%%%%%%
\begin{figure*}
\captionsetup{font=footnotesize}
\centering
\fbox{\includegraphics[width=0.70\textwidth]{img/mywishlist.png}}
\caption{My Wishlist interface.}
\label{fig:mywishlist}   
\end{figure*}
%%%%%%%%%%%%%%%%%%
My Wishlist interface contains the restaurants that are 
considered by the user as future options to vist (figure \ref{fig:mywishlist}) 
or simply to have available information for a particular 
ocasion. The wishlist allows the manipulation of restaurants, 
it means, user can add or delete restaurants as many as 
he/she wants.The wishlist could be empty or full of 
restaurants, this information does not affect any functionality 
of the prototype. \\
Making a comparison with others platforms (for instance 
Amazone), this information allows the user to have the own 
\textit{personalized store}, in terms of products sales, the wishlist 
should be the future products to buy. In a similar way, 
this prototype tries to infer the future tastes of the user 
or what restaurants could visit the user in a close future. \\
To facilitates the addition of restaurants the \textit{Add my wishlist} 
button was added in every restaurant of the home 
page (see figure \ref{fig:wishlist-home}) and in restaurants 
profiles (see figure \ref{fig:rest-profile2}).
%%%%%%%%%%%%%%%%%%%
\begin{figure*}
\captionsetup{font=footnotesize}
\centering
\fbox{\includegraphics[width=0.70\textwidth]{img/wishlist-home.png}}
\caption{Functionality of wishlist button in Home page.}
\label{fig:wishlist-home}   
\end{figure*}
\subsubsection{Community interface}
%%%%%%%%
Community interface contains the reviews of all the users (figure
\ref{fig:community}) in order to display information that may helps 
another users to select any restaurant to visit. In the real life, the 
persons recommend some or another restaurant in the common 
language, remarking the why the restaurant is good or bad according 
their perception. Community section has the same goal, in fact, 
some reviews could be useful for the other users, this opinion may 
express it through the \textit{Helpful button}.
%%%%%%%%%%%%%%%%%%%
\begin{figure*}
\captionsetup{font=footnotesize}
\centering
\fbox{\includegraphics[width=0.70\textwidth]{img/community.png}}
\caption{Community interface.}
\label{fig:community}   
\end{figure*}
%%%%%%%%%%%%%%%%%%%

These are the functionalities in the prototype, in order to validate
the performance, an on-line experiment was realized, the metrics 
of usability applied were Time-taks and Taks-success (mentioned 
in chapter  \ref{introduction}). In the next chapter are 
explained the tests  and the results obtained.







