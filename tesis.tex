\documentclass[oneside,numbers,english]{./ezthesis}
%% # Opciones disponibles para el documento #
%%
%% Las opciones con un (*) son las opciones predeterminadas.
%%
%% Modo de compilar:
%%   draft            - borrador con marcas de fecha y sin im'agenes
%%   draftmarks       - borrador con marcas de fecha y con im'agenes
%%   final (*)        - version final de la tesis
%%
%% Tama'no de papel:
%%   letterpaper (*)  - tama'no carta (Am'erica)
%%   a4paper          - tama'no A4    (Europa)
%%
%% Formato de impresi'on:
%%   oneside          - hojas impresas por un solo lado
%%   twoside (*)      - hijas impresas por ambos lados
%%
%% Tama'no de letra:
%%   10pt, 11pt, o 12pt (*)
%%
%% Espaciado entre renglones:
%%   singlespace      - espacio sencillo
%%   onehalfspace (*) - espacio de 1.5
%%   doublespace      - a doble espacio
%%
%% Formato de las referencias bibliogr'aficas:
%%   numbers          - numeradas, p.e. [1]
%%   authoryear (*)   - por autor y a'no, p.e. (Newton, 1997)
%%
%% Opciones adicionales:
%%   spanish         - tesis escrita en espa'nol
%%
%% Desactivar opciones especiales:
%%   nobibtoc   - no incluir la bibiolgraf'ia en el 'Indice general
%%   nofancyhdr - no incluir "fancyhdr" para producir los encabezados
%%   nocolors   - no incluir "xcolor" para producir ligas con colores
%%   nographicx - no incluir "graphicx" para insertar gr'aficos
%%   nonatbib   - no incluir "natbib" para administrar la bibliograf'ia

%% Paquetes adicionales requeridos se pueden agregar tambi'en aqu'i.
%% Por ejemplo:
%\usepackage{subfig}
%\usepackage{multirow}
\usepackage{graphicx,rotating,booktabs}
\usepackage{lscape}
%\usepackage[spanish]{babel}
\usepackage[spanish,USenglish]{babel} % espanol, ingles
%\usepackage[utf8]{inputenc} % acentos sin codigo
%\renewcommand{\chaptername}{chapter} 
%\renewcommand{\tablename}{Table}
%\renewcommand{\figurename}{Figure}
\usepackage{multirow, array} %para mover el ancho de las tablas.
\renewcommand{\baselinestretch}{2}
%% # Datos del documento #
%% Nota que los acentos se deben escribir: \'a, \'e, \'i, etc.
%% La letra n con tilde es: \~n.
%0.15
\usepackage{vmargin}
\setpapersize{USletter}
\setmargins{3.2cm}       % margen izquierdo
{2.5cm}                        % margen superior
{15cm}                      % anchura del texto
{20cm}                    % altura del texto
{10pt}                           % altura de los encabezados
{1cm}                           % espacio entre el texto y los encabezados
{10pt}                             % altura del pie de página
{1cm}                           % espacio entre el texto y el pie de página


\usepackage{caption}
\usepackage{subfig}

\usepackage{amssymb, amsmath, amsbsy} % simbolitos
\usepackage{upgreek} % para poner letras griegas sin cursiva
\usepackage{cancel} % para tachar
\usepackage{mathdots} % para el comando \iddots
\usepackage{mathrsfs} % para formato de letra
\usepackage{stackrel} % para el comando \stackbin

%\usepackage{algpseudocode} %Para el apendice.
\usepackage{algorithm}
\usepackage{algorithmic}
\usepackage{listings}
%\usepackage{algpseudocode}
%\usepackage{pifont}

\usepackage[verbose]{placeins}
%\usepackage[ngerman]{babel}
\usepackage{blindtext}


%\usepackage[showframe,paper=a4paper,top=1in,bottom=1in,right=0.5in,left=1.5in]{geometry}% http://ctan.org/pkg/%geometry

%\author{Xochilt Ramirez Garcia}
%\title{ Sistema de Recomendaci\'on Sensible al Contexto y su Aplicaci\'on al Secuenciado de Objetos }
%\degree{Doctor en Ciencias en Computaci\'on}
%\supervisor{Nombre de mi Asesor}
%\institution{Instituto Tecnol\'ogico de Tijuana}
%\faculty{Divisi\'on de Estudios de Postgrado e Investigaci\'on }
%\department{Departamento de Sistemas Computacionales}

%% # M'argenes del documento #
%% 
%% Quitar el comentario en la siguiente linea para austar los m'argenes del
%% documento. Leer la documentaci'on de "geometry" para m'as informaci'on.

%\geometry{top=40mm,bottom=33mm,inner=40mm,outer=25mm}

%% El siguiente comando agrega ligas activas en el documento para las
%% referencias cruzadas y citas bibliogr'aficas. Tiene que ser *la 'ultima*
%% instrucci'on antes de \begin{document}.

\hyperlinking
\begin{document}

\pagenumbering{Roman}
%\documentclass{article}
\usepackage{graphicx,rotating,booktabs}
\usepackage{vmargin}

\usepackage{geometry}
\geometry{letterpaper}

\setmargins{3cm}       % margen izquierdo
{2.5cm}                        % margen superior
{15cm}                      % anchura del texto
{22.5cm}                    % altura del texto
{10pt}                           % altura de los encabezados
{1cm}                           % espacio entre el texto y los encabezados
{10pt}                             % altura del pie de página
{2cm}                           % espacio entre el texto y el pie de página

\renewcommand{\baselinestretch}{2}

\begin{document} 

\begin{center} 
%\vspace*{\stretch{1.0}}
{\textsc{\LARGE \textbf{SEP} } \hfill  \textsc{\LARGE \textbf{TNM}}}\\[0.2cm]
	\textsc{\LARGE \textbf{INSTITUTO TECNOL\'OGICO DE TIJUANA}\\[0.5cm]
	\textsc{\Large Divisi\'on de Estudios de Posgrado }}\\[0.1cm]
	\textsc{\Large e Investigaci\'on }}\\[0.5cm]
	% Upper part of the page
	%\includegraphics[height=4cm]{img/logotec1.png}
	\includegraphics[width=0.4\textwidth]{img/logotec1}\\[0.3cm] %0.15
	%\HRule \\[.2cm]
	{ \LARGE \bfseries Sistema de Recomendaci\'on Sensible al Contexto y su Aplicaci\'on al Secuenciado de Objetos}\\ [1.0cm]
\end{center}

 \begin{minipage}{1.0\textwidth}
 	\begin{flushright} 
 	\textsc{Trabajo de tesis}\\[0.3cm]
 	\emph{Presentado por:} \\
 	\textsc{ M.C. Xochilt Ram\'irez Garc\'ia}\\
 	%\vspace{4 mm}}
	\emph{Para obtener el grado de:} \\
 	\textsc{Doctor en Ciencias en Computaci\'on} \\
 	%\vspace{4 mm}}
	\emph{Director:} \\
 	\textsc{Dr. Jose Mario Garc\'ia Vald\'ez} \\
 	\vspace{4 mm}
	\emph{Tijuana, B.C. Octubre del 2016.}
 	\end{flushright}
\end{minipage}


\pagestyle{empty} % Removes page numbers


\end{document}



%% Las secciones del "prefacio" inician con el comando \prefacesection{T'itulo}
%% Este tipo de secciones *no* van numeradas, pero s'i aparecen en el 'indice.
\prefacesection{Acknowledgments}

\textit{This work is for everybody that trusted my effort and
dedication to do everything that I proposed.\\  I specially want to thank
my sisters Rosy, Nely and Mary and of course my parents,
as they are the pillars that keeps me on the way.\\  I want to thank my
friends: Maribel, Cinthya, Sam and Lorenzo, they are so special for me
and they are always there, as partners, with me until the end.
I’m lucky to have their friendship.\\  I want to acknowledge the effort and
patience of my advisor  for four years Dr. Mario Garc\'ia Vald\'ez. I appreciate the
knowledge shared and the instructions about conducting the research. Nowadays I
have the experience and tools to face up a new phase in my life. Thank you!.\\  
Finally, I would like to express my gratitude to CONACYT and
Tijuana Institute of Technology for the facilities and resources
granted for the development of this thesis.}

%% Los cap'itulos inician con \chapter{T'itulo}, estos aparecen numerados y
%% se incluyen en el 'indice general.
%%
%% Recuerda que aqu'i ya puedes escribir acentos como: 'a, 'e, 'i, etc.
%% La letra n con tilde es: 'n.


% 
%MG: En el resumen debes hablar de todo el trabajo revisa: 
%% http://www.slideshare.net/jjmerelo/cmo-escribir-y-publicar-trabajos-cientficos-17948694

% Da una breve introducción al tema (problema)
% Los métodos que pleanteamos aquí son {metodología}
% para [mejorar, evitar, comprobar] {problema}  
% Se probó de la manera {x} que los objetivos se alcanzaron  
% con los resultados siguientes {x} 
 

\prefacesection{Resumen}

%%%% Empieza aquí:
Los sistemas de recomendaci\'on sensibles al contexto
consideran factores que dependen directamente del
dominio de aplicaci\'on, el objetivo del sistema y la situación actual del usuario.  
Normalmente en la implementaci\'on de este tipo de sistemas  
se utilizan t\'ecnicas h\'ibridas con el objetivo de 
mejorar el desempe\~{n}o, siguiendo estrat\'egias de pre y post
filtrado. En esta tesis se propone una
metodolog\'ia para el desarrollo de sistemas de recomendaci\'on sensibles al contexto,
utilizando l\'ogica difusa para expresar las reglas y variables utilizadas
al describir la situaci\'on actual y los atributos de las entidades que intervienen.
El objetivo de esta metodolog\'ia es permitir tanto a usuarios como dise\~nadores
el uso de variables ling\"uisticas para mejorar facilitarles la interacci\'on con el sistema.
Para validar la propuesta se implementaron dos aplicaciones web con las 
se hicieron pruebas de usabilidad y satisfacción. 

Al integrar estas técnicas el desempeño mejoró en {x} respecto a {y}
de manera estadísticamente significativa logrando para los datos de prueba 
un error absoluto de XX.


%\chapter{Resumen}
\prefacesection{Abstract}

Traditional recommender systems has been improved in recent years,
context has been used to implement personalized and contextualized
recommendations for this kind of systems in order to increase the user
satisfaction with respect of the recommended items. These systems are
called context-aware recommender systems and are based in pre-filtering and post-filtering mainly. The context is defined  into the
application due to the large amount of contextual factors, these
depend of the domain and the system goals. Context-aware recommender
systems use hybrid methods and fuzzy techniques to improve the
processing of language human-machine. This thesis defines  an
architecture  for a context-aware recommender system that utilizes
recommendation techniques (collaborative and content-based), Fuzzy
Inference Systems and fuzzy attributes that are involved in the
contextual recommendation process.












%% # 'Indices y listas de contenido #
%% Quitar los comentarios en las lineas siguientes para obtener listas de
%% figuras y cuadros/tablas.
%\begin{singlespace}
\tableofcontents
\listoffigures
\listoftables
%\end{singlespace}
\clearpage
%\listoffigures
%\listoftables
%% # Cap'itulos #
%% Por cada cap'itulo hay que crear un nuevo archivo e incluirlo en esta seccion.
\pagenumbering{arabic}
\chapter{Introduction} \label{introduction} 

The purpose of this research is to make a contribution in the
application of contextual information in recommender systems by
proposing a methodology for the development of context aware
recommender systems. The method follows an hybrid  approach by
integrating several recommendation techniques through a fuzzy
inference system. The proposed method was validated by the
execution of several experiments, and the implemantion of a case
study. The results presented are favorable, indicating the
strenghts of the hybrid method.

\section{Motivation}

Often people need to take decisions, even when they do not
have enough experience or information to discern among 
the possible alternatives. For this, is common for people to 
seek help by asking friends  or people which they  
have certain level of affinity for their recommendations.\\
The problem of making a decision is aggravated by the 
large amount of information generated by society, 
resulting in an information overload where users are not capable of
sorting out the relevant information from the irrelevant.
This lack of experience, time and other resources 
highlights the need of automatic or semi-automatic methods 
to filter relevant information. Currently, information filters are
an integral component of the Web, where for instance
automatic search engines help users to find or make 
decisions through (some times) personalized recommendations
of products and services. \\ On the other hand, in recent
years  mobile computing has drasticaly incremented their importance, 
because of the impact of their use in daily life, if an application 
is available in a smartphone it can be used all most anywhere, and thanks
to sensors and GPS technology location aware applications are now common.\\ 
New technologies also consider information about the user's current situation 
in mobile applications, as intelligent systems can take advantage of the
benefits that this technology provides, to manage the constantly changing 
context. \\ Developments in mobile and ubiquitous computing \cite{noguera2012mobile} 
\cite{chiou2010adaptive}, are proposing a wide variety of applications
that will benefit from recommendation engines using context to meet their
purpose. But still there is a need to model the information available in
the system in order to provide context awareness.\\  Information regarding 
context can also be described in natural language in a fuzzy manner. For
instance when describing the current situation an user could say: \textit{``Is early in
the morning and I am currently near my work and I need to find a coffee
shop that is not to expensive, is open and not very far"}.
This sentence uses many words to describe the situation that are not 
crisp, but instead, the description uses fuzzy variables to describe the situation.  

\section{Context of use}\label{contextofuse}

Context is an important concept in everyday life. People often provide
context when writing postcards, referring to the weather or the holiday
atmosphere. A knowledge of context can also help to explain why a certain
object is produced or designed the way it is. When a product 
(or system) is developed, it
will be used within a particular context. It will be used by a user
population with certain characteristics. The user will have certain
goals and has the intension to perform certain tasks. The product will also be used
within a certain range of technical, physical and social or
organizational environments \cite{maguire2001context} that may affect
its use.\\   
We can refer to these environments as the \textit{context of use} (see
Figure \ref{fig:logicalmodel}), this concept has been formally defined
by ISO standard 9241-11 \cite{international1998iso} as \textit{``users,
tasks and equipment (hardware, software and materials), and the
physical and social environments in which a product is used"}. \\ 
In daily life we often find products that are difficult to use or
understand. This type of difficulties are \textit{usability problems}
that arise from  diverse issues that have not been addressed in the
design of the product's \textit{user-artifact interaction}. The
\textit{user-artifact interaction} refers the way that the user
interacts with a product and vice versa, this term has been studied
in Human Computer Interaction. \\As emerging technologies constantly
change the way people interact with products and their physical
environment, recent studies have started looking at \textit{human
experience} as a source to generate products or systems that
\textit{engage} the user.\\
The \textit{user experience}, \textit{context of use} and
\textit{product usability} have been associated in computer sciences
field. The \textit{Usability} is became a well-established concept in
the IT world to represent the user-friendliness of a system. However,
there was a need to establish the concept more clearly and to
determine how to measure it. Probably the best known definition of
usability is by Nielsen \cite{nielsen1994usability}: 
\textit{``usability is about learnability, efficiency, 
memorability, errors, and satisfaction".}
However, the definition of usability from ISO
9241-11 \cite{international1998iso}:\textit{``the extent to which a
product can be used by specified users to achieve specified  goals
with effectiveness, efficiency and satisfaction in a  specified
context of use"}, is becoming the main reference of usability. \\
Thus, taking in account these definitions we can say that designing for 
\textit{usability} involves establishing user requirements for a new
\textit{system} or \textit{product}, developing design solutions,
prototyping the system and the user interface, and testing it with
representative users.\\
In the study of Sato \cite{sato2004context} that brings \textit{context
issues} into design practice, addressed the concept of
\textit{context} as a critical component of the design information in
order to enhance the \textit{human-centred design practice}. After a
literature's revision for  definitions of \textit{context} in diverse
fields, Sato explains that there are external and internal conditions
into the definitions and suggest that it has four characteristics:
\begin{enumerate}  
\item Aspects of context are based on the nature of actions and 
conditions.
\item Descriptions depends on the focus of the viewpoints.
\item Contextual changes are triggered from differents elements 
of the domain. 
\item Context evolves over the time, some aspects change fast
and others change slow. 
\end{enumerate} 
From this, Sato defined \textit{context} as a \textit{mental model} or a
\textit{pattern of one's memory} triggered by \textit{elements in the
situation}, where situation is a collective condition at the scene of
interaction composed of relations among \textit{variables of
conditions}. Sato employed this concept to describe the
\textit{influence of contexts} in \textit{people's interactions} and
\textit{system performance} and vice versa.

\subsection{Logical model of context} 

The \textit{logical model} explains the relationship between \textit{context},
\textit{contextual information} and \textit{contextual factors}. 
Figure \ref{fig:logicalmodel} gives a graphical representation of these relations. 
The goal is to facilitate the use and implementation of fuzzy context 
in recommender systems for different domains.\\
In the Figure \ref{fig:logicalmodel}, the first box shows different
real life situations of the user (contexts), the user can change from situation
to situation in a little time or in a long time, this situational or
contextual information in the real world provides the abstract knowledge that
the system will use to determine the \textit{``contextual information"} that will 
be used. 
From the real world, specific data will be extracted by different means as
sensors, user interaction or even manual input. In this work 
we call this data \textit{``contextual factors"}(for instance, place and
orientation, preferences, date and time, etc.) and represent the
information that affects the recommendation process. The information
could be represented as fuzzy variables or crisp variables depending of
its domain. Later, contextual factors are implemented as data
structures that define the domain for each one. 
%\begin{landscape} 
\begin{figure*}
\captionsetup{font=footnotesize} \centering
\includegraphics[width=1.0\textwidth]{img/context-scheme.png}  
\small
\caption{Logical data model of context.}
\label{fig:logicalmodel}    
\end{figure*} %\end{landscape}

\section{Context-awareness} \label{context-awareness}

Traditional recommender systems provide suggestions of useful items
for a certain user. The suggestion relates to various decision-making
processes, for instance, what items to buy, what music to listen to, or
what on-line news to read. \textit{Item} is the general term to denote
what product or service the system recommends for each user. A
recommender system normally focuses only on a single type of item
 \cite{resnick1997recommender}.
Some improvement efforts for recommender systems, have been mainly
focused on the
\textit{integration of context} in the recommendation process. 
The idea behind \textit{context-aware computing} is to provide
information or services for the user based in the user's situation
 \cite{dey2001understanding}. In order to do that, the application 
needs to obtain situational data, process it, and make use of it 
in a manner that benefits the user. \\ 
\textit{Context} is a concept that is not easy to define, as it is related with
several disciplines that propose different definitions. For example,
the authors Bazire et.al. \cite{bazire2005understanding} compare the
notion of context in different fields and conclude that is complicated to make a
unifying definition of context because of the nature of the concept in
many disciplines. In computer science Fischer \cite{fischer2012context}
defines context as \textit{``the interaction between humans and
computers in socio-technical systems that takes place in a certain
context referring to the physical and social situation in which
computational devices and environments are embedded"}. 
Fischer also identifies
the important aspects to consider when the context is used: how it is
the contextual data obtained, how the context is represented and what
goals and purposes the context has when is used in a particular
application. \\
Probably the definition most used in the field of recommender systems to 
define \textit{context} is proposed by Dey \cite{dey2001understanding}:
\textit{``Context is any information that can be used to characterize
the situation of an entity. An entity is a person, place or object
that is considered relevant to the interaction between a user and an
application, including the user and applications themselves.''}  This
definition makes it easier to define the contextual factors in a
specific application. For instance, in a tourist guide application, the
entities can be companion(friends, family, couple), place of interest,
season and weather, these could be considered as relevant contextual
factors that help the recommender system to provide items adjusted the
situational data of the user.\\
\textit{Context-aware recommender systems} are gaining even more
attention because of their performance and implementation for
different domains, the  way to improve personalized recommendations
based in contextual factors is an important technique to increase the
benefits in  many domains. For instance, taking in account the
\textit{hour of the day},  or the \textit{day of the week} when
recommending restaurants could  filter out restaurants that are
currently closed or near closing time, when the user receives this
information in real time, the user has the  way for taking
alternatives of restaurants that provide services. Nowadays, many
companies are incorporating some type of context (as time, location or
companion) in their recommendation engines, the application can be
found in fields such as e-commerce \cite{schafer1999recommender}
 \cite{bulander2005enabling}, music \cite{ricci2012context}
 \cite{baltrunas2011incarmusic}  \cite{huq2010automated}, places of
interest \cite{baltrunas2012context},
movies \cite{eyjolfsdottir2010moviegen}, vacation
packages \cite{liu2011personalized}  \cite{liu2014cocktail},  travel
guides \cite{savage2012m}, e-learning \cite{ortigosa2010entornos}  and
restaurants \cite{chu2013chinese}.\\
Plus, context can be used to improve the user satisfaction  in
recommender systems, thus the quality and accuracy of predictions  
is improved too. \\
The method proposed in this thesis uses three recommendations techniques,
applied as a case study in a restarant domian:
\begin{enumerate} 
\item \textit{Expert's Fuzzy Inference System}, this is a rule based 
recommender defined by an expert in the domain, in the case of 
a restaurant recommender, it considers the following
variables: \textit{ratings average: low,medium and high},
\textit{price of restaurant: cheap, average and expensive}, and
\textit{number of ratings of item: few, several and many}, these
variables are used to infer how relevant a restaurant is, for the user.
This recommendation is based on the popularity of each item in the
user's community.
\item \textit{Content-based technique} utilizes the item profiles 
to compare how \textit{similar} is an item with respect to 
another, i.e. restaurants that are \textit{similar} (same cuisine, 
ambient, price range) to others that the user has rated high. 
The idea is to find items with similar features. 
\item \textit{Collaborative filtering technique} is based on the user
profile to identify user's preferences and to find neighbors that
have the same tastes. The recommendation consist in the suggestions of
other users with similar tastes that rated restaurants again in a
similar way but where have not been rated by the current user. A Top-N
list of restaurants is obtained to recommend for the user.
\end{enumerate} 
The results of the three techniques are a list of recommendations for
the user, later, these recommendations are adjusted for the current context.
This is the last step and is represented as a \textit{context filter} in the
method, as a result a list of contextualized
recommendations is obtained. \\
In the method, each technique works simultaneously to obtain
recommendations, the hybrid method allows to generate suggestions even
without user information, i.e., using content-based technique or the
fuzzy inference system, so the system faces the cold-start or the
overspecialization problem using these thechniques, these problems are
described in section \ref{coldstart} and 
 \ref{overspecialization}, respectively.\\
To evaluate the performance of the proposed recommender several experiments
were maded, the algorithms were tested using contextual datasets and
the number of contextual factors used varied according to the information
provided by the dataset. The goal of the experiments were to observe
the role of contextual data  in the performance of the algorithms and to 
have a better understanding of what contextual
factors are more important for users in a specific situation, how recommendations
are improved using context and, the accuracy in recommendations.
Chapter 5 shows the results while discussion about results are
explained in each section.\\
As a user-cenetered system another important aspect 
to consider and measure is the \textit{user satisfaction}, for this two
metrics were used: \textit{task-success} and \textit
{time-on-task}. These usability metrics allow designers to measure 
the user experience.
\\ Usability
metrics can help reveal patterns that are hard or even impossible to
see. Evaluating software applications with a small sample size 
(between 5 and 8 tests) usually reveals the most obvious 
usability problems \cite{albert2013measuring}.\\ 
Then, as
a general rule of thumb, during the early stages of design, it needs
fewer participants to identify the major usability issues. As the
design gets closer to completion, the tests should include more
participants to identify the remaining
issues \cite{albert2013measuring}.\\ 
Following this precept, ten representative users were selected to test
the system, subsequently, it was realized an analysis about the system
performance and issues presented in the user interaction. The chapter
6 explains the process to evaluate the system and the
results obtained.

\section{Aims}

Recommender systems has been used to obtain relevant items for current user
through simple models (user-item), the technique is limited 
because of ignore information about behaviour of users and others
factors that are involved in the user environment. 
To implement new tecnologies to adapt
the recommender algorithms for the user needs is the novel approach in
recommender systems. 
The purpose of them is to improve the suggestions in order to increase the user
satisfaction as well as to facilitate the interaction user-system.
The improves proposed by researchers in this field involve the context
implementation, through contextual factors defined according the domain 
proposed, and considering 
experiences and recommendations of other authors.
The goal of this thesis highliths the implementation of context in 
a method that includes fuzzy logic
techniques and pre-filtering paradigm that uses traditional recommender
techniques improved to make recommendations. To achieve this, 
uses of fuzzy rules to treat linguistic terms that include fuzzy variables, 
this allows the use of aproximated information 
to the real context of the user, and it allows an improved
performace of the recommender system because analyzes the user 
preferences and obtain recommendations
based in contextual factors. 
For instance, the system provides a list of
range of prices, this allows the users to select a specific range of
price to get recommendations adjusted for the tag selected, this tag represents
a common word of their native language.\\
The contribution of this thesis is the novel method that 
includes a hybrid recommender
system (content-based and collaborative filtering techniques). It
provides a \textit{useful knowledge} 
to utilize in the hybrid recommender system, provides techniques to 
maximize the use of context by fuzzy variables that represent a part
of context. The context complementation is represented by external factors
such as web services. 
Plus, the method has features that allow to adapt it to be
implemented in different domains such as e-learning, movies,
music, tourism, etc. For a case of study, the restaurants domain
was used to test the method.\\
Several particular aims were done in order to support the achievement 
of contributions:
\begin{itemize}  
\item Elaborate an analysis about the state of the art in the field
of context-aware recommender systems through  the revision of 
the relevant literature. State of the art helps to define the tendency
of currents recommender systems, the recent improves and helps to understand
what is the relevance of the proposed method as well as the factibility and 
how we can make an important contribution to the recommender systems field.
\item Selection of algorithms that are representative of the alternatives for
this problem domain in order to compare their perfomance. An analisys of the 
features and performs of the algorithms was done, it was extracted of papers,
books, and toolbox results to choose the suitable algorithm
for the method. There are enough papers to make a consense about 
what is that we want from each one. 
\item Design and conduct several experiments with the proposed 
algorithms. The experiments were based in previous results of similar 
methods, the algoritms were adjusted to improve their performance and 
testing with datasets for several applications, such as travels, hotels, 
restaurants and movies. The  metric implemented to measure the experiments 
results was root mean square error beause of is the most recommended 
by literature.
\item Propose a hybrid method and apply it in a case of study. 
Subsequently the results obtained, the proper algorithms and paradigm was 
included in the proposed method. As well as the fuzzy inference system and 
the fuzzy variables to represent the context used by the 
recommender system.
\item Develop a prototype of a context-aware recommender system 
using the proposed method. Using recent technologies to develop a proper 
interface for restaurant recommender system as case of study, 
the development was done to apply the proposed method. The 
context is represented by fuzzy variables and database models 
related the user model and characteristics
that describe the restaurant model.
\item To evaluate the proposed method the usability metrics were proposed. 
We consider the nature of the recommender system and the results 
that we can get of usability test. The test was prepared to be realized
by real users that never have been interacting with recommender systems. 
Results are explained in a posterior chapter.
\end{itemize} 

\section{Outline}

The rest of this thesis is organized as follows: 
\begin{itemize}  
\item \textbf{Chapter 2} describes an in-depth study
and background of current and related work, presenting a general
overview of recommender systems and their evolution in recent years.
This study includes traditional recommender systems, their methods
and techniques to improve recommendations, aswall as the challenges 
of these systems. Subsequently, hybrid methods used in different
applications, their limitations and advantages for each hibridation and
the domains of application. Finally, context-aware recommender systems
are disscussed, in the same way,  with an analysis of the advantages and
disadvantages of the use of context in recommender systems.
\item \textbf{Chapter 3} describes the fundamental
concepts that form the basis of the proposed method.
\item \textbf{Chapter 4} presents a model of context-aware
recommender system, the proposed method involves the paradigm of
post-filtering in a restaurants domain. This chapter includes the
overall explanation of data models and  the functionality, as
well as its components for this case of study.
\item \textbf {Chapter 5}, the general results of different
projects involved are presented along with the validation of every
experimentation. The experiments were realized using different
datasets and different algorithms in order to find an optimal manner
to reduce the error level. This chapter also details the results for
each experiment from a point of view of scientific results.
\item \textbf {Chapter 6}, after the development of
context-aware recommender system, the impact of
context in recommendation process was evaluated. 
This chapter describes the
usability tests that were applied on-line in order to evaluate the
satisfaction of users. Details of the environment and the
characteristics of the tests are described, as well as 
the results of each one.
\item \textbf {Chapter 7}, this chapter concludes with a
summary of its contributions and limitations. Final
conclusions are drawn and also proposals for future work are presented.
\end{itemize}  
At the end, this thesis includes appendices that describe
detailed technical aspects about installation and libraries of python
that uses the context-aware recommender system  \textit{(appendix \ref{appendixa})}, 
the pseudocode of algorithms proposed in the method are available in  \textit{(appendix \ref{appendixb})}, 
and experiment study materials used to obtain the test results are in  \textit{(appendix \ref{apendixc})}.

\chapter{State of the art} \label{stateoftheart}

%Propongo una Sección de RS tradicionales y
%otra de Context Aware


In this section the state of the art in 
context aware recommender systems is presented. 
As a technology recommender systems have beem applied 
in many domains, and sometimes they represent the  key
technology for the success of web and mobile applications.

%Debemos presentar el SoA con cierto orden, ya sea como una linea
%del tiempo, empezando de los primeros trabajos a los más recientes
%o organizarlo por dominios, tecnologías etc. Voy a 
%clasificar los diferentes trabajos y después los organizamos
% 


%Tipo de información: Social, Aplicación Social 
%Tags
Some works utilize social information to recommend such
as Manca et.al.\cite{manca2014mining} where the friend recommender
system is applied in the social bookmarking domain, its goal was to
infer the interest of users from content selecting the available
information of the user behavior and analyzing the resources and the
tags bookmarked for each user, therefore the recommendations are
through mining user behavior in a tagging system, analyzing the
bookmarks tagged of the user and the frecuency for each used tag. 
%Keywords
J.Yao et.al.\cite{yao2012product} proposes a new product recommendation
approach for new users based on the implicit relationships between
search keywords and products. The relationships between keywords and
products are represented in a graph and relevance of keywords to
products is derived from attributes of the graph.
%Semantic
The relevance
information is utilized to predict preferences of new users. J.
Golbeck et.al.\cite{golbeck2006filmtrust} presents FilmTrust, a
website that integrates Semantic Web-based social networks, augmented
with trust, to create prediction movie recommendations. Trust takes on
the role of a recommender system forming the core of an algorithm to
predict a rating for recommendations of movies. This is an example of
how the Semantic Web, and Semantic trust networks in particular, can
be exploited to refine the user experience. \\  
%Eso de pros y cons me lo han criticado, me pedían algo más específico
%Este párrafo está bien por que se organiza por problema que atacan las
%propuestas
Traditional recommender techniques has its pros and cons, for
instance, the ability to handle data sparsity and cold-start problems
or considerable ramp-up efforts for knowledge acquisition and
engineering. Establish hybrid systems that combine the strengths of
algorithms and models to overcome some of the shortcomings and
problems has become the properly manner to improve the difficults for
each algorithm.
%Turistas
An example is presented by L.Castro et.al.\cite{castro2012prototype} 
a hybrid recommender system for the province of San Juan, Argentina, 
to recommend tourist packages  based on preferences and interest 
of each user, artificial intelligence
techniques are used to filter and customize the information. The
prototype of recommender system utilizes three techniques to
recommend: demographic, collaborative and content-based. The goal is
to recommend tourist packages that matches with the user profile.
%Restaurantes
L. Martinez et al.\cite{martinez2009reja} presents REJA, a hybrid
recommender system that involves collaborative filtering and
knowledge-based model, that is able to provide recommendations in some
situation for user; besides it provides georeferenced information
about the recommended restaurants.
%De los primeros
Balabanovic et.al.\cite{balabanovic1997fab} presents
Fab, a hybrid recommender system for automatic recognition of
emergent issues relevant to various groups of users. It also enables
two scaling problems, pertaining to the rising number of users and
documents, to be addressed. Claypool et.al.\cite{claypool1999combining} 
presents P-Tango system that utilizes content-based and collaborative
filtering techniques, it makes a prediction through the weighted
average that includes content-based prediction and collaborative
filtering prediction. The weights of predictions are determined on a
per-user basis, allowing the system to determine the optimum mixture
of content-based and collaborative recommendation for each user.
Pazzani M.\cite{pazzani1999framework} presents Entree as a hybrid
recommender system that it does not use numeric scores, but rather
treats the output of each recommender (collaborative, content-based
and demographic) as a set of votes, which are then combined in a
consensus scheme. The recommender system includes information such as
the content of the page, ratings of users and demographic data about
users. 
%Hibridos
Others works with hybrid recommender systems are ProfBuilder
\cite{al1999semantic}, PickAFlick\cite{burke1999integrating}  and
\cite{tran2000hybrid}, where are presented multiple recommendation
techniques. Usually, recommendation requires ranking of
items or selection of a single best recommendation, at this point some
technique must be employed to recommend. \\ 
Traditional recommender systems such as above mentioned, tend to use
simple user models. For example, user-based collaborative filtering
generally models the user as a vector of item ratings. As additional
observations are made about users’ preferences, the user models are
extended, and the user preferences is used to generate
recommendations. This approach, therefore, ignores the notion of any
specific situation, the fact that users interact with the system
within a particular context and  that preferences of items might 
change in another context.  
Overall, the context is able to make the recommender system be 
powerful that is adaptable to the changing user's situation.\\
The context is defined in the domain of the application and the system
has a context model that provides the information for the recommender
system. For instance Ricci et.al. \cite{baltrunas2011incarmusic} uses
the context in music domain using a model-based paradigm, in this
context-aware recommender system the context was defined as a set of
independent contextual factors(independent in order to get a
mathematical model) such as \textit{driving style, road type,
landscape, sleepiness, traffic conditions, mood weather and natural
phenomena} to specifies the relevant context for the music
recommendation. In order to estimate the relevance of selected
contextual factor, the users were requested to evaluate music tracks
in different contextual situations for each genre. The prediction
takes in account this relevance to recommend music tracks prefered by
the user according the genre and the contexts mentioned.In
restaurant domain Chung-Hua et al.\cite{chu2013chinese} presents a
context-aware recommender system for mobiles using a post-filtering
paradigm, the architecture involves a model client-server that works
with a request of data in the client side for the server side.
Subsequently, taking in account the contextual factors to filter the
properly restaurants to recommend. The context-aware recommender
system uses such as contextual factors \textit{location and season}, 
also utilize the user preferences to personalize the recommendations
in the user context.
Baltrunas et.al.\cite{baltrunas2011context} presents ReRex for tourism, 
a context-aware recommender system based in a model-based paradigm, the system
recommends and provides explanations about the why the places of
interest(PoI) are recommended. The proposed model computes a
personalized context dependent rating estimation. Subsequently, in
order to generates the explanation of recommendation the system uses
the factor that in the predictive model has the lasgest positive
effect on the rating prediction for the point of interest. The set of
contextual factors considered in ReRex are \textit{distance},
temperature, weather, season, companion, time day, weekday,
crowdedness, familiarity, mood, budget, travel length, transport and
travel goal. The main issue in ReRex system is the low user
satisfaction because of the explanations not able to be understood,
however the users recognize that the explanation is a very important
component that it influence the system acceptance. Noguera et. al.
\cite{noguera2012mobile} presents a context-aware recommender system
for tourism based in REJA that utilizes the location through a 3D-GIS
system, the application uses progressive downloading and rendering of
3D maps over mobiles networks. It is also in charge of tracking the
user’s location and speed based on GPS and the requesting. The system
utilizes pre-filtering and post-filtering paradigm. Pre-filtering is
used to reduce the number of items considered for the recommendation
according to the user’s location, and  post-filtering is applied to
re-rank the previous top-N list according to the physical distance
from the user for each recommended restaurant. The disadvantage in this
system is the lack of user reviews, because the recommendations are
based only in the location point without consider the experience of
other diners concerning the recommended restaurant. 
Cena et al.\cite{cena2006integrating} presents a tourist guide for
context in intelligent content adaptation. UbiquiTO system is a
tourist guide that integrates different forms of context-related
adaptation: for media device type, for user characteristics and
preferences, for the physical context of the interaction. UbiquiTO uses
a rule-based modeling approach to adapt the content of the provided
recommendation, such as the amount, type of information and features
associated with each recommendation. 
Bulander et.al\cite{bulander2005comparison} presents the MoMa-system that
offers proactive recommendations using a post-filtering approach for
matching order specifications with offers. When creating an order, the
client application will automatically fill in the appropriate physical
context and profile parameters, for example, \textit{location} and \textit{weather},
then, for example, the facility should not be too far away from the
current location of the user and beer should not be
recommended if it is raining. On the other side, advertisers’
suppliers put offers into the MoMa-system. These offers are also
formulated according to the catalogue. When the system detects a pair
of context matching order and offer, the end user is notified, in the
preferred manner (for example, SMS, email). At this point, the user
must decide whether to contact the advertiser to accept the offer.
Finally, Schifanella et al.\cite{schifanella2008mobhinter} develops
Mob-Hinter, a \textit{context-dependent} distributed model,where a user device
can directly connect to other mobile devices that are in \textit{physical
proximity} through ad-hoc connections, hence relying on a very limited
portion of the users’ community and just on a subset of all available
data (pre-filtering). The relationships between users are modeled with
a similarity graph. MobHinter allows a mobile device to identify the
affinity network neighbors from random ad-hoc communications. The
collected information is then used to incrementally refine locally
calculated predictions, with no need of interacting with a remote
server or accessing the Internet. The Recommendations are computed
using the availables rating of the user neighbors.
Abowd et. al. presents Cyberguide project \cite{abowd1997cyberguide},
which encompassed several tour guide prototypes for different handheld
platforms. Cyberguide provided tour guide services to mobile users,
exploiting the contextual knowledge of the user’s current and past
locations in the recommendation process. The PECITAS system
\cite{tumas2009personalized} presented by Thumas offers location-aware
recommendations for personalized point-to-point paths. The paths are
illustrated by listing the various connections that the user must take
to reach the destination using public transportation and walking. An
interesting aspect of PECITAS is that, although an optimal shortest 
path facility is incorporated, users may be recommended alternative 
routes that pass through several attractions, given that
their specified constraints (e.g. latest arrival time) and travel-related 
preferences (maximum walking time, maximal number of transport
transfers, sightseeing preferences, etc) are satisfied. Yu and Chang
presents LARS \cite{yu2009personalized} which supports personalized
tour planning using a rule-based recommendation process. This system
packages ‘where to stay’ and ‘where to eat’ features together with
‘typical’ tourist recommendations for sightseeing and activities. For
instance, recommended restaurants (selected based on their location,
menu, prices, customer rating score, etc) are integral part of the
tour and the time spent for lunch/dinner is taken into account to
schedule visits to attractions or to plan other activities.
Savage et. al. presents  "I'm feeling LoCo" system \cite{savage2012m}
that proposes a ubiquitous location­ based recommendation algorithm
that focuses on user experience by considering user preferences, time,
location and similarity measures automatically, having Foursquare as a
dataset. We also focus on user experience and aim that user input is
minimal. The information  om the user's social network, form of
transportation and phone's sensors is inferred to provide
recommendation of places  om the dataset.
Reddy et.al\cite{reddy2006lifetrak} presents LifeTrack system that
incorporates sensor information into song selection. The songs are
represented in terms of tags that the user assigns in order to link
the songs to the appropriate contexts in which they should be played.
User feedback is incorporated to make a song more or less likely to
play in a given context. Context considered relevant to song selection
includes location, time of operation, velocity of the user, weather,
traffic and sound. User locations and velocity are determined by GPS.
Location information includes tags based on zip code and whether the
user is inside or outside (inferred by the presence or absence of a
GPS signal). The times of the day are divided out into configurable
parts of the day (morning, evening, etc). The velocity is abstracted
into one of four states: static, walking, running and driving. Use of
accelerometers are planned to enable indoor velocity information. If
the user is driving, an RSS feed on traffic information is used to
typify the state as calm, moderate or chaotic. If the user is not
driving, a microphone reading is used for the same purpose.
Additionally, an RSS feed provides a meteorological condition (frigid,
cold, temperate, warm or hot).\\ The table \ref{tab:stateoftheart}  
describes examples of contextual factors in different domains 
of application, specifies the contextual factors considered 
such as  part of the context, the methodology for each 
application and  kind of devices.

\begin{sidewaystable}[]
  \caption{Comparison of context-aware recommender systems.}
    \label{tab:stateoftheart}
  \bigskip
    \centering\small\setlength\tabcolsep{2pt}
        \hspace*{-1cm}\begin{tabular}{p{3.5cm} p{6cm} p{4cm} p{3cm} p{3cm} }%{l l l l l}
           \toprule
             \textbf{Application} &\textbf{Contextual Factor} &\textbf{Domain} &\textbf{Paradigm} &\textbf{Device}  \\ \hline

           \midrule
             \textbf{CoMoLE} & \textbf{Time, available time, place, device, level of knowledge, learning style.} & \textbf{E-learning} & \textbf{Pre-filtering} & \textbf{Mobiles, PC, laptop.}   \\ \hline 

             \textbf{Moma-System} & \textbf{Location, time.} & \textbf{E-commerce} & \textbf{Post-filtering} & \textbf{PC, laptop.}  \\ \hline

             \textbf{UbiquITO} & \textbf{Season, time, temperature.} & \textbf{Tourism} & \textbf{Post-filtering} & \textbf{Mobiles} \\ \hline

             \textbf{ReRex} & \textbf{Distance of the point of interest,  temperature, weather, season, weekend, companion, travel goal, transport.} & \textbf{Tourism} & \textbf{Model-based} & \textbf{Mobiles} \\ \hline

             \textbf{LifeTrack} & \textbf{Location, time, day of the week, traifc noise(level), temperature, weather.} & \textbf{Music} & \textbf{ Post-filtering} & \textbf{PC, Mobiles.} \\ \hline

             \textbf{CARS} & \textbf{Location and season.} & \textbf{Restaurants} & \textbf{Post-filtering} & \textbf{PC, laptop.} \\ \hline

             \textbf{InCarMusic} & \textbf{Driving style, road type, landscape, sleepiness, traffic conditions, mood weather and natural phenomena.} & \textbf{Music} & \textbf{Model-based} & \textbf{Mobiles} \\ \hline

            \textbf{REJA} & \textbf{Location.} & \textbf{Restaurants} & \textbf{Pre-filtering and Post-filtering} & \textbf{PC, laptop, mobiles.} \\ \hline

            \textbf{CiberGuide} & \textbf{Location, time, weather.} & \textbf{Tourism} & \textbf{Post-filtering} & \textbf{Mobiles} \\ \hline

            \textbf{PECITAS} & \textbf{Location, routes.} & \textbf{Transport} & \textbf{Post-filtering} & \textbf{Mobiles} \\ \hline

            \textbf{LARS} & \textbf{Tourists’ location and time.} & \textbf{Tourist packages} & \textbf{Post-filtering} & \textbf{Mobiles} \\ \hline

            \textbf{I'm feeling LoCo} & \textbf{Location, transportation.} & \textbf{Tourism} & \textbf{Model-based} & \textbf{Mobiles} \\ \hline

            \textbf{MOPSI} & \textbf{Location} & \textbf{Tourism and transport} & \textbf{Post-filtering} & \textbf{Mobiles} \\ \hline

           \bottomrule
        \end{tabular}\hspace*{-1cm}
\end{sidewaystable}









%% La letra n con tilde es: 'n.
\chapter{Background}\label{background}

In this chapter the fundamental concepts related this work are presented:
Formal definitions referring to fuzzy systems, contextual factors and
recommender system techniques used by the proposed method.
%--------------------------------------------------------------
%Agregue la seccion de logica difusa del articulo que me mando.

\section{Production systems and fuzzy models}

A central aspect of the proposed method is the use of both fuzzy logic and
fuzzy inference systems, in this section formal definitions of
these models of knowledge representation are presented.  

\subsection{Traditional Production Systems}
Production Systems represent knowledge in form of rules, which specify
actions that will be executed when certain conditions are met. In these 
systems Experts in a certain domain identify a set of rules based 
on their experience to
resolve different kinds of problems. Also known as rule based systems,
many implementations consist of mainly these three
components \cite{brachman1992knowledge} \cite{konar2006computational}:
\begin{enumerate}   
\item \textbf{Production Rules (PR)}. A set of
production rules (also known as \textit{IF-THEN} rules) having a two part
structure; the antecedent, conformed by a set of conditions and a
consequent set of actions. 
\item \textbf{Working Memory (WM)}.
Represents the current knowledge or facts that are known to be true so
far. These facts are tested by the antecedent conditions of the rules
and the consequent part can change them. 
\item \textbf{Inference Engine (IE)}. 
This interpreter matches the conditions in the
production rules with the data/instantiations found in the WM,
deriving new consequences.
\end{enumerate}
The basic operation of these systems is described as a cycle of 
three steps \cite{brachman1992knowledge}:
\begin{enumerate}
\item \textbf{Recognize}: Find which rules are satisfied by 
the current WM. The antecedent part of the productions consists 
of a set of clauses connected by AND operators, when all these 
clauses have matching data on the WM the production has a chance 
of firing.
\item \textbf{Conflict Resolution}: Only one production can be 
fired at a time, so when two or more rules can be fired concurrently 
a conflict occurs. Among the production rules found in the first 
step, choose which rules should fire.
\item \textbf{Actions}: Change the working memory by performing 
the actions specified in the consequent part of all the rules 
selected in the second step. Changes occur by adding or 
deleting elements of the WM.
\end{enumerate}
This cycle continues until no further production rules can be fired.
This control strategy is data driven because whenever the antecedent
part is satisfied the rule is recognized, this strategy is also named
chain-forward. Other strategy is chain-backward in which case the work
is done from the conclusion to the facts, to chain-backward, goals in
the WM are matched against consequents of the production
rules.\\A drawback that has been recognized in these traditional
productions systems, is that some times rules are not fired in the
Recognize step because no appropriate match exists in the WM. Partial
matching of rules is not possible and this can be a limitation in some
systems because premature termination of the cycle is not desired. An
approach to handle partial matching is using fuzzy logic
\cite{konar2006computational}. In the next section a review of the
extension of production systems with fuzzy logic is presented.\\

\subsection{Fuzzy Production Rules}

Fuzzy production rules use fuzzy logic sets to characterize the
variables and terms used in the propositions of the rules. Fuzzy
production rules or fuzzy \textit{IF-THEN} rules are expressions of
the form \textit{IF} antecedent \textit{THEN} consequent, where the
antecedent is a proposition of the form \textit{"x is A"} where
\textit{x} is a linguistic variable and \textit{A} is a linguistic
term. The truth value of this proposition is based on the matching
degree between \textit{x} and \textit{A}. Propositions are connected
by \textit{AND}, \textit{OR} and \textit{NOT} operators. Some
implementations of fuzzy rule-based systems also include other kinds
of data types in their propositions, for example the FLOPS system
includes fuzzy numbers, hedges, and non fuzzy data types (integers,
strings and float) \cite{siler2005fuzzy}. Depending on the form of the
consequent, two main types of fuzzy production systems are
distinguished \cite{babuvska1996fuzzy}:
\begin{itemize}  
\item \textbf{Linguistic fuzzy model}: where both the antecedent 
and consequent are fuzzy propositions.
\item \textbf{Takagi-Sugeno fuzzy model}: the antecedent is a fuzzy 
proposition; the consequent is a crisp function.
\end{itemize}  
As before, other non-fuzzy consequents can also be implemented, like
the execution of commands or the addition of new data.\\
\textbf{Linguistic Variables (LV)} are variables that can be assigned
linguistic terms as values, i.e. if we define a linguistic variable
\textit{SPEED} we can assign it the linguistic terms \textit{SLOW},
\textit{MEDIUM} or \textit{FAST}. The meaning of these linguistic
terms is defined by their membership functions (MF). \textit{LV} can
be defined as a \textit{5-tuple} \textit{LV=}$<v,T,X,g,m>$ where
\textit{v} is the name of the variable, \textit{T} is the set of
linguistic terms of \textit{v}, \textit{X} is the domain (universe) of
\textit{v},\textit{g} is a syntactic rule to generate linguistic terms,
\textit{m} is a semantic rule that assigns to each term \textit{t} its
meaning \textit{m(t)}, which is a fuzzy set defined in \textit{X}.

\subsection{Fuzzy Inference Systems}

\textit{Fuzzy Inference Systems} (FISs) also called \textit{Fuzzy
Models} are fuzzy production systems used for modeling input-output
relationships. From this input-output view, Babuŝka
\cite{babuvska1996fuzzy} describes these systems as \textit{``flexible
mathematical functions which can approximate other functions or just
data (measurements) with a desired accuracy"}. Fuzzy Productions Rules
define the relationship between input and output variables. Input
variables are defined in the antecedent part of the rule and the
consequent part defines the output variables. These FISs are used
mainly in control systems, and are basically composed of five
modules\cite{babuvska1996fuzzy}:
\begin{enumerate}  
\item \textbf{Rule Base.} The set of fuzzy production rules.
\item \textbf{Database.} Where the membership functions are defined.
\item \textbf{Fuzzy Inference Engine.} This module executes the 
fuzzy inference operations.
\item \textbf{Fuzzifier.} This interface transforms the inputs 
of the systems (numerical data) into linguistic values.
\item \textbf{Defuzzifier.} This interface transforms the fuzzy 
results into numerical data.
\end{enumerate}
Usually the Rule Base and Data Base modules are collectively 
called the Knowledge Base module. The steps involved in fuzzy 
inference in a FIS are \cite{dubois1980fuzzy}:
\begin{enumerate} 
\item Compare the input variables with the membership functions 
in the antecedent, to obtain the membership values of each 
linguistic term. This step is frequently called fuzzification.
\item Compose through a specific T-Norm operator (mainly max-min 
or max-product) the membership values to obtain the degree of 
support of each rule.
\item Generate the qualified consequence (fuzzy or numeric) of 
each rule depending on the degrees of support. These outputs 
are then aggregated to form a unified output.
\item Then the output fuzzy set is resolved or defuzzified 
to a single numeric value.
\end{enumerate} 
Three main inference systems can be described:
\begin{itemize} 
\item \textbf{Tsakumoto}: The output is the average of the 
weights of each rule numeric output, induced by the degree of 
support of each rule, the min-max or min-product with the 
antecedent and the membership functions of the output. The 
membership functions used in this method must be 
non-decrease monotonic. 
\item \textbf{Mamdani}: The output is calculated by applying 
the min-max operator to the fuzzy output (each equal to the 
minimum support degree and the membership function of the rule). 
Several schemes have been proposed to choose the numeric output 
based on the fuzzy output; these include the centroid area, 
area bisection, maximum mean, maximum criteria.
\item \textbf{Sugeno}: The fuzzy production rules are used. The 
output of each rule is a linear combination of the input 
variables plus a constant term, and the output is the average 
of the support degree of each rule.
\end{itemize} 
%%-----------------------------------------------------------


\section{Context}
People transmit ideas to each in a complex way. This
is due to many factors such as: the richness of the language shared, the common
understanding of how the world works, and an implicit understanding of
situations in daily life. When people talk, they are able to use implicit
situational information (contextual information), to increase the
conversational bandwidth. \\Unfortunately, this ability to transmit
ideas does not transfer well to persons interacting with computers. In
traditional interactive computing, users have poor mechanisms for
providing input to computers. Consequently, computers are not
currently enabled to take full advantage of the context of the 
human-computer dialogue. By improving the computer's access to context, 
we increase the richness of communication in a human-computer interaction
enabling the development of more useful computational 
services.\\
In order to use context effectively, we must define what context
is and how it can be used. An understanding of \textit{how context can
be used} will help application designers to determine what 
context-aware behaviours to use in applications\cite{dey2001understanding}.\\
To stablish a specific definition of \textit{context} that can be used
in the \textit{context-aware} computing field, is necessary to review
how researchers define the context in their own work. Schilit and
Theimer\cite{abowd1999towards} refer to context as \textit{location},
\textit{identities of nearby people and objects}, and \textit{changes
to those objects}.\\
This type of definitions that define context by example
are difficult to apply when developers try to determine whether a type of
information not listed in the definition is part of the context or not, 
as it is not clear how it can be used by the definition.\\ 
Schilit et al.\cite{schilit1994context} affirms that the most important
aspects of context are: \textit{where you are}, \textit{who you are
with}, and \textit{what resources are nearby}.
Pascoe\cite{pascoe1998adding} defines context to be the
\textit{``subset of physical and  conceptual states of interest to a
particular entity"}.\\
These definitions are too general, context is all about the
whole situation relevant to an application and its set of users. It is
complicated enumerate which aspects of all situations are important,
as this will change from situation to situation. For this reason and
for the purpose of this thesis, the definition of context
proposed by Dey\cite{dey2001understanding} has been adopted (see section 
\ref{context-awareness}). \\However, another important aspect 
is to stablish a meaningful classification that covers the 
characteristics that describe the contextual factor.\\
Dourish\cite{dourish2004we} has distinguished between two different
views of context: the \textit{representational view} and the
\textit{interactional view}. The \textit{representational view} makes
four key assumptions: context is a \textit{form of information}, it is
\textit{delineable}, it is \textit{stable}, and it is
\textit{independent} from the underlying activity. In this view,
context can be described using a set of observable attributes that are
known a priori. Furthermore, the structure of these contextual
attributes does not change over time. The \textit{interactional view},
takes a different stance on the key assumptions made by the
representational view. In the interactional view, the scope of
contextual features is defined dynamically, and it is occasioned
rather than static. Rather than assuming that context acts as a set of
conditions under which an activity occurs, this view assumes a
cyclical relationship between context and activity, where the activity
gives rise to context and the context influences activities.\\
Context should include information to allow systems to use contextual
information about users and their situation, enabling the system 
to provide users personalised and contextual services. The importance of
context lies in the  assumption of the influence of \textit{contextual
factors} that matter for users when they decide, choose or discard an
item.\\
In the real world, the context in a situation is involved in the
\textit{environment} of the people, the \textit{entities} belong at
the \textit{situational information}, but an entity  becomes in a
\textit{contextual factor} when its information \textit{affects} the
recommendation process, therefore, the entity and its values of 
domain will be involved in the process such as a contextual factor.\\
The domain values of a contextual factor change over time, in
real life the situation occurs when we decide that, for instance,
we like a kind of clothes and the next day, for a any reason we don't
like it anymore. As for the representation of the \textit{"change of time"} , a data 
model of \textit{time} should be specified in a way that the system
\textit{interprets} time as a data structure (for instance weeks, 
days, hours, minutes, seconds, etc.). \\
Assuming the existence of certain contextual factors such as
\textit{time}, \textit{location} and \textit{purchasing purpose} that
are identified in the context of recommendations,
Adomavicious\cite{adomavicius2011context} proposes two important
aspects that highligh when different kinds of context are defined:
\textit{what a recommender system may know about these contextual
factors} and, \textit{how contextual factors change over time}.\\ \\

A recommender system can have different types of knowledge, which may
include  the exact list of all the relevant factors, their structure,
and their values, about the contextual factors. Depending on what
exactly the system knows (that  is, what is being observed),
Adomavicious categorizes the knowledge of a recommender system about
the context as the following:
	\begin{itemize}
	\item \textbf{Fully observable}: The contextual factors relevant to the 
	application, as well as their structure and their values at the time when 
	recommendations are made, are known explicitly. For example, when
	recommending the purchase of a certain product, like a shirt, the 
	recommender system may know that only the \textit{Time}, \textit{PurchasingPurpose}, 
	and \textit{ShoppingCompanion} factors matter in this application. Further more, 
	the recommender system may know the structure of all three contextual 
	factors, such as having categories of \textit{weekday}, \textit{weekend}, 
	and \textit{holiday} for \textit{Time}. Further, the recommender system 
	may also know the values of the contextual factors at the recommendation 
	time, for instance, \textit{when this purchase is been made}, 
	\textit{with whom}, and \textit{for whom}.
	\item \textbf{Partially observable}: Only some of the information about 
	the contextual factors described above, is explicitly known. For example, 
	the recommender system may know all the contextual factors, such as Time, 
	PurchasingPurpose, and ShoppingCompanion, but not their structure. Note that 
	there can possibly be different levels of \textit{"partial observability"}. 
	\item \textbf{Unobservable}: No information about contextual factors is 
	explicitly available to the recommender system, and it makes recommendations 
	by utilizing only the latent knowledge of context in an implicit manner. 
	For example, the recommender system may build a latent predictive model, 
	such as hierarchical linear or hidden Markov models, to estimate unknown 
	ratings, where unobservable context is modeled using latent variables.
	\end{itemize}
\textbf{How contextual factors change over time.} Depending on whether 
contextual factors change over time or not, two categories are proposed: 
	\begin{itemize}
	\item \textbf{Static}: The relevant contextual factors and their structure
	remain the same (stable) over time. For example, when recommending the
	purchase of a certain product, such as a shirt, we can include the
	contextual factors of Time, PurchasingPurpose and ShoppingCompanion 
	without change during the entire lifespan of the purchasing recommendation
	application.
	\item \textbf {Dynamic}: This is the case when the contextual factors 
	change in some way. For example, the recommender system (or the 
	system designer) may realize over time that the \textit{ShoppingCompanion} 
	factor is no longer relevant for purchasing recommendations and may 
	decide to drop it. Furthermore, the structure of some of the contextual
	factors can change over time, for instance, new categories can be
	added to the \textit{PurchasingPurpose} contextual factor over time.
	\end{itemize}
On the other hand, Fling\cite{fling2009mobile} generalizes four types of
contexts that can be used in different applications:  
\begin{itemize}  
\item \textbf{Physical context}: representing the time, position, and
activity of the user, but also the weather, light, and temperature
when the recommendation is supposed to be used.  
\item \textbf{Social context}: representing the presence and role 
of other people (either using or not using the application) around 
the user and whether the user is alone or in a group when using 
the application. 
\item \textbf{Interaction media context}: describing the device used to
access the system (for example, a mobile phone or a kiosk) as well as
the type of media that are browsed and personalized. The latter can be
ordinary text, music, images, movies, or queries made to the
recommender system.  
\item \textbf{Modal context}: representing the current state 
of mind of the user,  the user's goals, mood, experience, 
and cognitive capabilities. 
\end{itemize} 
Subsequently the revision of the literature of context, it is important
to mention a formal definition that describes what features it has a
context-aware system, this definition is proposed by
Dey\cite{dey2001understanding}: \textit{``a system is context-aware if
it uses context to provide relevant information and/or services for
the user, where relevancy  depends on the user's task."}\\ This
definition is closer  to the reality about behaviour of \textit
{context-aware recommender system} when incorporates contextual
information.\\  
Based in this definition, Dey proposes some characteristics 
that a context-aware application should be support:
\begin{itemize}  
\item \textbf{Presentation of information} and services to a user.
\item \textbf{Automatic execution} of a service for a user.
\item \textbf{Tagging of context} to information to support later retrieval.
\end{itemize} 
An example to explain the context in a context-aware application, for
instance, it can be an indoor mobile tour guide. Here, the entities
are the user, the application and the tour sites. We will look at two
pieces of information (weather and the presence of other people) and
use the definition to determine if either one is context. The weather
does not affect the application because it is being used for indoor activities.
Therefore, it is not context. The presence of other people, however,
can be used to characterize the user’s situation. If a user is
traveling with other people, then the sites that they visit may are
the points of interest for the user. Therefore, the presence of other
people is context because it can be used to characterize the user's
situation.

\section{Recommender systems}

\subsection{Collaborative Filtering algorithm}

The idea behind collaborative recommendation approaches is to exploit
information about past behavior or opinions of an exisiting user
community  for predicting which items certain user of the system will
most probably like or be interested in\cite{jannach2010recommender}. 
Recommender systems are useful in several types of  applications,
however, their biggest impact has been mainly in ecommerce  web sites
in order to personalize the information for a particular user as the
system can help to promote several items of his or her interest, thus
increasing the sales of the on-line store. In traditional
implementations a collaborative filtering algorithm takes as
input a given \textit{user-item} matrix of ratings to generate a
prediction for each item-user pair indicating to what degree the
current user will like or dislike an item. Subsequently with that
information a list of the top \textit{n} recommended items for the
user can be generated. The generated list contains only those items
that have not been reviewed by the user. Differents approaches are
utilized for collaborative filtering such as: a) user-based nearest 
neighbor recommendation, b) Item-based nearest neighbor 
recommendation and c) model-based recommendation.\\
\textbf{a) User-based nearest neighbor} is an approach that only 
needs the rating matrix to obtain recommendations. 
The neighborhood selection consists in taking
the \textit{k}  nearest neighbors into account usind the threshold to
define the size of the neighborhood. A neigborhood of small can not
make accurate predictions, and on the other hand if the neighborhood
is too large the information about the nighbours could not be
significant.\\ To obtain the similarity value between a user and his
neighbors, the Pearson correlations measure is commonly used, taking
the values from $+1$ (strong positive correlation) to $-1$ (strong
negative correlation) to define how similar a neighbor is. The
similarity $sim(a,b)$ of users $a$ and $b$, given the rating matrix
$R$ is denoted by the following equation:
\begin{equation}\label{eq:pearson1}
\displaystyle sim(a,b) = {\sum_{p \in P}(r_{a,p} - 
\bar{r_a})(r_{b,p}- \bar{r_b}) 
\over \sqrt{\sum_{p \in P}(r_{a,p} - \bar{r_a})^2} 
\sqrt{\sum_{p \in P} 
(r_{b,p}- \bar{r_b})^2}}
\end{equation}
Where the symbol $\bar{r_a}$ corresponds to the average rating of user
$a$. Subsequently, a formula to calculate the prediction of the user
$a$ for item $p$ that also factors the relative proximity of the
nearest neighbors $N$ and $a's$ average rating $\bar{r_a}$ is denoted
by the following equation:
\begin{equation}\label{eq:prediction}
\displaystyle pred(a,b) = \bar{r_a} + 
{\sum_{b \in N} sim(a,b) * (r_{b,p}- \bar{r_b}) 
\over \sum_{b \in N} sim(a,b)} 
\end{equation}
\textbf{b) Item-based nearest neighbor} is the same idea than the \textit
{user-based}, the difference is that this approach tries to find
similar items instead of similar users to make a prediction using the rating
matrix. Then, in a \textit{item-based} recommendation is to compute
predictions using the similarity between items and not the similarity
between users. To find similar items cosine similarity measure is
defined, this metric measures the similarity between two
\textit{n-dimensional} vectors based on the angle between them.
Therefore, the similarity between two items \textit{a} and \textit{b}
viewed as the corresponding rating vectors $a$ and $b$, is formally
defined as follows:
\begin{equation}\label{eq:cosine}
\displaystyle sim(\overrightarrow{a},\overrightarrow{b})= 
{\overrightarrow{a}* \overrightarrow{b} \over
|\overrightarrow{a}|*|\overrightarrow{b}| }
\end{equation}
Where the * symbol is the dot product of vectors and $|a|$ is the Euclidian
length of the vector, which is defined as the square root of the dot
product of the vector with itself.\\

\textbf{c) Model-based approach}, in this technique  the raw data are
first processed off-line, as described for \textit {item-based}
filtering or some dimensionality reduction techniques. At run time,
only the learned model is required to make predictions. Although
\textit{memory-based approach} is theoretically more precise because
full data is available for generating recommendations, such systems
face problems of scalability when databases of tens of millions of
users and items are used. An example of this approach is
\textit{matrix factorization} or \textit{latent factors model},
normally used to fill a rating matrix to calculate predictions taking
in account the \textit{latent factors}.

\subsubsection{Data sparsity and cold-start problem}\label{coldstart}

In real-world applications, the ratings matrix tend to be \textit{very
sparse}(sparcity problem), as customers typically provide  ratings
for (or have bought) only a small fraction of the catalog items. In
general, the challenge in that context is thus to compute good
predictions when there are relatively few ratings available. One
straightforward option for dealing with this problem is to exploit
additional information about the users, such as gender, age,
education, interests, or other available information that can help to
classify the user. The set of similar users (neighbors) is thus based
not only on the analysis of the explicit and implicit ratings, but
also on information external to the ratings matrix. These hybrid
systems \cite{pazzani2007content}, however, no longer
\textit{“purely”} collaborative, and new questions of how to acquire
the additional information and how to combine the different
classifiers arise. Still, to reach the critical mass of users needed
in a collaborative approach, such techniques might be helpful in the
\textit{ramp-up phase} of a newly installed recommendation service. \\
The \textit{cold-start problem} can be viewed as a special case of the
sparsity \cite{huang2004applying}. The questions here are (a)\textit{how 
to make recommendations to new users that have not rated
any item yet} and (b)\textit{how to deal with items that have not
been rated or bought yet}. Both problems can be addressed with the
help of hybrid approaches \cite{adomavicius2005toward}.  To face the
\textit{new-users problem}, one option could be to ask the user for a
minimum number of ratings before the service can be used. In this
situation the system could intelligently ask for ratings for items
that, from the view point of information theory, carry the most
information\cite{rashid2002getting}. A similar strategy of asking
the user for a gauge set of ratings is used for the Eigentaste
algorithm presented in \cite{goldberg2001eigentaste}.

\subsection{Content-based algorithm}

In content-based the recommendation task then consists of determining
the items that match the user’s preferences best. Although such an
approach must rely on additional information about items and user
preferences, it does not require the existence of a large user
community or a rating history, i.e., recommendation lists can be
generated even if there is only one single user. \\In practical
settings, technical descriptions of the features and characteristics
of an item (such as the genre of a book or the list of actors in a
movie) are more often available in electronic form, as they are
partially already provided by the providers or manufacturers of the
goods. What remains challenging, however, is the acquisition of
subjective, qualitative features. \\In domains of quality and taste, for
example, the reasons that someone likes something are not always
related to certain product characteristics and may be based on a
subjective impression of the item’s exterior design 
\cite{jannach2010recommender}.   

\subsubsection{Content representation} 

The simplest way to describe catalog
items is to maintain an explicit list of features for each item (also
often called attributes, characteristics, or item profiles). For a
book recommender, one could, for instance, use the genre, the author’s
name, the publisher, or anything else that describes the item and
store his information in a relational database system. When the user’s
preferences are described in terms of his or her interests using
exactly this set of features, the recommendation task consists of
matching item characteristics and user preferences 
\cite{jannach2010recommender}.  

\subsubsection{Vector space model}  

Content-based systems have historically
been developed to filter and recommend text-based items such as e-mail
messages or news. The standard approach in CB recommendation is,
therefore, not to maintain a list of \textit{meta-information
features}, but to use a list of relevant keywords that appear within
the document. The main idea, of course, is that such a list can be
generated automatically from the document content itself or from a
free-text description thereof \cite{jannach2010recommender}.

\subsubsection{Overspecialization and cold-start problem}
\label{overspecialization}

Learning-based methods quickly tend to propose
more of the same, that is, such recommenders can propose only items
that are somehow similar to the ones the current user has already
(positively) rated. This can lead to the undesirable effect that
obvious recommendations are made and the system, for instance,
recommends items that are \textit{too similar to those the user already knows}.
\\A typical example is a news filtering recommender that proposes a
newspaper article that covers the same story that the user has already
seen \textit{in another context}. The system in \cite{billsus1999personal} 
defines a threshold to filter out not only items that
are \textit{too different} from the profile but also those that are \textit{too
similar}. A general goal to face the \textit{overspecialization},
therefore is to increase the serendipity of the recommendation lists
that includes “unexpected” items in which the user might be
interested, because expected items are of little value for the user.\\
The \textit{cold-start problem}, which we discussed for collaborative
systems, also exists in a slightly different form for content-based
recommendation methods. Although content-based techniques do not
require a large user community, they require at least an initial set
of ratings from the user, typically a set of explicit “like” and
“dislike” statements. \\In all described filtering techniques,
recommendation accuracy improves with the number of ratings;
significant performance increases for the learning algorithms were
reported in \cite{pazzani1997learning}  when the number of ratings was
between twenty and fifty. \\However, in many domains, users might not be
willing to rate many items before the recommender service can be
used. In the initial phase, it could be an option to ask the user to
provide a list of keywords, either by selecting from a list of topics
or by entering free-text input.

\subsection{Hybrid recommender systems} 

Each recommender system technique has its pros and cons, for
instance, the ability to handle data sparsity and cold-start problems
or considerable efforts for knowledge acquisition and engineering. \\
User models and contextual information, community and product data,
and knowledge models constitute the potential types of recommendation
input. However, none of the basic approaches is able to fully exploit
all of these. Consequently, building hybrid systems that combine the
strengths of different algorithms and models to overcome some of the
afore mentioned shortcomings and problems has become the target of
recent research. Hybrid recommender systems are technical approaches
that combine several algorithms or recommendation components
\cite{jannach2010recommender}.

\subsection{Context-aware recommender systems}

Traditionally, the recommendation problem has been viewed as a
prediction problem in which, given a user profile and a target item,
the recommender system's task is to predict that user's rating or that
item, reflecting the degree of user's preference for that 
item\cite{jannach2010recommender}. \\
Specifically, a recommender system tries to estimate a rating
function: $R$ : $Users * Items$ $ \rightarrow Ratings$, that maps
\textit{user-item} pairs to an ordered  set of rating values.\\
In contrast to the traditional model, context-aware recommender system
tries to incorporate or utilize additional evidence (beyond
information about users and items) to estimate user preferences on
unseen items.\\ When such contextual evidence can be incorporated as
part of the input to the recommender systems, the rating function can
be viewed as \textit{multidimensional}: $R$ : $Users * Items *
Contexts$ $ \rightarrow Ratings$, where \textbf{contexts} represents a
\textbf{set of factors} that further delineate the conditions under which the
\textit{user-item} pair is assigned a particular rating. \\ The
underlying assumption of this extended model is that user preferences
for items are not only a function of items themselves, but also a
function of the context in which items are being
considered\cite{lim2009assessing}. \\
A multidimensional model of data warehousing\cite{kimball2011data} is
used to depics the context dimensions, in figure
\ref{fig:multidimension} the time dimension belongs to the \textbf{set of 
contextual factors} and, is described as a \textbf{set of
attributes}, for instance it may consist of
attributes as \textit{morning}, \textit{evening},  \textit{nigth},
etc., such as it was mentioned in section  \ref{contextofuse}. 
\begin{figure*}
\captionsetup{font=footnotesize}
\centering
\includegraphics[width=0.40\textwidth]{img/multidimension.png}
\small
\caption{Multidimensional model of context.}
\label{fig:multidimension}   
\end{figure*}

\subsection{Paradigms for using of contextual information}

When recommender system uses the contextual information, it starts
with the data having the form \textit{U * I * C * R}, where \textit{C}
is additional contextual dimension and end up with a list of
contextual recommendations $i_{1}$,$i_{2}$,$i_{3}$...$i_{n}$ for each
user. However, when the recommendation process does not take into
account  the contextual information, is posible to apply the
information about the current (or desired) context \textit{c} in
various stages of the recommendation process.
Adomavicious\cite{adomavicius2011context} defines three paradigms for
the context-aware recommendation process that is based on contextual
user preference:
\begin{itemize}
\item \textbf{Contextual pre-filtering (or contextualization of
recommendation input).} The approach uses contextual information to
select the most relevant 2D (Users x Items) data for generating
recommendations. One major advantage of this approach is that it
allows deployment of any of the numerous traditional recommendation
techniques previously proposed in the literature\cite{adomavicius2005toward}.
In particular, when using this approach, context
\textit{c} essentially serves as a query (or a filter) for selecting
relevant rating data. An example of a contextual data filter for a
movie recommender system would be: if a person wants to see a movie on
Saturday, only the Saturday rating data is used to recommend movies.
Note that this example represents an exact pre-filter because the data
was filtered using exactly the specified context (figure
\ref{fig:paradigms}.a).
\item \textbf{Contextual post-filtering (or contextualization of
recommendation output).} In this approach ignores context information
in the input data when generating recommendations, that is, when
generating the ranked list of all candidate items from which any
number of \textit{top-N} recommendations can be made. Instead,  the
contextual post-filtering approach uses contextual information to
adjust the obtained recommendation list for each user. The
recommendation list adjustments can be made by: (1) filtering out
recommendations that are irrelevant in a given context, or (2)
adjusting the ranking of recommendations in the list. For example, in
a movie recommendation application, if a person wants to see a movie
on a weekend, and on weekends he or she only watches comedies, the
system can filter out all noncomedies from the recommended list
(figure \ref{fig:paradigms}.b).
\item \textbf{Contextual modeling (or contextualization of
recommendation function).} This approach uses contextual information
directly in the recommendation function as an explicit predictor of a
user's rating for an item and, thus, gives rise to truly
multidimensional recommendation functions representing either
predictive models (such as decision trees, regressions, and so on) or
heuristic calculations that incorporate contextual information in
addition to the user and item data (figure \ref{fig:paradigms}.c).\\
\end{itemize}
\begin{figure*}
\captionsetup{font=footnotesize}
\centering
\fbox{\includegraphics[width=0.80\textwidth]{img/paradigms.png}}
\small
\caption{Paradigms for incorporating context in recommender 
systems\cite{adomavicius2011context}.}
\label{fig:paradigms}   
\end{figure*}








\chapter{Proposed method}\label{method}

\section{Data models} 

The data models was implemented in PostgreSQL database, 
the information in context-aware recommender system was
managed in a scheme of a relational database. Technical support 
about installations of dependencies and the application are 
mentioned in appendix \ref{appendixc}.

\subsection{Restaurant model} 

An effective on-line recommender system must be based upon an
understanding of consumer  preferences and successfully mapping
potential products into the consumer's
preferences\cite{adomavicius2011context}. Pan and
Fesenmaier\cite{pan2006online} argued that this can be achieved
through the understanding of how consumers describe in their own
language a product, a place, and the experience when they are
consuming the product or visiting the place.\\ Traditionally, choosing
a restaurant has been considered as rational behavior where a number
of attributes contribute to the overall usefulness of a restaurant.
For example: food type, service quality, atmosphere of the restaurant,
and availability of information about a restaurant, plays an important
role at different stages in consumer's desitions
making\cite{auty1992consumer}. While food quality and food type have
been perceived as the most important variables for consumers'
restaurant selection, situational and contextual factors have been
found to be important also. Due to this in
Kivela\cite{jack1997restaurant} identifies four types of restaurants:
1) \textit{fine dining/gourmet}, 2) \textit{theme/atmosphere}, 3)
\textit{family/popular}, and 4) \textit{convenience/fast-food}; and
Auty\cite{auty1992consumer} identifies four types of dining out
occasions: 1) \textit{namely celebration}, 2) \textit{social
occasion}, 3) \textit{convenience/quick meal}, and 4) \textit{business
meal}.\\ Taking in account the context, the restaurant model proposed
for context-aware recommender system was definded with 55 attributes
about the restaurants features. An exploration about contents of
models of others works were compared to define the suitable
information into the model. Therefore, the restaurant model is a
binary vector with the following contextual attributes: 1)
\textit{price range}, 2) \textit{payment type}, 3) \textit{alcohol
type}, 4) \textit{smoking area}, 5) \textit{dress code}, 6)
\textit{parking type}, 7) \textit{installations type}, 8)
\textit{atmosphere type}, and 9) \textit{cuisine type}. An example of
restaurant model in the context-aware recommender system is depicted
in figure \ref{fig:restaurantmodel} with some domain values of the
context represented by a binary vector where \textit{1} means that the
restaurant has the property that corresponds to the position value.
Additionally, the restaurant model contains contextual information
such as \textit{users's reviews}, \textit{ratings average}, and
\textit{geographycal location.}\\
\begin{figure*}
\captionsetup{justification=centering,margin=2cm,font=footnotesize}
\centering
\setlength\fboxsep{0pt}
\setlength\fboxrule{0.7pt}
\fbox{\includegraphics[width=10cm,height=10cm,keepaspectratio]
{img/restaurant-model.png}}
\caption{User interface of the restaurant model.}
\label{fig:restaurantmodel}       
\end{figure*}

\subsubsection{Data model} \label{datamodelsection}   

The data model in postgreSQL is depicted in the figure
\ref{fig:restaurantmodeldata}, the model contains the restaurant
entity and its attributes. The restaurant entity is related to
\textit{Item entity} in a \textit{``one-to-one"} relation that at the
same time is related to the \textit{RecommenderRule entity} which
specifies some restrictions for item recommendations. A large view of
all the entities related is depicted in the whole scheme refered in
figure \ref{fig:datamodel}.
\begin{figure*}
\captionsetup{justification=centering,margin=2cm,font=footnotesize}
\centering
\includegraphics[width=10cm,height=10cm,keepaspectratio]{img/data-resmodel.png}
\caption{The data model of restaurant.}
\label{fig:restaurantmodeldata}     
\end{figure*}
Some related entity corresponds to the proposed contextual factors, 
that are defined as following: 
\begin{itemize}
\item \textbf{Price:} \textit{cheap, regular, expensive, too expensive.}
\item \textbf{Payment:} \textit{credit/debit card, cash.}
\item \textbf{Alcohol:} \textit{no alcohol, wine-beer.}
\item \textbf{Smoking area:} \textit{yes, no.}
\item \textbf{Dresscode:} \textit{casual, informal, formal.}
\item \textbf{Installations:} \textit{garden, terrace, indoor, outdoor.}
\item \textbf{Atmosphere:} \textit{relax, familiar, friends, bussines, romantic.}
\item \textbf{Parking:} \textit{no parking, free parking, valet parking.}
\item \textbf{Cuisine:} \textit{japanese, chinese, italian, argentinean,
cantonese, mandarin, mongolian, arabic, greek, spanish, brasilian,
swiss, szechuan, asian, international, steak grill,vegetarian,
natural/healthy/light, traditional mexican, tacos, mediterranean,
middle eastern, american/fast food, gourmet, pizza, bar/beer, tapas
cafe/bar, french, birds, seafood.}
\end{itemize}
The cuisines were defined according the food variety of restaurants in
Tijuana, there are 30 types of cuisines defined in the system. \\
The smoking area is an attribute with boolean value, it
defines if a restaurant has a smoking area in its installation.

\subsection{User model} 

The user's profile is derived from the ratings matrix. Let $U=[u_1,u_2,...u_n]$
the set of all users and $ I=[i_1,i_2,$...$i_n] $ the set of all items, if $R$
represent the ratings matrix,  an element  $R_{u,i}$ represents a user’s rating
$u \in U$  in a item $i \in I$.  The unknown ratings are denoted as $\neq $. The
matrix $R$ can be decomposed into rows vectors, the row vector is denoted as $
\overrightarrow{r_u} $=$[R_{u,1}$...$R_{u,|I|}]$ for every $u \in U$. Therefore,
each row vector represents the ratings of a particular user over the items. Also
denote a set of items rated by a certain user u is denoted as $ I_u = i \in I |
\forall  i: R_{u,i} \neq \emptyset $. This set of items rated represents the
user preferences where for each domain element $R_{u,i} \in [1-5]$ represents
the intensity of the user interest for  the item.\\  
\begin{figure*}
\captionsetup{justification=centering,margin=2cm,font=footnotesize}
\centering
\setlength\fboxsep{0pt}
\fbox{\includegraphics[width=10cm,height=10cm,keepaspectratio]
{img/user-profile.png}}
\caption{Example of user interface for user profile.}
\label{fig:user-profile}      
\end{figure*}
In context-aware recommender system, user profile has contextual
information such as: 1) price range, 2) current location, 3) cuisine
types, 4) attributes or features of restaurants that the user want, 5)
the reviews posted, and 6) the favorite restaurants list. The user
profile is stored in database and it is available for queries request,
and it can be changed by users many times in a session. The
information used to recommendations is the last one register stored.
The user interface is represented in figure \ref{fig:user-profile}.

\subsubsection{Data model} 

The user's data model in postgreSQL is represented in the figure
\ref{fig:datausermodel}, the model involves the entities:
\textit{User, UserProfile, and Friends.} \textit{UserProfile entity}
provides the contextual information of user, \textit{User entity} is
the default model defined in the system and is related to userProfile
for suplies valuable information. The \textit{Friends entity}
represents the social aspect into the userProfile, Friends involves
the users related to the current user taking in account the
preferences of each other. \\The user profile entity is related with: 
price and cuisine are the same that in restaurant model,
attribute groups corresponds to restaurant model mentioned (section
\ref{datamodelsection}). A total of 55 attributes(or characteristics) could
be contained in user profile, this information is used such as 
contextual information also. The domain values are following:
\begin{itemize}
\item \textbf{Price:} cheap, regular, expensive, too expensive.
\item \textbf{Cuisine:} japanese, chinese, italian, argentinean,
cantonese, mandarin, mongolian, arabic, greek, spanish, brasilian,
swiss, szechuan, asian, international, steak grill,vegetarian,
natural/healthy/light, traditional mexican, tacos, mediterranean,
middle eastern, american/fast food, gourmet, pizza, bar/beer, tapas
cafe/bar, french, birds, seafood.
\item  \textbf{Attribute groups:} Installations, atmosphere, 
parking, payment, smoking area, dresscode, alcohol.
\end{itemize}
\begin{figure*}
%\captionsetup{justification=centering,margin=2cm}
\captionsetup{font=footnotesize}
\centering
\includegraphics[width=8cm,height=8cm,keepaspectratio]
{img/data-usermodel.png}
\caption{The data model of user profile.}
\label{fig:datausermodel}     
\end{figure*}

\subsection{Relational data model} 

A complete database relational scheme is represented in the figure
\ref{fig:datamodel}. This model involves the whole database for
context-aware recommender system, as well as the entities and
relations among them. \\The context is modeled as a relational database,
each user context is a new register into data table to store user
contexts.\\Contextual information is also stored in the entities:
\textit{Reviews, CurrentLocation, DistancePoi and Ratings.} For
instance, \textit{Reviews entity} describes the user’s opinion about
visited restaurants, this information contributes to have additional
information about recent preferences of diners.\\   
\textit{CurrentLocation entity} stores the geographical position of
user to get a \textit{"nearby recommendation"}, the system locates
restaurants around 2 kilometers from the user position. The position
is changed frequently, in this manner, it allows the adaptation for
each particular situation. \textit{Distance Poi entity} stores the
distances (kilometers) between the user and restaurants, this
information is used to calculate \textit{"nearby recommendation"}, 
each recommended restaurant ought be over the threshold defined.\\   
Finally, \textit{Rating entity} represents the user preferences 
in a vector of scores, ratings could be increased in time and 
the user's preferences patterns could be changed in time also.
\begin{landscape} 
\begin{figure}[!h] 
\captionsetup{font=footnotesize}
\centering
\includegraphics[width=1.3\textwidth]{img/recomet.png}
\caption{The data model of context-aware recommender system.}
\label{fig:datamodel}    
\end{figure}
\end{landscape}

\section{Expert recomendation} 

Fuzzy logic is a methodology that provides a simple way to obtain
conclusions of linguistic data. Is based on the traditional process of
how a person makes decisions based in linguistic information. \\    Fuzzy
logic is a computational intelligence technique that allows to use
information with a high degree of inaccuracy; this is the difference
with the conventional logic that only uses concrete and accurately
information \cite{zedeh1989knowledge}.\\  In this work, fuzzy logic is
used to model fuzzy variables that highligh in the popularity of a
restaurant. The context-aware recommender system has implemented a
fuzzy inference system that represents the expert recommendation. \\   
The expert(fuzzy inference system) generates recommendations when the
recommendation techniques (collaborative filtering, content-based) are
not getting results because of the cold start problem.\\   The fuzzy
inference system proposed has 3 \textbf{input variables} (such as in
previous work realized\cite{garcia2009hybrid}): 1)\textit{rating} is
an average of ratings that has a particular restaurant inside the user
community; the domain of variable is 0 to 5 and contains 2 membership
functions labeled as \textit{low} and \textit{high}(figure
\ref{fig:mf:a}), 2)\textit{price} represents the kind of price that
has a particular restaurant; the domain of variable is 0 to 5 and
contains 2 membership functions labeled as \textit{low} and
\textit{high} (figure \ref{fig:mf:b}), and 3)\textit{votes} is used to
measure how many items have been rated by the current user, i.e., the
participation of the user, if the user has rated few items (less than
10) is not sufficient to make accurate predictions(figure
\ref{fig:mf:c}), the domain of variable is 0 to 10 and contains 2
membership functions labeled as \textit{insufficient} and
\textit{sufficient}. \\ The \textbf{output variable} is
\textit{recommendation}, represents a weight for each restaurant
proposed by the expert considering the inputs mentioned above, the
domain of variable is 0 to 5 and contains 3 membership functions
labeled as \textit{low}, \textit{medium} and \textit{high} (figure
\ref{fig:mf:d}).

\begin{figure}[ht!]
   \captionsetup{font=footnotesize}
   \centering
   %%----primera subfigura----
   \subfloat[]{
        \label{fig:mf:a}
        \includegraphics[width=0.42\textwidth]{img/mf-rating.png}}
   \hspace{0.1\linewidth}
   %%----segunda subfigura----
   \subfloat[]{
        \label{fig:mf:b} 
        \includegraphics[width=0.42\textwidth]{img/mf-price.png}}\\[20pt]
   %%----tercera subfigura----
   \subfloat[]{
        \label{fig:mf:c} 
        \includegraphics[width=0.42\textwidth]{img/mf-votes.png}}
    \hspace{0.1\linewidth}
   %%----cuarta subfigura----
    \subfloat[]{
        \label{fig:mf:d} 
        \includegraphics[width=0.42\textwidth]{img/mf-recommendation.png}}
   \caption{The Gaussian membership functions of the expert system.
   }
   \label{fig:mfexpert} 
\end{figure}

The proposed fuzzy inference system(figure \ref{fig:expertfis})
represents the users experience and their knowledge about restaurants.
This factors are considered important for users that visiting a
restaurant. This information is recovered of user profile and
restaurant profile; and the system uses this information to get
weights that influence in the final recommendations. The fuzzy
inference system uses 5 inference rules that involve the variables of
inputs and output. The input variables determine the recommendation
activation; each input variable contains labels as \textit{low} and
\textit{high} that also correspond to memberships functions of
Gaussian type. For the output variable \textit{recommendation} the
labels \textit{low}, \textit{medium}, and \textit{high} are used with
membership functions Gaussian type also. The rules are:
\begin{enumerate} 
\item \textit{If \textbf{rating} is high and \textbf{price} is low then 
\textbf{recommendation} is high.}
\item \textit{If \textbf{rating} is high and \textbf{votes} is sufficient then 
\textbf{recommendation} is high.}
\item \textit{If \textbf{rating} is high and \textbf{votes} is insufficient then 
\textbf{recommendation} is medium.}
\item \textit{If \textbf{rating} is low and \textbf{price} is high and then 
\textbf{recommendation} is low.} 
\item \textit{If \textbf{rating} is low and \textbf{votes} is insufficient then 
\textbf{recommendation} is low.}
\end{enumerate} 
\begin{figure*}
\captionsetup{justification=centering,margin=2cm,font=footnotesize}
\centering
\fbox{\includegraphics[width=0.70\textwidth]{img/expert.png}}
\caption{Fuzzy Inference System of expert.}
\label{fig:expertfis}      
\end{figure*}

\section{Fuzzy inference system to assing weights} 

The main goal of this fuzzy inference system is to define weights for
each recommendation list. The recommendation technique is based in the
amount of available information stored, so each technique utilizes
this information to provide a list of restaurants as well as a weight
for each one, therefore, these are used for  recommendations if its
weight is upper the threshold.  The fuzzy inference system has inputs
and outputs to infer each list's weight, its variables are depicted in
figure \ref{fig:fis-pesos}.  There are 3 membership functions for
inputs and 3 for outputs. The input variables are:
\textit{userSimilarity, restaurantSimilarity and Participation} and
are depicted in figure \ref{fig:fis-inputs}. The (\ref{fig:fis-inputs}.a) 
and(\ref{fig:fis-inputs}.b) are in a range from 0 to 1 to
define the similarity average among users and restaurants,
respectively. The figure (\ref{fig:fis-inputs}.c) has a range from 0
to 15  to represent the ratings of the user(participation). This
information is stored in the Popularity entity (see figure
\ref{fig:datamodel}). \\
\begin{figure*}
\captionsetup{justification=centering,margin=2cm,font=footnotesize}
\centering
\fbox{\includegraphics[width=0.70\textwidth]{img/fis-pesos.png}}
\caption{Fuzzy Inference System to assign weights.}
\label{fig:fis-pesos}       
\end{figure*}
By other side, the output variables are: \textit{Expert,
RestaurantProfile and Correlation}, these are depicted in figure
\ref{fig:fis-outputs}. The figure (\ref{fig:fis-outputs}.a) represents
the weight for expert recommendation list, figure (\ref{fig:fis-outputs}.b) 
represents the weight of the content-based list and figure
(\ref{fig:fis-outputs}.c) represents the weight of collaborative
recommendation list, their membership functions are in a range from 0
to 1 to get the value.
\begin{figure}[ht!]
   \captionsetup{font=footnotesize}
   \centering
   %----primera subfigura----
   \subfloat[]{
        \label{fig:mf:a}
        \fbox{\includegraphics[width=0.42\textwidth]{img/usersimilarity.png}}}
   \hspace{0.1\linewidth}
   %%----segunda subfigura----
   \subfloat[]{
        \label{fig:mf:b} 
        \fbox{\includegraphics[width=0.42\textwidth]{img/restaurantsimilarity.png}}}\\% 
   %%----tercera subfigura----
   \subfloat[]{
        \label{fig:mf:c} 
        \fbox{\includegraphics[width=0.42\textwidth]{img/participation.png}}}
        %\hspace{0.1\linewidth}
   \caption{The Gaussian membership functions of input variables.
   }
   \label{fig:fis-inputs} 
\end{figure}
\begin{figure}[ht!]
   \captionsetup{font=footnotesize}
   \centering
   %%----primera subfigura----
    \subfloat[]{
        \label{fig:mf:d} 
        \fbox{\includegraphics[width=0.42\textwidth]{img/expert-mf.png}}}%\\ %[20pt]
        \hspace{0.1\linewidth}
      %%----segunda subfigura----
    \subfloat[]{
        \label{fig:mf:d} 
        \fbox{\includegraphics[width=0.42\textwidth]{img/restaurantprofile-mf.png}}}\\
     %%----tercera subfigura----
    \subfloat[]{
        \label{fig:mf:d} 
        \fbox{\includegraphics[width=0.42\textwidth]{img/correlation-mf.png}}}%\\ %[20pt]
   \caption{The Gaussian membership functions of output variables.
   }
   \label{fig:fis-outputs} 
\end{figure}
Taking in account the input variables, the rules utilized to infer the 
output values are following:
\begin{enumerate} 
\item \textit{If \textbf{userSimilarity} is low and 
\textbf{restaurantSimilarity} is low and \textbf{participation} 
is insufficient then \textbf{expert} is high, \textbf{restaurantProfile} 
is low, \textbf{correlation} is low.}
\item \textit{If \textbf{userSimilarity} is low and 
\textbf{restaurantSimilarity} is low and \textbf{participation} 
is sufficient then \textbf{expert} is low, \textbf{restaurantProfile} 
is low, \textbf{correlation} is high.}
\item \textit{If \textbf{userSimilarity} is low and 
\textbf{restaurantSimilarity} is low and \textbf{participation} 
is minimun then \textbf{expert} is low, \textbf{restaurantProfile} 
is low, \textbf{correlation} is high.}
\item \textit{If \textbf{userSimilarity} is low and 
\textbf{restaurantSimilarity} is high and \textbf{participation} 
is insufficient then \textbf{expert} is low, \textbf{restaurantProfile} 
is high, \textbf{correlation} is low.}
\item \textit{If \textbf{userSimilarity} is low and 
\textbf{restaurantSimilarity} is high and \textbf{participation} 
is minimun then \textbf{expert} is low, \textbf{restaurantProfile} i
s high, \textbf{correlation} is low.}
\item \textit{If \textbf{userSimilarity} is low and 
\textbf{restaurantSimilarity} is high and \textbf{participation} 
is sufficient then \textbf{expert} is low, \textbf{restaurantProfile} 
is high, \textbf{correlation} is low.}
\item \textit{If \textbf{userSimilarity} is high and 
\textbf{restaurantSimilarity} is low and \textbf{participation} 
is insufficient then \textbf{expert} is low, \textbf{restaurantProfile} 
is low, \textbf{correlation} is high.}
\item \textit{If \textbf{userSimilarity} is high and 
\textbf{restaurantSimilarity} is low and \textbf{participation} 
is minimun then \textbf{expert} is low, \textbf{restaurantProfile} 
is low, \textbf{correlation} is high.}
\item \textit{If \textbf{userSimilarity} is high and 
\textbf{restaurantSimilarity} is low and \textbf{participation} 
is sufficient then \textbf{expert} is low, \textbf{restaurantProfile} 
is low, \textbf{correlation} is high.}
\item \textit{If \textbf{userSimilarity} is high and 
\textbf{restaurantSimilarity} is high and \textbf{participation} 
is insufficient then \textbf{expert} is low, \textbf{restaurantProfile} 
is low, \textbf{correlation} is high.}
\item \textit{If \textbf{userSimilarity} is high and 
\textbf{restaurantSimilarity} is high and \textbf{participation} 
is sufficient then \textbf{expert} is low, \textbf{restaurantProfile} 
is low, \textbf{correlation} is high.}
\item \textit{If \textbf{userSimilarity} is high and 
\textbf{restaurantSimilarity} is high and \textbf{participation} 
is minimun then \textbf{expert} is low, \textbf{restaurantProfile} 
is low, \textbf{correlation} is high.}
\end{enumerate} 
At the end, a \textit{weighted average} allows to get predictions for
each restaurant in the list of recommendations. In this way, for
instance if the \textbf{expert} has a weight of \textbf{0.5}, the
\textbf{restaurantProfile} is \textbf{0.8} and the
\textbf{correlation} is \textbf{0.6}, the system uses these weights to
calculate the final prediction for a particular restaurant using the
formula of the weighted average:
\begin{equation}\label{eq:prediction}
\displaystyle prediction_{i} = {(0.5 * 4.0) + ( 0.8 * 5) + (0.6 * 4.5)
\over (0.5 + 0.8 + 0.6)}
\end{equation}
The prediction corresponds the final value of recommendation for the
item, and is used to include it or exclude it of the list of
recommendation if is not over the threshold. So, for this case, the
prediction is \textbf{4.57}, it means that this restaurant will be in the
recommendation list of the user, subsequently, the recommendations
list will be contextualized.

\section{Contextual recommendation} 

The interface of the system(figure \ref{fig:context}) allows to
collect contextual information such as type of price, restaurant's
attributes, type of cuisine and geographical location. The context-aware recommender system uses post-filtering paradigm, then the
contextual information is used for adjust the final recommendations
list. For example, geographical location is used to get restaurants
around 2 kilometers of distance, next, the list of nearby restaurants
is displayed for the user. If context-aware recommender system
considers another attributes as type of price and type of cuisine
preferred by the user, the system gets restaurants matched in the
context especified by the user in this time. In the attributes box,
the user can chose any preference about what things are importants to
select a restaurant. The features are collected from the dataset of
Tijuana restaurants. In the cuisine box, the user choosen his/her
favorite cuisine, it can be one or more cuisines such as in attributes 
also.\\  
The context changes constantly, indeed, the users migh change 
it many times such as them wish.
\begin{figure*}
\captionsetup{justification=centering,margin=2cm,font=footnotesize}
\centering
\fbox{\includegraphics[width=0.60\textwidth]{img/context.png}}
\caption{System interface to collect contextual information.}
\label{fig:context}   
\end{figure*}
After the post-filtering, the system displays the  recommended
restaurants according the information provides by the user. The
context-aware recommender system contains four techniques to display
recommendations. The interface in figure \ref{fig:recom} shows
recommendations: \textit{1) Expert, 2) Content-based, 3) Collaborative
filtering and 4) Nearby.} Each one was explained above, except the
nearby recommendations. For nearby recommendations the system
calculates the approximate distance between the current geographical
location of the user and the available restaurants in the area.  The
threshold is 2 kilometers around the user position to determine what
restaurants will be recommended. The geographical position is
obtained throught Google maps services.
\begin{figure*}
\captionsetup{font=footnotesize}
\centering
\fbox{\includegraphics[width=0.60\textwidth]{img/recom.png}}
\caption{System inferface of recommendations for the user.}
\label{fig:recom}    
\end{figure*}

\section{Methodology} 

The scheme of proposed method is depicted in the figure
\ref{fig:archit}. In the first part, the three techniques of
recommendations are suplied by the rating matrix. 
\begin{figure*}
\captionsetup{font=footnotesize}
\centering 
%\fbox{\includegraphics[width=10cm,height=10cm,keepaspectratio]{img/archit.png}}
\includegraphics[width=10cm,height=10cm,keepaspectratio]{img/archit.png}
\caption{Scheme of the proposed method.}
\label{fig:archit}  
\end{figure*}
Ratings matrix makes that \textbf{fuzzy inference system} can obtain
the inputs values to calculate the output value. \textbf{Content-based} 
utilizes the rating matrix and user profiles to compare the
similarity among the restaurants through cosine similarity measure.
\textbf{Collaborative filtering} is based in user profiles content in
ratings matrix, using Pearson correlation calculates the similarity
among the users, subsequently, a list of neighbors is obtained to use
their preferences to calculate predictions.\\
The second part shows the recommendation lists for the user. Later,
the recommendation lists are reduced when filter of context is applied,
i.e., the recommendations are adjusted for the user current context.
Finally, the contextual recommendations list is displayed in the user
interface (figure \ref{fig:recom}).







\chapter{Experiments and Results} \label{sec:4}


In this chapter are explained the experiments performed in order to
reach the goal of this thesis.  Every experiment utilizes different
datasets and context domains.

\section{Tijuana Restaurants dataset} 

The context-aware recommender system  
uses collaborative filtering to find restaurants for the
user\cite{ramirez2013restaurant}. The ratings of user profiles are
used to determine the similarity among users using Pearson correlation. \\   
The similarity between the user and  neighbors is used to calculate a
weighted average of ratings for each particular item;  the top-N
ratings are used as a list of recommended restaurants for the active
user. The output of the collaborative filtering algorithm (top-N list)
is supplied to the next step of the post-filtering process. \\   The
restaurants are adjusted to the context in the next step in order to
make ranking of restaurants in the current context. Post- filtering is
based on the average of ratings in a specific context, so prediction
is made with: 1) the average post-filtering a restaurant has in the
current context (that is the mean of user ratings) and 2) the rating
predicted by the collaborative filtering algorithm. The top-N list
contains the restaurants with highest predictions, so each restaurant
is adjusted for the user’s context and listed in contextual
recommendations; the process is depicted in figure
\ref{fig:postfiltering}.
\begin{figure*}
%\captionsetup{justification=centering,margin=2cm}
\centering
\setlength\fboxsep{0pt}
\fbox{\includegraphics[width=0.60\textwidth]{img/posfil.png}}
\caption{The Post-Filtering architecture for Tijuana restaurants.}
\label{fig:postfiltering}     
\end{figure*}

In order to validate the proposed approach, data about restaurant
preferences of users in different contexts was collected. The study
subjects were students enrolled in a computer engineer major, a
master’s program and professors of the Tijuana Institute of
Technology. A total of \textbf{50 users} answered a questionnaire; the
questions were about their preferences for nearby restaurants and the
technology used by them. The \textit{questionnaire} consisted of \textbf{8
questions} and they rate restaurants from a list of 40 restaurants.
Each restaurant chosen was rated 6 times one per context considered, a
matrix rating with \textit{1,422 ratings} were collected. The
questions are shown in the table \ref{tab:questions}.
\begin{table}
\small
\caption{Questionnaire applied to collect contextual dataset.}
\label{tab:questions} 
\centering
\small
\begin{tabular}{p{7cm} p{5cm} }
\hline\noalign{\smallskip}
Question & Answers \\
\noalign{\smallskip}\hline\noalign{\smallskip}
\small{1.What is your occupation?} & \small{1. Student 2.Employee} \\ \hline  
\small{2.According your priority, order by importance the features 
you consider when you choose to visit a restaurant.} & 
\small{1.Installation/decoration 2.Prices 3.Service 4.Dishes
5.Atmosphere 6.Location} \\ \hline  
\small{3.What are the devices that you used
utilizes?} & \small{1.Smartphone 2.Tablet 3.Laptop 4.PC} \\ \hline   
\small{4.What Operating System do you used?} & 
\small{1.Android 2.Windows 3.iOS 4.Symbian 5.Blackberry 6.Other}
\\ \hline  
\small{5.Did you use an application to search restaurants in Tijuana?} &
\small{1.Yes 2.No 3.Which one?} \\ \hline   
\small{6.Would you like to use an application of
recommender systems of Tijuana?} & \small{1.Yes 2.No} \\ \hline  
\small{7.Please, rates your favorites restaurants(without context).} & 
\small{Restaurant list} \\ \hline
\small{8.Please, rates your favorites restaurants in contextual situations.} & 
\small{Restaurant list} \\
\noalign{\smallskip}\hline
\end{tabular}
\end{table}

The user's answers from question 1 to question 6 are represented in
the figure \ref{fig:cakeschart}. \textbf{Figure \ref{fig:cakeschart}a}
represents the percentage of surveyed students and teachers;
\textbf{figure \ref{fig:cakeschart}b}  the percentage of the element
that users consider the most important to visit a restaurant;
\textbf{figure \ref{fig:cakeschart}c} represents the preferences of
devices when are using Internet for restaurant recommendations;
\textbf{figure \ref{fig:cakeschart}d} represents the percentage of
operating system that often used, \textbf{figure
\ref{fig:cakeschart}e} shows the percentage of users that use the
Internet to search restaurants in Tijuana; and \textbf{figure
\ref{fig:cakeschart}f}, shows the percentage of users that would like
using a restaurant recommender system of Tijuana.
\begin{figure*}
\captionsetup{justification=centering,margin=2cm}
\centering
\setlength\fboxsep{0pt}
%\setlength\fboxrule{0.7pt}
\includegraphics[width=0.9\textwidth]{img/cakes.png}
\caption{The chart shows the users preferences for questions from 1 to 6.}
\label{fig:cakeschart}     
\end{figure*}
For questions 7 and 8 only the top-ten restaurants are shown,
without/with the contextual situation. In figure \ref{fig:barschart}a,
the favorite restaurant is \textbf{Daruma}(178 votes),  whereas in
figure \ref{fig:barschart}b, \textbf{Daruma} does not appear in the
top-ten. When considering the context \textit{midweek}, the favorite
restaurant was \textbf{Carls Jr.}, which appears in both graphs; this
restaurant was also the most voted in the different contexts.
\begin{figure*}
\captionsetup{justification=centering,margin=2cm}
\centering
\setlength\fboxsep{0pt}
%\setlength\fboxrule{0.7pt}
\includegraphics[width=0.5\textwidth]{img/bars.png}
\caption{The chart shows the users preferences for questions 7 and 8.}
\label{fig:barschart}     
\end{figure*}
Contextual recommendations of post-filtering approach depends of
context \textit{midweek} or \textit{weekend}, which is the day when
the restaurants were rated. Subsequently, the result of the query is
refined according to the user context; the 6 contexts mentioned
correspond to combinations of contextual factors shown in table
\ref{tab:contextstijuana}.
\begin{table}
\small
\caption{Contextual factors considered in the questionnaire.}
\label{tab:contextstijuana} 
\centering
\small
\begin{tabular}{p{2.5cm} p{7cm} }
\hline\noalign{\smallskip}
Contextual Factor & Context \\
\noalign{\smallskip}\hline\noalign{\smallskip}
\small{Day} & \small{1.Midweek(Monday, Thuesday,Wednesday and Thursday) 
2.Weekend(Friday,Saturday and Sunday)}  \\ \hline 
\small{Place} & \small{1.School 2. Home 3.Work} \\ 
\noalign{\smallskip}\hline
\end{tabular}
\end{table}
The dataset was explicitly collected from \textbf{50 users} whom
answered questionnaire (see table \ref{tab:questions}). A total of 172
predictions was made for different users and the error
\textbf{MAE=0.5859} when the context \textbf{midweek} for current user
was considered. The observation for this result is that using a small
dataset the performance of the method proposed is limited. By other
hand, having only one contextual factor does not improve the accuracy
of the recommendations in this domain.

\section{MovieLens dataset} 

GroupLens Research has collected and made available rating data sets
from the MovieLens  web site (http://movielens.org). \\   The data sets
were collected over various periods of time,  depending on the size of
the set.
\begin{itemize} 
\item \textbf{MovieLens 1M Dataset}: Stable
benchmark dataset, 1 million ratings from 6000 users on 4000 movies.
Released 2/2003.\\ Downloaded from
\textit{http://grouplens.org/datasets/movielens/1m/}. 
\item \textbf{MovieLens
10M Dataset}: Stable benchmark dataset, 10 million ratings and 100,000
tag applications applied to 10,000 movies by 72,000 users. Released
1/2009. \\Downloaded from
\textit{http://grouplens.org/datasets/movielens/10m/}. 
\end{itemize}
The recommender system proposed for MovieLens uses post-filtering 
and time segmentation.
Time in recommender systems is used  as a contextual factor in the
research reviewed \cite{baltrunas2009context},
\cite{baltrunas2009towards}, \cite{koren2010collaborative}, and
\cite{he2009time}, results vary according the techniques that were
done.\\  In \cite{he2009time} the pre-filtering approach was used, time
was divided in time intervals and the size of time intervals is
directly proportional to the distance from initiating the historical
information to the current user context. In
\cite{koren2010collaborative} a tracking model of user behavior over
the life-time of data is proposed, in order to exploit the relevant
components of all  data instances , while discarding only what is
modeled as being irrelevant.\\  In \cite{baltrunas2009context} it is
shown that the time division is beneficial and its performance depends
on the items selection method and influence of contextual variables in
item ratings. In \cite{baltrunas2009towards} the user profile is
segmented into micro-profiles corresponding to a particular context,
each context represents a time span in which recommendations for users
are derived.\\  This experiment implements fuzzy logic on time
segmentation, in order to improve user satisfaction by providing
recommendations based to context and recents user preferences without
discarding tastes in the past, as they include important information
for the recommender system proposed. The first phase is division of
three time segments based on the current context of the user is
performed such as in is depicted in figure \ref{fig:context-ml}.
\begin{figure*}
\captionsetup{justification=centering,margin=2cm}
\centering
\setlength\fboxsep{0pt}
%\setlength\fboxrule{0.7pt}
\fbox{\includegraphics[width=0.60\textwidth]{img/context-ml.png}}
\caption{Time segmentation of contexts based on current user
context.}
\label{fig:context-ml}     
\end{figure*}

In recommender system, the first step is get the current user context
(user-application interaction), from this information three contexts
(figure \ref{fig:context-ml})  will be obtained that representing a
time segment of three months each one, in total the algorithm
considers all the ratings users did during nine months prior the
current context. Subsequently, ratings are classified by contexts and
reused as contextual rating matrix being, a ratings matrix for each
context. \\ The size of matrix depends of users’ participation during
the last nine months. One of the aims is to identify the user behavior
through recent information, in order to, for instance, know whether
the user changes ratings constantly; whether usually assign high, low
or mixed ratings; whether user likes to see different items or whether
have a favorite category.\\  Recommender systems use the collaborative
filtering algorithm in order to find relevant items for the user
\cite{ramirez2013restaurant}. User's profiles are used for determine
the similarity between users calculated with Pearson correlation. The
similarity between users can provide valuable information as long as
user participation is enough (less than 10 ratings). The next step is
to obtain recommendations list (Top-N), three contextual lists are the
outputs of collaborative filtering algorithm and contain items with
user's predictions for each context.\\  Popularity's prediction
considers other variables: 1) user’s participation in respect of an
item in the context and, 2) the rating's average that users have given
for item in the same context. \\   A Fuzzy Inference System (FIS) uses
these parameters to assign a weight within a scale from 1 to 5
(prediction value). These recommendations are used when the ratings
matrix is sparse, a popularity prediction is done. \\ Finally, the
system gets the recommendations list for each user in different
contexts. The recommendation process of pre-filtering is depicted in
figure \ref{fig:archi-ml}.
\begin{figure*}
\captionsetup{justification=centering,margin=2cm}
\centering
\setlength\fboxsep{0pt}
%\setlength\fboxrule{0.7pt}
\fbox{\includegraphics[width=0.60\textwidth]{img/archi-ml.png}}
\caption{Pre-filtering process for context-aware recommender
system.}
\label{fig:archi-ml}     
\end{figure*}
The dataset used to test the algorithm was MovieLens(100000 ratings)
with 943 users and 1,682 movies. The ratings were collected in a
period  of 2 years. MovieLens is not a contextual dataset, however,
the  timestamp was used to determine the rating time, i.e., in this
way it  was noted the day to know whether the rating time was in
weekday or  weekend. In this terms, the context was used. Then, the
time for each  context was divided in 3 months each one, this span
covers 9 months  before the user's current context. \\ The neighbors
in each context are  considered to recommend movies in that context
only.  An average of  predictions are considered for add a movie to
the top-N list of contextualized recommendations.The result in
table \ref{tab:contexts} shows the error in three contexts. The
error increase in context 3, in this  context the ratings matrix is a
little bit sparse;  the error is  justifiable because user has less
participations.
\begin{table}
\small
\caption{Results of comparison by contexts in MovieLens dataset.}
\label{tab:contexts} 
\small
\centering
\begin{tabular}{llll}
\hline\noalign{\smallskip}
Context & \# Preditions & MAE &    \\
\noalign{\smallskip}\hline\noalign{\smallskip}
1 & 12235 & 0.28 \\
2 & 21049 & 0.24 \\
3 & 1075  & 0.38 \\
\noalign{\smallskip}\hline
\end{tabular}
\end{table}


%%-------------------------------------------------------------------
%%--EXPERIMENTO DE TRIPADVISOR---------------------------------------

\section{Tripadvisor dataset} 

The dataset used to evaluate the algorithm was TripAdvisor in two
versions downloaded \cite{linkzeng}, this datasets was used in
\cite{zheng2014context}, \cite{zheng2012differential} to  evaluate the
performance of context-aware recommender systems. \\  The first
dataset contains 4669 contextual ratings, 1202 users and 1890 hotels;
the second dataset contains 14175 contextual ratings, 2731 users and
2269 hotels. Data were collected of reviews online in tripadvisor.com.
There is only one context: type of trip (family, friends, bussines,
romantic and relax).\\ 
The proposed method consists of three algorithms to recommend: Fuzzy
Inference System, collaborative filtering and content-based. Each one
uses rating matrix to get recommendations.\\    The context-aware
recommender system uses the post-filtering
paradigm\cite{adomavicius2011context} for adjust recommendations in
context. The recommendation by popularity is through the Fuzzy
Inference System depicted in figure \ref{fig:fis}, the Fuzzy Inference
System contains the variables that are involved in the process to
recommend in a human interaction, this process is the same that the
recommender system does. \\   The output represents how matter each item
into the users community, i.e. if it was a popular item for users. \\  The
FIS has Gaussians membership functions and are depicted in figure
\ref{fig:mffis}.
\begin{figure}[ht!]
   \centering
   %%----primera subfigura----
   \subfloat[]{
        \label{fig:1a}
        \includegraphics[width=0.42\textwidth]{img/ratingaverage.png}}
   \hspace{0.1\linewidth}
   %%----segunda subfigura----
   \subfloat[]{
        \label{fig:1b} 
        \includegraphics[width=0.42\textwidth]{img/userparticipation.png}}\\[20pt]
   %%----tercera subfigura----
    \subfloat[]{
        \label{fig:1c} 
        \includegraphics[width=0.42\textwidth]{img/recommendation.png}}
   \caption{Gaussian Membership functions in the input are: a) RatingAverage, 
   b) UserParticipation, and an output: c) Recommendation.}
   \label{fig:mffis} 
\end{figure}
The Fuzzy Inference System uses fuzzy rules to infer the inputs and 
output (a numeric value) that represents the weight of the recommendation. 
The rules are following: 
\begin{enumerate}
\item \textit{If \textbf{RatingAverage} is low and \textbf{UserParticipation} is insufficient then \textbf{recommendation} is low.}
\item \textit{If \textbf{RatingAverage} is low and \textbf{UserParticipation} is sufficient then \textbf{recommendation} is high.}
\item \textit{If \textbf{RatingAverage} is high and \textbf{UserParticipation} is insufficient then \textbf{recommendation} is low.}
\item \textit{If \textbf{RatingAverage} is high and \textbf{UserParticipation} is sufficient then \textbf{recommendation} is high.}
\end{enumerate}
\begin{figure*}
\captionsetup{justification=centering,margin=2cm}
\centering
\setlength\fboxsep{0pt}
\setlength\fboxrule{0.7pt}
\includegraphics[width=0.75\textwidth]{img/fis.png}
\caption{Fuzzy Inference System.}
\label{fig:fis}   
\end{figure*}
Content-based uses cosine similarity to compare the binary
vectors representing the profile of each item, thereby obtaining a
numerical value that determines similarity, based on a threshold. \\   In
other words, it makes a comparison of profiles of each item to
determine the most similar to items the user has rated with highest
score, context-aware recommender system proposed has a scale from 1 to
5. 
\begin{table}[htb]
\small
\centering
\caption{Example of contextual ratings in the user profile.}
\label{tab:2}
\small
\begin{tabular}{lll}
\hline
\multicolumn{3}{c}{User profile} \\ \hline
Item1 & Rating1 & Context1 \\ 
Item2 & Rating2 & Context2 \\ 
Item3 & Rating3 & Context3 \\ \hline
\end{tabular}
\end{table}
In the next step the outputs of every recommender algorithm is
represented by a list of recommended items. Subsequently applies the
context filter and context-aware recommender system gets the final
contextual recommendations. \\   Context-aware
recommender system identifies contextual data of the user profile (see
table \ref{tab:2}), and compares recommended items to filter those
items that are adjusted to the user context. 

The context filtering is the last step before to get the recommended
items. The schema of architecture for context-aware recommender system
is depicted in figure \ref{fig:architecture}.
\begin{figure*}
\captionsetup{justification=centering,margin=2cm}
\centering
\fbox{\includegraphics[width=0.70\textwidth]{img/archit-ta.png}}
\caption{Recommender system architecture}
\label{fig:architecture}   
\end{figure*}
Two experiments were performed using TripAdvisor dataset, table
\ref{tab:3} describes the data sets and the scarcity percentage of the
specified data. Scarcity of 99\% mean that there are problems to
recommend items because the information is not enought to get 
good recommendations.\\  By other side, in table \ref{tab:4} the comparison
shows that the algorithm has a acceptable performance, i.e., the error
falls into the range of results obtained with others algorithms. Then,
contextual recommendations were evaluated with the Root Mean Square
Error in order to compare the results with context relaxation
algorithm\cite{zheng2012differential} that is evaluated with the same
dataset.
\begin{table}
\centering
\small
\caption{Datasets description.}
\label{tab:3}      
\begin{tabular}{lllll}
\hline\noalign{\smallskip}
Dataset & Users & Items & Ratings & Scarcity (percent) \\
\noalign{\smallskip}\hline\noalign{\smallskip}
TripAdvisor v1 & 1202 & 1890 & 4669 & 99.79 \\
TripAdvisor v2 & 2731 & 2269 & 14175 & 99.77 \\
\noalign{\smallskip}\hline
\end{tabular}
\end{table}

\begin{table}
\centering
\small
\caption{Comparison of RMSE.}
\label{tab:4}  
\small   
\begin{tabular}{lll}
\hline\noalign{\smallskip}
Dataset & Algorithm & RMSE \\
\noalign{\smallskip}\hline\noalign{\smallskip}
TripAdvisor v2 & FC + Post-filtering  & 0.504  \\
               & FC          & 0.994  \\
               & Pre-filtering + Relaxation & 0.985  \\
\noalign{\smallskip}\hline
\end{tabular}
\end{table}

The fundament of content-based is the cosine similarity; this means
that if similarity value among items is high, the recommendations will
improve the degree of user satisfaction. This is observed when
calculating the similarity average in each dataset as shown in table
\ref{tab:5}.
\begin{table}
\centering
\small
\caption{Level of similarity among items in datasets. }
\label{tab:5}      
\begin{tabular}{lll}
\hline\noalign{\smallskip}
Dataset  & Similarity  & Avg.votes per user. \\
\noalign{\smallskip}\hline\noalign{\smallskip}
TripAdvisor v1 & 0.448  & 5  \\
TripAdvisor v2 & 0.508  & 8  \\
\noalign{\smallskip}\hline
\end{tabular}
\end{table}

FIS can provides a list of popular items for each dataset,
recommendations through averages are obtained, and recommendations are
conditioned to show it when the collaborative filtering and content-
based are not delivering recommendations because of data scarcity.
However, the majority of popular items of dataset were rated in contexts: romantic, family and bussines, that means that the dataset has
biases.\\  In this experiment  the context-aware recommender system
proposed involves the paradigm of post-filtering for contextual
recommendations. The structure of the datasets facilitated the
evaluation of recommendations although the rating matrix has been
scarce in both cases. Anyway, information of items and users was used
to test the system and a good performance of the system was done.\\   With
respect the performance, post-filtering allows select relevant
items that are adjusted into the context, indeed, post-filtering and
implementation of different recommendation techniques the system has
suitable performance and the datasets help the processes performed.

\section{Datasets in matrix factorization}

\subsubsection{Filmtrust dataset} 

FilmTrust is a small dataset crawled from the entire FilmTrust website
in June, 2011. Filmtrust contains a ratings matrix of 35498 ratings,
1504  users and 2071 movies. The dataset has a density of 1.14\% and
was used in \cite{guo2013novel} using the trust level such as
context. The web page is \textit{http://www.librec.net/datasets.html}.

\subsubsection{InCarMusic dataset}

InCarMusic dataset\cite{baltrunas2011incarmusic} has 8 
contextual factors and the possible values for contextual conditions 
are explained in table \ref{tab:incarmusic}.
Music tracks were ten different genres. There is not unified music
genre taxonomy, for this reason the recommender system uses the genres
defined in \cite{tzanetakis2002musical}: classical, country, disco, 
hip hop,  jazz, rock, blues, reggae, pop and metal, 50 music tracks 
and 42 users in dataset.
\begin{table}
\centering
\small
\caption{Contexts in InCarMusic dataset.}
\label{tab:incarmusic}   
\begin{tabular}{ll}
\hline\noalign{\smallskip}
Context  			& Values \\
\noalign{\smallskip}\hline\noalign{\smallskip}
Driving style 		&  elaxed, driving, sport driving.   \\
Road type 			&  city, highway, serpentine. \\
Landscape 			& coast line, country side, mountains/hills, urban.\\
Sleepiness 			& awake, sleepy. \\
Traffic conditions 	& free road, many cars, traffic jam. \\
Mood 				& active, happy, lazy, sad. \\
Weather 			& cloudy, snowing, sunny, rainy. \\
Natural phenomena 	& day time, morning, night, afternoon. \\
\noalign{\smallskip}\hline
\end{tabular}
\end{table}

\subsection{Results} 

For experiments with matrix factorization technique the Graphlab
toolbox was used. Both mentioned datasets and Movielens (1 millon and
10 millions) were used to test the algoritmh. The test involves K
factors that are increasing for 50 iterations. previously, was done a
test to identify what number of iterations are enough to get a good
result with no overload of process in the algorithm. Results are
depicted in the chart \ref{fig:fm-test}  where the axis (x, y)
represent the K value and the error value, respectively. The
observations deal to small differences among the datasets, in a range
of 0.80-0.90, and the high variability is in MovieLens dataset 10
millions. The big dataset implies more unstable behaviour, while in a
small dataset (Filmtrust) the error is less variable. A comparison
among MovieLens 1 million and 10 millions shows that there's not a
significant difference.

\begin{figure*}
\centering
\fbox{\includegraphics[width=0.70\textwidth]{img/fm-test.png}}
\caption{RMSE results of matrix factorization test.}
\label{fig:fm-test}   
\end{figure*}
By other side,  other datasets were used to test matrix factorization under the same parameters to calculate the RMSE for each one. Table \ref{tab:mf-set}
presents the total of ratings of each dataset, the cosine similarity,
it means how similar are the items into the dataset, and the RMSE
error obtained in the test with matrix factorization technique. 
\begin{table}
\centering
\small
\caption{RMSE of datasets using matrix factorization.}
\label{tab:mf-set}   
\begin{tabular}{llll}
\hline\noalign{\smallskip}
Dataset & Ratings & Cosine Sim. & RMSE \\
\noalign{\smallskip}\hline\noalign{\smallskip}
Tijuana Rest.  &    896       &  0.67    &   0.60 \\
Mexico Rest.  &   1161       &  0.25    &  0.54 \\
InCarMusic    &    4012      &  0.45     &  0.93 \\
TripAdvisor    &    4669      &  0.17     &  0.85 \\
MovieLens    &    10000     &   0.46    &  0.51 \\
Movielens    &     100000   &   0.94    &  0.42 \\
\noalign{\smallskip}\hline
\end{tabular}
\end{table}
The datasets contain less ratings than the presented in the chart
\ref{tab:mf-set}, according the table \ref{tab:mf-set} is not possible
to assum that matrix factorization has a better performance with small
datasets, because TripAdvisor and InCarMusic datasets obtain an error
in the same range that the large datasets of the previuos chart.



\chapter{System evaluation} \label{sec:6}

\section{Metrics}

To evaluate context-aware recommender system was used the \textbf{task success}
and \textbf{time-on-task} metrics. \\ The \textbf{task success metric} is
perhaps the most widely used performance metric. It measures how effectively
users are able to complete a given set of tasks.  The \textbf{time-on-task
metric} is a common performance metric that measures how much time is required
to complete a task\cite{albert2013measuring}.\\ Task success is something that
almost anyone can do.  If the users can’t complete their tasks, then something
is wrong.  When the users fail to complete a simple task can be an evidence that
something needs to be fixed in the recommender system.  The usability test
consist of a list of simple tasks for users that they shall perform in the
system to complete the test. Before to start, a minimal description about the
system for user was explained. The tasks list are the following:
\begin{enumerate} 
\item \textit{Rated a restaurant without context.}
\item \textit{Add context to the user profile.}
\item \textit{Filter restaurants by favorite context.}
\item \textit{Find information of a specific restaurant.}
\item \textit{Find all the reviews of a specific restaurant.} 
\item \textit{Find section of my favorite restaurants.}
\item \textit{Add a review of a restaurant.}
\item \textit{Find the most popular restaurants.}
\item \textit{Add a restaurant to your wishlist.}
\item \textit{Get recommendations based on expert opinion.} 
\item \textit{Get the recommendations content-based.}
\item \textit{Get the collaborative recommendations.}
\item \textit{Get recommendations of the nearby restaurants.}
\end{enumerate} 

\section{Enviromental set up}

Each user did the task list, one by one, with previous instructions. It gives a brief explanation about the general features of system before to start. The time average for each user was around 10 minutes to finished all activities without disruptions.\\ 
After, the results was depicted in a chart to observe the user behaviour for each task, in the figure \ref{fig:tsuccess}  the axis (x, y) represent the task number and percent of success, respectively. The chart shows that only 3 tasks weren’t accomplished successfuly, the task 5, 6 and 7. \\ The issue with task 5 was that users can not found easily the reviews section in the interface, the issue in task 7 is derived of task 5 because the user couldn’t find the manner to add a review. The task 6 correspond to the favorite restaurants, but the issue is that it was confused to chose favorite restaurants in place of wishlist section. \\ In general, these results mean a possible redisign in the interfacte to facilitate the performance of these tasks.
\begin{figure*}
%\captionsetup{justification=centering,margin=1cm}
\centering
\captionsetup{font=footnotesize}
\fbox{\includegraphics[scale=0.75]{img/tsuccess.png}} %[width=0.7\textwidth]
\caption{Representation of the percent of success for each task.}
\label{fig:tsuccess}   
\end{figure*}
The time it takes a participant to perform a task says a lot about the usability
of the application. In almost every situation, the faster a participant can
complete a task, the better the experience. In fact, it would be pretty unusual
for a user to complain that a task took less time than expected
\cite{albert2013measuring}.\\ Then, task-on-time was applied to measure time
that an user did the task. A resume of the time tasks for each user it is in
table \ref{tab:datausers}, \textit{null} values mean that the user didn't the task.
\begin{table}
\centering
\small
\captionsetup{font=footnotesize}
\caption{Time on task data for 10 users and 13 tasks. }
\label{tab:datausers}  
\begin{tabular}{lllllllllll}
\hline\noalign{\smallskip}
Task  & Us1  & Us2 & Us3 & Us4 & Us5 & Us6 & Us7 & Us8 & Us9 & Us10 \\
\noalign{\smallskip}\hline\noalign{\smallskip}
1 & 12  & 28 & 24 & 30 & 19 & 33  & 23 & 16 & 5  & 7 \\
2 & 3   & 4  & 17 & 5  & 17 & 134 & 9  & 16 & 12 & 11 \\
3 & 123 & 69 & 159& 53 & 69 & 113 & 44 & 41 & 70 & 98 \\
4 & 20  & 4  & 86 & 40 & 13 & 4   & 17 & 3  & 20 & 3 \\
5 & 50  & 10 & 63 & 50 & 7  & 11  & 10 & 5  & 20 & Null \\
6 & 10  & 30 & 28 & 27 & 5  & 46  & Null  & 7  & Null  & 34 \\
7 & 10  & 20 & 16 & 8  & 15 & Null & 9  & 24 & 16 & 28 \\
8 & 18  & 24 & 10 & 10 & 5  & 3   & 27 & 4  & 5  & 6 \\
9 & 5   & 6  & 31 & 4  & 45 & 9   & 12 & 5  & 3  & 8 \\
10 & 15 & 17 & 15 & 11 & 10 & 19  & 13 & 10 & 20 & 20 \\
11 & 30 & 15 & 20 & 16 & 20 & 22  & 15 & 13 & 18 & 20 \\
12 & 12 & 14 & 19 & 14 & 40 & 10  & 17 & 17 & 15 & 15 \\
13 & 25 & 15 & 15 & 14 & 10 & 10  & 11 & 10 & 10 & 25 \\
\noalign{\smallskip}\hline
\end{tabular}
\end{table}

\section{Results}

To measure the efficiency of the metric it was chose an confidence interval.  In
this way, it is observed the time variability within the same task and  also
helps visualize the difference across tasks to determine whether there is a
statistically significant difference between tasks. The obtained information is
in table \ref{tab:ic}, the median was used to calculate the confidence interval.
\begin{table}
\centering
\small
\captionsetup{font=footnotesize}
\caption{Confidence interval per task with a confidence level of 95\%. }
\label{tab:ic}    
\begin{tabular}{lllll}
\hline\noalign{\smallskip}
Task  & Median & CI 95\% & Upper bound & Lower bound  \\
\noalign{\smallskip}\hline\noalign{\smallskip}
1 &    20         & 5.96  & 25.96 & 14.04  \\
2 &    11.5      & 0.81  & 12.31  & 10.69   \\
3 &    69.5      &  25.57   &  95.07  &  43.93   \\
4 &    15        & 16.34  &  31.34  &  -1.34   \\
5 &    15.5     &  14.84  &  30.34  &  0.66  \\
6 &     27.5    &   11.57  &  39.07  &  15.93    \\
7 &     16       &  5.19  & 21.19  &  10.81   \\
8 &     8         &   5.80  &  13.80  & 2.20 \\
9 &     7         & 9.43  &  16.43  &  -2.43  \\
10 &   15       &   2.44  &  17.44   &  12.56   \\
11 &   19       &  3.00  &  22.00  &  16.00   \\
12 &   14.5    &  5.51  &  20.01  &  8.99   \\
13 &   12.5    &  3.89  &  16.39  &  8.61    \\
\noalign{\smallskip}\hline
\end{tabular}
\end{table}
In the next step the USE \textit{(Usefulnes, Satisfaction, and Ease of Use)}
questionnaire \cite{morris2001experience} was applied in order to get the user's
feedback and comments for to know about the difficults in the test.  The USE
questionnaire consists of 30 rating scales divided into 4 categories:
\textit{Usefulness, Satisfaction, Ease of Use, and Ease of Learning}. Each is a positive
statement to which the user rates level of agreement on a 7-point Likert scale.
The USE questionnaire(see appendix \ref{apendixb}) allows to get values for Usefulness, Satisfaction, Ease of Use, and Ease of Learning, the visualizing the results is in the
Fig.\ref{fig:radial} , where the four axis of the radar chart represent the
values of percent which users rated positively this factors with respect to
their interaction with the context-aware recommender system.  The accurate values are
\textit{Usability 83\%, Satisfaction 84\%, Easy of use  92\%, and Easy of Learning 81\%.}
\begin{figure*}
\centering
\small
\captionsetup{font=footnotesize}
\includegraphics[width=0.8\textwidth]{img/radial.png}%[scale=0.5]{img/radial.png} 
\caption{\small{The radar chart that depicts the four axis evaluated in the questionnaire.}}
\label{fig:radial}   
\end{figure*}



\chapter{Conclusions and future work} \label{sec:5}

We observed the users behaviour to identify the most frecuently difficults and
doubts about tasks. We did a brief interview with users after the test in order
to understand their  feelings or mood, their ideas about the experience, and
overall, their opinion about the context-aware recommender system.  The
conclusions are based in user's comments, then the main errors in the system
interface are summarized in three points:
\begin{enumerate}  
\item  Incomplete information for user, i.e., the system doesn't had enough and clear information to be a friendly interface, and therefore the user couldn't do easily a task.
\item Fails in design, because of unordered elements in the screen, in other words, the elements are not in the correct site into the screen to be easily identified per users.
\item Fails in the language and confusion, because of the english language is not the native language of the users.
\end{enumerate}

The three points mentioned are related to the null values in data table (see
Table \ref{tab:datausers}), some users didn't the task because they were
confused, so they decided to omit the task. The null values weren't took in
account when the median was calculated (see Table \ref{tab:ic}).\\  The USE
questionnaire was useful to identify the weaknesses in the context-aware
recommender system.  The percent is upper of the acceptable (80\%), the results
allow to say that the system has a good performance. \\ For the future work we
proposed to improve the problems found in the user interface, so the proposals
are the following:

\begin{enumerate}  
\item  Redesign the user interface could helps to be more friendly for users. Due to the issues, the redesign involves: 
  \begin{enumerate}  
  \item Analyze the amount of information enough for a easy understanding, i.e., how much information the user needs seeing without overload it.
  \item Modify the tasks descriptions in the most simple way to avoid confusion.
  \item Add more language functionalities for to facilitate the tasks for users.
  \end{enumerate}
\item  To apply the usability test again with the changes in the interface in order to observe the level of improves and to compare the results. 
\item  Apply an statistical test to analize the results.
\item  Add collaborative filtering based on model (matrix factorization technique) within the context-aware recommender system in order to improve the level of user satisfaction in the context. 
\item  Add any contextual factors (such as companion, time of day, budget, etc.) in order to include more context information that could be relevant in the recommendations.
\end{enumerate}


\prefacesection{Publications}
%\begin{singlespace}
\begin{enumerate}
\item \textit{Restaurant Recommendations based on a Domain Model and Fuzzy Rules.  Xochilt Ram\'irez-Garc\'ia, Mario Garc\'ia-Vald\'ez. Recent Advances on Hybrid Approaches for Designing Intelligent Systems. Springer International Publishing Switzerland. (2012).}
\item \textit{Post-filtering for a Context-Aware Recommender System. Xochilt Ram\'irez-Garc\'ia, Mario Garc\'ia-Vald\'ez. Recent Advances on Hybrid Approaches for Designing Intelligent Systems . Springer International Publishing Switzerland. (2013).}
\item \textit{Recomendaciones contextuales basadas en el enfoque de post-filtrado. Xochilt Ram\'irez-Garc\'ia, Mario Garc\'ia-Vald\'ez. Modelado computacional de Habilidades Linguisticas y Visuales. Vol.74. Research in Computer Sciences, IPN. (2014).}
\item  \textit{Context-aware Recommender System Based in Pre-filtering Approach and Fuzzy Rules. Xochilt Ram\'irez-Garc\'ia, Mario Garc\'ia-Vald\'ez. Recent Advances on Hybrid Approaches for Designing Intelligent Systems . Springer International Publishing Switzerland. (2014).}
%\item  \textit{Context-Aware Recommender System Using Collaborative Filtering, Content-Based Algorithm and Fuzzy Rules. Xochilt Ram\'irez-Garc\'ia, Mario Garc\'ia-Vald\'ez, 2016.}
%\item \textit{A Hybrid Context-aware Recommender System for Restaurants. Xochilt Ram\'irez-Garc\'ia, Mario Garc\'ia-Vald\'ez, 2016.}
\end{enumerate}
%\end{singlespace}
\appendix
\chapter{Pseudocode}\label{apendixa}

\begin{algorithm}
\caption{Get Cosine similarity values}
\begin{algorithmic} 
\REQUIRE The list of itemProfilesUser and  itemProfilesAll in binary format.
\ENSURE The list of cosine similairty value for each item of the itemProfilesUser  
with each element of itemProfilesAll. 
\STATE $allProfiles \leftarrow $[ ]
\FOR {$itemu$ to size of $itemProfilesUser$}
\FOR {$itema$ to size of $itemProfilesAll$}
\IF {$itemu$ = $itema$}
\STATE jump next item
\ELSE
\STATE $cosineSimilarityValue \leftarrow $ among $itemu$ and $itema$
\STATE $itemProfiles \leftarrow itemu, itema, cosineSimilarityValue$
\ENDIF
\ENDFOR
\ENDFOR
\RETURN $allProfiles$
\end{algorithmic}
\end{algorithm}
%%----------------------------------------------------------------------------
%%----------------------------------------------------------------------------
\begin{algorithm}
\caption{Collaborative filtering algorithm}
\begin{algorithmic} 
\REQUIRE The userId.
\ENSURE The Top-N list of recommendations for the current user. 
%\COMMENT {Este es un comentario...}
\STATE $ratingMatrix \leftarrow {allRatings}$
\STATE Call $Recommendations \leftarrow getRecommendations()$ module
\RETURN $Recommendations$
\end{algorithmic}
\end{algorithm}
%%----------------------------------------------------------------------------
%%----------------------------------------------------------------------------
\begin{algorithm}
\small
\caption{Content-Based Algorithm}
\begin{algorithmic} 
\REQUIRE The user id.
\ENSURE The Top-N list of recommendations.
\STATE $RV \leftarrow$ All items that user rated with 5
\FOR {$item$ to size of $RV$}
\IF {$item$ is not in $RV$}
\STATE $UV \leftarrow itemid$
\ENDIF
\ENDFOR
\STATE $allItems \leftarrow$ [ ]
\STATE $getItemsProfilesUser \leftarrow$ Binary vectors of RV
\STATE $allRatings \leftarrow$ Rating matrix
\FOR {$item$ to size of $allRatings$}
\IF {$itemid$ is not in $allItems$}
\STATE $allItems \leftarrow item$
\ENDIF
\ENDFOR
\STATE $getAllItemsProfiles \leftarrow$ Binary vectors of allItems
\STATE $getCosineSim \leftarrow$ {{getItemsProfilesUser},{getAllItemsProfiles}}
\FOR {$item$ to size of $highCosineSim$}
\IF {$itemsimilarity \geq  0.8$}
\STATE $highCosineSim \leftarrow item$
\ENDIF
\ENDFOR
\STATE Sort $highCosineSim$ list
\RETURN $itemProfiles$
\end{algorithmic}
\end{algorithm}
%%----------------------------------------------------------------------------
%%----------------------------------------------------------------------------
\begin{algorithm}
\small
\caption{Get item profiles}
\begin{algorithmic} 
\REQUIRE The UV vector, allItems vector and boolean value of userProfile.
\ENSURE The list of ítemProfiles in binary vectors. 
\IF {$userProfile$ \TRUE }
\STATE $getItemsProfilesUser \leftarrow UV$
\FOR {$itemp$ to size of $UV$}
\STATE get binary vector of itemp
\STATE $itemProfiles \leftarrow itemp$
\ENDFOR
\ELSE
\STATE $allItemProfiles \leftarrow allItems$
\FOR {$itemp$ to size of $allItems$}
\STATE get binary vector of itemp
\STATE $itemProfiles \leftarrow itemp$
\ENDFOR
\ENDIF
\RETURN $itemProfiles$
\end{algorithmic}
\end{algorithm}
%%----------------------------------------------------------------------------
%%----------------------------------------------------------------------------
\begin{algorithm}
\small
\caption{Calculate Cosine similarity}
\begin{algorithmic} 
\REQUIRE The itemProfileUser and itemProfileAll, both vectors in binary format.
\ENSURE The cosine similarity value.
\STATE $sum \leftarrow 0$
\STATE $normaItemUser \leftarrow 0$
\STATE $normaItemAll \leftarrow 0$
\FOR {$position$ to size of $itemProfileUser$}
\STATE $sumProduct \leftarrow sumProduct+(itemProfileUser[position]*itemProfileAll[position])$
\ENDFOR
\FOR {$item$ to size of $itemProfileUser$}
\STATE $normaItemUser \leftarrow normaItemUser + itemProfileUser[item]^2$
\ENDFOR
\FOR {$item$ to size of $itemProfileAll$}
\STATE $normaItemAll \leftarrow normaItemAll+itemProfileAll[item]^2$
\ENDFOR
\STATE $squareRootUser \leftarrow squareroot(normaItemUser)$
\STATE $squareRootAll   \leftarrow squareroot(normaItemAll)$
\STATE $cosineSimilarity \leftarrow sumProduct/(squareRootUser*squareRootAll)$
\RETURN $cosineSimilarity$
\end{algorithmic}
\end{algorithm}
%%----------------------------------------------------------------------------
%%----------------------------------------------------------------------------
\begin{algorithm}
\caption{Create a binary vector of item profile}
\begin{algorithmic} 
\REQUIRE The ítem profile content in r.
\ENSURE The ítemProfile of r in a binary vector.
\STATE $price \leftarrow $ [4]
\STATE $payment \leftarrow $ [2]
\STATE $alcohol \leftarrow $ [2]
\STATE $smokingarea \leftarrow $ [2]
\STATE $dresscode \leftarrow $ [3]
\STATE $parking \leftarrow $ [3]
\STATE $installation \leftarrow $ [4]
\STATE $atmosphere \leftarrow $ [5]
\STATE $cuisine \leftarrow $ [30]
\STATE $price$[$positionPriceId-1$] $\leftarrow 1$
\STATE $payment$[$positionPriceId-1$] $\leftarrow 1$
\STATE $alcohol$[$positionPriceId-1$] $\leftarrow 1$
\STATE $smokingarea$[$positionPriceId-1$] $\leftarrow 1$
\STATE $dresscode$[$positionPriceId-1$] $\leftarrow 1$
\STATE $parking$[$positionPriceId-1$] $\leftarrow 1$
\STATE $installation$[$positionPriceId-1$] $\leftarrow 1$
\STATE $atmosphere$[$positionPriceId-1$] $\leftarrow 1$
\STATE $cuisine$[$positionPriceId-1$] $\leftarrow 1$
\STATE $itemProfile$ $\leftarrow price+payment+alcohol+smookingarea+dresscode+parking+installation+atmosphere+cuisine$
\RETURN $itemProfile$
\end{algorithmic}
\end{algorithm}
%%----------------------------------------------------------------------------
%%----------------------------------------------------------------------------
\begin{algorithm}
\caption{Get recommendations}
\begin{algorithmic} 
\REQUIRE The currentUser and ratingMatrix.
\ENSURE The Top-N list of recommendations for the current user. 
\STATE Dictionaries $totals \leftarrow \{ \}, sumSimilarity \leftarrow \{ \}$ 
\STATE $predictions \leftarrow $[ ]
\FOR {$otherUser$ to size of $ratingMatrix$}
\IF {$otherUser$ = $currentUser$}
\STATE jump next $otherUser$
\ENDIF
\STATE $similarityValue \leftarrow $ get pearsonSimilarity
\IF {$similarityValue \leq 0 $}
\STATE jump next $otherUser$
\ENDIF
\FOR {$item$ to size of $profileOther$}
\IF {$item$ is not in $profileUser$}
\IF {$profileUser[item]=0$}
\STATE Set in $totals \leftarrow item$
\STATE $totals[item]$ Add $ratingMatrix[otherUser][item]*similarityValue$
\STATE Set in $sunSimilarity \leftarrow item$
\STATE $sumSimilarity$ Add $similarityValue$
\ENDIF
\ENDIF
\ENDFOR
\ENDFOR
\FOR {each $(item,total)$ in $totals$}
\STATE $predictions \leftarrow [(total/sumSimilarity[item], item)]$
\ENDFOR
\STATE Ranking of $predictions$
\RETURN $predictions$
\end{algorithmic}
\end{algorithm}
%%----------------------------------------------------------------------------
%%----------------------------------------------------------------------------
\begin{algorithm}
\caption{Get Pearson correlation}
\begin{algorithmic} 
\REQUIRE The currentUser, otherUser and preferences.
\ENSURE The pearsonCorrelation score.
\STATE Dictionaries $itemsRatedMutually \leftarrow \{ \}$ 
\FOR {each $item$ in preferences of $currentUser$}
\IF {$item$ is in preferences of $currentUser$}
\STATE jump next $itemsRatedMutually[item] \leftarrow 1$
\ENDIF
\ENDFOR
\STATE $numberElements \leftarrow $ size of $itemsRatedMutually$
\IF {$itemsRatedMutually = 0 $}
\RETURN $0$
\ENDIF
\FOR {$item$ to size of $itemsRatedManually$ to get all preferences}
\STATE $sumCurrentUser \leftarrow preferences[currentUser][item]$
\STATE $sumOtherUser \leftarrow preferences[otherUser][item]$
\ENDFOR
\FOR {$item$ to size of $itemsRatedManually$ to get squares}
\STATE $squareCurrentUser \leftarrow square(preferences[currentUser][item])^2$
\STATE $squareOtherUser \leftarrow square(preferences[otherUser][item])^2$
\ENDFOR
\FOR {$item$ to size of $itemsRatedManually$ to get sum of products}
\STATE $sumProduct \leftarrow preferences[currentUser][item]*preferences[otherUser][item]$
\ENDFOR
\STATE $pearsonNumerator \leftarrow sumProduct-((sumCurrentUser*sumOtherUser)/numberElements)$
\STATE $pearsonDenominator \leftarrow square(squareCurrentUser-((sumCurrentUser)^2/numberElements)*squareOtherUser-((sumOtherUser)^2/numberElements))$
\STATE $pearsonCorrelation \leftarrow pearsonNumerator/pearsonDenominator$
\RETURN $pearsonCorrelation$ among two users
\end{algorithmic}
\end{algorithm}
%%----------------------------------------------------------------------------
%%----------------------------------------------------------------------------
\begin{algorithm}
\caption{Matrix factorization}
\begin{algorithmic}
\REQUIRE R is a matrix to be factorized, dimension N * M, P an initial matrix of dimension N * K, Q an initial matrix of dimension M * K, K is the number of latent features, steps for the maximum number of steps to perform the optimization,  alpha is the learning rate and  beta  is the regularization parameter.
\ENSURE The factorized matrix P and Q.
\STATE $alpha \leftarrow 0.0001, beta \leftarrow 0.001$ 
\STATE $QMatrix \leftarrow QMatrix * T $ 
\FOR {$step$ to $rangeSteps$}
\FOR {$i$ to size of $RMatrix$}
\FOR {$j$ to size of $RMatrix[i]$}
\IF {$RMatrix[i][j] > 0$}
\STATE $e_{i,j} \leftarrow RMatrix[i][j]-dotProduct(PMatrix[i to end],QMatrix[init to j])$
\ENDIF
\FOR {$k$ to range of $KFactors$}
\STATE $PMatrix[i][k] \leftarrow PMatrix[i][k]+alpha * (2*e_{i,j}*QMatrix[k][j] - beta *PMatrix[i][k])$
\STATE $QMatrix[k][j] \leftarrow QMatrix[k][j]+alpha * (2*e_{i,j}*PMatrix[i][k] - beta *QMatrix[k][j])$
\ENDFOR
\ENDFOR
\ENDFOR
\STATE $eR \leftarrow dotProduct (PMatrix * QMatrix)$
\FOR {$i$ to range of $RMatrix$}
\FOR {$j$ to size of $RMatrix[i]$}
\IF {$RMatrix[i][j] > 0$}
\STATE $e \leftarrow e + (beta/2) * PMatrix[i][k]^2 + QMatrix[i][j]^2$
\ENDIF
\ENDFOR
\ENDFOR
\IF {$e < 0$}
\STATE $break$
\ENDIF
\ENDFOR
\RETURN $PMatrix, QMatrix * T$ 
\end{algorithmic}
\end{algorithm}


\chapter{USE Questionnaire}\label{appendixb}
\begin{singlespace}
\small
\textbf{Usefulness}
\begin{itemize}
\item It helps me be more effective.
\item It helps me be more productive.
\item It is useful.
\item It gives me more control over the activities in my life.
\item It makes the things I want to accomplish easier to get done.
\item It saves me time when I use it.
\item It meets my needs.
\item It does everything I would expect it to do.
\end{itemize}
\small
\textbf{Ease of Use}
\begin{itemize}
\item It is easy to use.
\item It is simple to use.
\item It is user friendly.
\item It requires the fewest steps possible to accomplish what I want to do with it.
\item It is flexible.
\item Using it is effortless.
\item I can use it without written instructions.
\item I don't notice any inconsistencies as I use it.
\item Both occasional and regular users would like it.
\item I can recover from mistakes quickly and easily.
\item I can use it successfully every time.
\end{itemize}
\small
\textbf{Ease of Learning}
\begin{itemize}
\item I learned to use it quickly.
\item I easily remember how to use it. • It is easy to learn to use it.
\item I quickly became skillful with it.
\end{itemize}
\small
\textbf{Satisfaction}
\begin{itemize}
\item I am satisfied with it.
\item I would recommend it to a friend. 
\item It is fun to use.
\item It works the way I want it to work.
\item It is wonderful.
\item I feel I need to have it.
\item It is pleasant to use.
\end{itemize}
\small
\end{singlespace}
\textit{Source: From the work of Lund (2001).}
Note: Users rate agreement with these statements on a 7-point Likert
scale, ranging from strongly disagree to strongly agree. Statements in
italics were found to weight less heavily than the others.


\chapter{Technical support of installation}\label{apendixc}


Dependencies of the application:
\begin{singlespace}
\begin{itemize}
\item Django framework 1.7. Url: \textit{https://www.djangoproject.com/download/}
\item Django-registration library. Url: \textit{https://pypi.python.org/pypi/django-registration}
\item Django-countries library. Url: \textit{https://pypi.python.org/pypi/django-countries}
\item Django-geoposition library. Url: \textit{https://pypi.python.org/pypi/django-geoposition}
\item Python-dateutil library. Url: \textit{https://pypi.python.org/pypi/python-dateutil/2.4.1}
\item Pyproj library. Url: \textit{https://pypi.python.org/pypi/pyproj?}
\item Numpy library. Url: \textit{https://pypi.python.org/pypi/numpy}
\item PostgreSQL database. Url: \textit{http://www.postgresql.org/}
\item Psycopg2 connection to database. Url: \textit{http://initd.org/psycopg/docs/install.html}
\end{itemize}
\end{singlespace}


\chapter{System interfaces}\label{appendixd}

\begin{figure}[ht!]
   \captionsetup{font=footnotesize}
   \centering
   %%----primera subfigura----
   \subfloat[]{
    \label{fig:a}
    \fbox{\includegraphics[width=0.7\textwidth]{img/recomet-home.png}}}
   \hspace{0.01\linewidth}
   %%----segunda subfigura----
   % \subfloat[]{
   %      \label{fig:mf:b} 
   %      \fbox{\includegraphics[width=6.5cm]{img/recomet-recom.png}}}
   \caption{Home page of the system prototype.
   }
   \label{fig:home} 
\end{figure}

\begin{figure}[ht!]
   \captionsetup{font=footnotesize}
   \centering
   %%----primera subfigura----
   \subfloat[]{
        \label{fig:mf:a}
        \fbox{\includegraphics[width=0.7\textwidth]{img/recomet-myprofile.png}}}
   \hspace{0.1\linewidth}
   %%----segunda subfigura----
   \subfloat[]{
        \label{fig:mf:b} 
       \fbox{ \includegraphics[width=0.7\textwidth]{img/recomet-wishlist.png}}}\\[20pt]
   %%----tercera subfigura----
   % \subfloat[]{
   %      \label{fig:mf:c} 
   %      \includegraphics[width=0.6\textwidth]{img/recomet-wishlist.png}}
    %\hspace{0.1\linewidth}
   \caption{\textbf{a)}\textit{My profile}(user profile) and \textbf{b)}\textit{My wishlist}(the user whishlist) interfaces of the system.
   }
   \label{fig:wishlist} 
\end{figure}


%% Cap'itulos incluidos despues del comando \appendix aparecen como ap'endices
%% de la tesis.

%% Incluir la bibliograf'ia. Mirar el archivo "biblio.bib" para m'as detales
%% y un ejemplo.
\bibliography{biblio}

\end{document}
