\chapter{Introduction} \label{introduction} 

This thesis represents a state of the art of recommender systems in
the computing field. Through a study of literature it was done an
analysis of the different approaches applied to context-aware
recommender systems, as well as their strengths and weaknesses. The
purpose of this research is to contribute in the knowledge of the use of
context in recommender systems and to propose a hybrid method for
recommendations. Equally important in the method is the fuzzy
inference system implemented, because this system represents a
technique for recommendation else. These goals were achieved by means
of the collection of contextual information used in the system
through questionnaires and several experiments to validate the
method.

\section{Motivation}

Often people need take decisions, even although their experience is
not enought to decide among the possible alternatives. Generally,
persons trust in recommendations of anyone else, to taking desitions
according their personal interest. There is a large amount of
information generated in a society, so, the lack of experience of
persons highlights the importance of having automatic methods to
filter relevant information that helps to taking desitions. Every day,
a lot of information is generated in Internet and it is available for
users, then the overload information problem arises, in a manner that
users can't identify the relevant information of the irrelevant. The
users need tools to facilitate tasks, in a manner that this tools
provide a recommendation service that helps to use efficiently the
available information. The overload of information and the lack of
experience of users promotes the need of automatic search engines 
that contributes to taking desitions through recommendations of
personalized products or services.

\section{Aims}

The main contribution is in the state of the art in the field of
context-aware recommender systems, the goal is to provide a
\textit{useful knowledge} to utilize context-aware recommender systems
using a hybrid method that is easily implemented in different domains
such as e-learning, movies, music, tourism, etc. In this particular
case, the restaurants domain is used as a case of study to  test the
propose method.   The particular aims are following:
\begin{itemize}  
\item To elaborate an analysis about state of the art in the field
of context-aware recommender systems through  the revision of
literature. 
\item To evaluate the algorithms that represent alternatives of 
solution for the problem. 
\item To propose new recommendation techniques for context applied 
in a case of study.  
\item To develop a prototype of context-aware recommender system 
using the method. 
\item To evaluate the performance of the method using 
different datasets in order to observe the system behaviour 
for each particular case.
\end{itemize} 

\section{Context-awareness}

Traditional recommender systems provides suggestions of useful
items for a certain user. The suggestion relates to various 
decision-making processes, for instance, what items to buy, what music to
listen to or what on-line news to read. \textit{Item} is the general
term to denote what product or service the system recommends for each
user. A recommender system normally focuses only in a type of item
\cite{resnick1997recommender}.
The improvements of previous recommender systems are focused in the
\textit{integration of context} in its recommendation process. 
The importance of contextual information has been recognized 
in different domains and disciplines.  % Agregar referencias
The idea of context-aware computing is to provide
information or services for the user based in the user’s situation
\cite{dey2001understanding}. In order to do that, the application 
needs to obtain situational data, process it and make use of it 
in a way that benefits the user. \\ 
Context is a concept not easy to define, it is related with 
several disciplines that propose different definitions.\\  %referecnicas
Some authors make a comparison of definitions such as Bazire et.al.
\cite{bazire2005understanding} that compares the context in different
fields and conclude that is complicated makes a unifying definition of
context because of the nature of the concept in the disciplines.
In computer sciences Fischer G.\cite{fischer2012context} defines
context as the interaction between humans and computers in socio-
technical systems that takes place in a certain context referring to
the \textit{physical} and \textit{social situation} in which 
computational devices and environments are embedded.
Also identifies the important aspects to consider when the context is
used in that framework: how it is the contextual data obtained, how
the context is represented and what goals and purposes the
context has when used in a particular application. \\
The definition most used in the field of recommender systems to
define context is proposed by Dey\cite{dey2001understanding}:
\textit{``Context is any information that can be used to characterize
the situation of an entity. An entity is a person, place or object
that is considered relevant to the interaction between a user and  an
application, including the user and applications themselves.''}  This
definition makes it easier to define the contextual factors in a
specific application. This definition makes it easier to define the
contextual factors in a specific application. For instance, in a
tourist guide application the entities can be companion(friends,
family, couple), placeof interest, season and weather, these could be
considered as relevant contextual factors that help the recommender
system to provide items adjusted the situational data of the user.\\
\textit{Context-aware recommender systems} are gaining even more
attention because of their performance and implementation for
different domains, the  way to improve personalized recommendations
based in contextual factors is an important technique to increase the
benefits in  many domains. For instance, taking in to account the
\textit{hour of the day},  or the \textit{day of the week} when
recommending restaurants could  filter out restaurants that are
currently closed or near closing time, when the user receives this
information in real time, the user has the  way for taking
alternatives of restaurants that provide services. Nowadays, many
companies are incorporating some type of context (as time, location or
companion) in their recommendation engines,  the application can be
found in fields such as e-commerce\cite{schafer1999recommender},
\cite{bulander2005enabling}; music\cite{ricci2012context}, 
\cite{baltrunas2011incarmusic}, \cite{huq2010automated}; 
places of interest\cite{baltrunas2012context},
movies\cite{eyjolfsdottir2010moviegen}; vacation
packages\cite{liu2011personalized}, \cite{liu2014cocktail}; 
travel guides\cite{savage2012m}; e-learning\cite{ortigosa2010entornos} 
and restaurants\cite{chu2013chinese}.\\
Plus, context can be used to improve the user satisfaction  in
recommender systems, thus the quality and accuracy of predictions  
is improved too.\\
In this thesis the proposed method is implemented in a context-aware 
recommender systems and evaluated in a restaurant domain. The method 
use three recommendations techniques:
\begin{enumerate}  
\item \textit{Fuzzy Inference System} This a rule based recommender 
defined by an expert in the domain, it considers as linguistic variables 
for instance for the restaurant domain: \textit{ratings average:
(low,medium,high)}, \textit{price of restaurant:(cheap, average, 
expensive)} and \textit{number of ratings of item:(few,several,many)} 
to infer how relevant a restaurant is for the user. Generally, 
this recommendation is based on the popularity of each item in the 
user community.
\item \textit{Content-based technique} utilizes the item profiles 
to compare how \textit{similar} is an item is with respect to 
another, i.e. restaurants that are \textit{similar} (same cuisine, 
ambient, price range)  to others that the user has rated favorably. 
The idea is to find items with similar features. 
\item \textit{Collaborative filtering technique} is based on the user
profile to identify user's preferences and to find neighbors that
have the same taste. The recommendation consist in the suggestions of
other users with similar tastes that rated restaurants again in a
similar way but where have not been rated by the current user. A Top-N
list of restaurants is obtained to recommend for the user.
\end{enumerate} 
The results of the three techniques are recommendations for the user,
then the next step are adjusted 
for user in context.
This is the last step and is represented as \textit{context filter} in
the method, then the recommender system obtains a list of
contextualized recommendations. The proposed methodology works
simultaneously to obtain recommendations, the hybrid method allows the
recommender system to generate suggestions despite the scarse
user information. When the system faces up the the cold-start
problem or the over-specialization problem, the hybrid method
responses properly to show results for the user. These problems  
are described in a later section. \\
The expert's recommendation contributes to this goal, the rules 
specified in the fuzzy inference system help to infer the most suitable 
restaurant considering the only the input variables and the knowledge 
represented by the rules.\\
The context-based filter then makes recommendations based on those
items that are appropiate to the context of the user to produce
the final recommendation list. The context-aware recommender
system tries to increase the level of user satisfaction. To meet this
goal two metrics were utilized to test the system in an on-line test.
% Hablar del uso de pruebas de satisfaccion y usabilidad,
% se debe justificar en base a esto el por que son
% aparentemente pocos usuarios.
% Menciona también las pruebas de error de recomendacion
% Fijate como cambié lo de arquitectura por metodología
% ya que si te das cuenta este proceso se puede utilizar de manera
% generica para otros sistemas de recomendación
% La implementación especifica es la de recommet
% pero lo mismo puede usarse para Protoboard.
% Al principio debemos hablar de manera general.
Ten users were selected to interact with the system, subsequently 
an analysis about the system performance and highlight the issues 
that users had.\\

\section{Outline}

The rest of this thesis is organized as follows: 
\begin{itemize}  
\item Chapter 2 begins since the traditional recommender systems,
their methods and techniques to improve recommendations, as so the
problems to face up this systems. Subsequently, the hybrid methods
used in applications are explained, their strenghts and weaknesses
for each hibridation and the most proper domain of application.
Finally, context-aware recommender systems are mentioned, in the same
way, we speak about the advantages and disadvantages of the use of
context in recommender systems and the application domains. Some
examples for each type of systems are mentioned in the chapter.
%%%falta continuar..........
\item chapter 3 contains the background of the related topics, 
\item chapter 4  describes the proposed context-aware recommender methodology describing each one of its components, 
\item then in chapter 5 the experimental set up and the results are presented, 
\item chapter 6 describes the evaluation of a case study a restaurant domain and finally, 
\item chapter 7 presents conclusions and a proposal for future work .
\end{itemize}  




