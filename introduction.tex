\chapter{Introduction} \label{introduction} 

The purpose of this research is to contribute in the knowledge of the
<<<<<<< HEAD
use of context in recommender systems and to propose a method
for recommendations. The method involves recommendation techniques and a fuzzy inference system, both techniques make recommendations.
The goals were achieved by means of the collection of contextual
information through questionnaires and several experiments to validate the method.

\section{Motivation}

Often people need take desitions, even although their experience is
not enought to decide among the possible alternatives. Overall,
persons trust in recommendations of someone else, to taking desitions
according their personal interest.\\
There is a large amount of information generated in a society, so, the
lack of experience of persons highlights the importance to provide
automatic methods to filter relevant information that helps to taking
desitions. \\Every day, a lot of information is generated in Internet
and it is available for users, then the overload information problem
arises, in a manner that users can't identify the relevant information
of the irrelevant. The users need tools to facilitate tasks, as well
as tools to provide a recommendation service that helps to use
efficiently the available information. The overload of information and
the lack of experience of users promotes the need of automatic search
engines that contributes to taking desitions through recommendations
of personalized products or services. \\ By other hand, in recent
years  the mobile computing rocketed its importance because of the
impact of its use in daily life, any application can be applied in
mobiles and anywhere it can be used, their limitations are every time
less. For this reason, new technologies for use of context in mobile
applications arise, intelligent systems can take advantage of the
benefits that technology provides to manage the context that changing
constantly. Mobile and ubiquitous computing\cite{noguera2012mobile}
\cite{chiou2010adaptive} are proposing a wide variety of applications
that need recommendation engines using context for meet their
purposes.
=======
use of context in recommender systems and to propose a hybrid method
for recommendations. Equally important in the method is the fuzzy
inference system because represents a
technique of recommendation also. The goals were achieved by means of
the collection of contextual information used in the system through
questionnaires and several experiments to validate the method.

\section{Motivation}

Often people need take decisions, even although their experience is
not enought to decide among the possible alternatives. Generally,
persons trust in recommendations of anyone else, to taking desitions
according their personal interest. There is a large amount of
information generated in a society, so, the lack of experience of
persons highlights the importance of having automatic methods to
filter relevant information that helps to taking desitions. \\Every day,
a lot of information is generated in Internet and it is available for
users, then the overload information problem arises, in a manner that
users can't identify the relevant information of the irrelevant. The
users need tools to facilitate tasks, in a manner that this tools
provide a recommendation service that helps to use efficiently the
available information. The overload of information and the lack of
experience of users promotes the need of automatic search engines
that contributes to taking desitions through recommendations of
personalized products or services. \\
By other hand, in recent years  the mobile computing rocketed its
importance because of the impact of its use, any application can be
applied in mobiles and anywhere it can be used, their limitations
are every time less. For this reason, new technologies for use of
context in mobile applications arise, intelligent systems can take
advantage of the benefits that technology facilitates to meet tasks.
Mobile and ubiquitous
computing\cite{noguera2012mobile},\cite{chiou2010adaptive} are
proposing a wide variety of applications that need recommendation
engines for meet their purposes, even more if one of their purpose 
is to increase sales.
Nowadays, It's important to propose new methods to use in the recent
technology, these methods will cover the needs of users and helps 
to improve mobile applications  when personalized systems are 
used to facilitate the user's tasks. The method proposed is used 
in order to help users to achieve their purposes.
>>>>>>> origin/master

\section{Context-awareness} \label{context-awareness}

Traditional recommender systems provides suggestions of useful items
for a certain user. The suggestion relates to various decision-making
processes, for instance, what items to buy, what music to listen to or
what on-line news to read. \textit{Item} is the general term to denote
what product or service the system recommends for each user. A
recommender system normally focuses only in a type of item
\cite{resnick1997recommender}.
The improvements of previous recommender systems are focused in the
\textit{integration of context} in its recommendation process. 
The idea of \textit{context-aware computing} is to provide
information or services for the user based in the user's situation
\cite{dey2001understanding}. In order to do that, the application 
needs to obtain situational data, process it and make use of it 
in a manner that benefits the user. \\ 
\textit{Context} is a concept not easy to define, it is related with
several disciplines that propose different definitions. For example,
the authors Bazire et.al.\cite{bazire2005understanding} compare the
context in different fields and conclude that is complicated makes a
unifying definition of context because of the nature of the concept in
the disciplines. In computer sciences Fischer
G.\cite{fischer2012context} defines context as the interaction between
humans and computers in socio-technical systems that takes place in a
certain context referring to the \textit{physical} and \textit{social
situation} in which  computational devices and environments are
embedded. Also identifies the important aspects to consider when the
context is used in that framework: how it is the contextual data
obtained, how the context is represented and what goals and purposes
the context has when used in a particular application. \\
The definition most used in the field of recommender systems to
define \textit{context} is proposed by Dey\cite{dey2001understanding}:
\textit{``Context is any information that can be used to characterize
the situation of an entity. An entity is a person, place or object
that is considered relevant to the interaction between a user and  an
application, including the user and applications themselves.''}  This
definition makes it easier to define the contextual factors in a
specific application. For instance, in a
tourist guide application the entities can be companion(friends,
family, couple), place of interest, season and weather, these could be
considered as relevant contextual factors that help the recommender
system to provide items adjusted the situational data of the user.\\
\textit{Context-aware recommender systems} are gaining even more
attention because of their performance and implementation for
different domains, the  way to improve personalized recommendations
based in contextual factors is an important technique to increase the
benefits in  many domains. For instance, taking in account the
\textit{hour of the day},  or the \textit{day of the week} when
recommending restaurants could  filter out restaurants that are
currently closed or near closing time, when the user receives this
information in real time, the user has the  way for taking
alternatives of restaurants that provide services. Nowadays, many
companies are incorporating some type of context (as time, location or
companion) in their recommendation engines,  the application can be
found in fields such as e-commerce\cite{schafer1999recommender},
\cite{bulander2005enabling}; music\cite{ricci2012context}, 
\cite{baltrunas2011incarmusic}, \cite{huq2010automated}; 
places of interest\cite{baltrunas2012context},
movies\cite{eyjolfsdottir2010moviegen}; vacation
packages\cite{liu2011personalized}, \cite{liu2014cocktail}; 
travel guides\cite{savage2012m}; e-learning\cite{ortigosa2010entornos} 
and restaurants\cite{chu2013chinese}.\\
Plus, context can be used to improve the user satisfaction  in
recommender systems, thus the quality and accuracy of predictions  
is improved too.\\
The proposed method is implemented in a context-aware 
recommender systems and uses three recommendations techniques:
\begin{enumerate} 
\item \textit{Fuzzy Inference System} This a rule based recommender
defined by an expert in the domain, it considers as linguistic
variables for instance for the restaurant domain: \textit{ratings
average: (low,medium,high)}, \textit{price of restaurant:(cheap,
average, expensive)} and \textit{number of ratings of
item:(few,several,many)} to infer how relevant a restaurant is for the
user. Generally, this recommendation is based on the popularity of
each item in the user community.
\item \textit{Content-based technique} utilizes the item profiles 
to compare how \textit{similar} is an item is with respect to 
another, i.e. restaurants that are \textit{similar} (same cuisine, 
ambient, price range)  to others that the user has rated favorably. 
The idea is to find items with similar features. 
\item \textit{Collaborative filtering technique} is based on the user
profile to identify user's preferences and to find neighbors that
have the same tastes. The recommendation consist in the suggestions of
other users with similar tastes that rated restaurants again in a
similar way but where have not been rated by the current user. A Top-N
list of restaurants is obtained to recommend for the user.
\end{enumerate} 
The results of the three techniques are recommendations for the user,
then the next step are adjusted  for user in context. This is the last
step and is represented as \textit{context filter} in the method, then
the recommender system obtains a list of contextualized
recommendations. \\The proposed method works simultaneously to
obtain recommendations, the hybrid method allows the recommender
system to generate suggestions despite the scarse user information.
When the system faces up the the cold-start problem or the over-
specialization problem, the hybrid method responses properly to show
results for the user. These problems are described in a later
section. \\
The expert's recommendation contributes to this goal, the rules
specified in the fuzzy inference system help to infer the most
suitable restaurant considering the only the input variables and the
knowledge represented by the rules.\\ The context-based filter then
makes recommendations based on those items that are appropiate to the
context of the user to produce the final recommendation list. \\
The method tries to increase the  level of user satisfaction.
To meet this goal two metrics were utilized to test the system:
\textit{task-success} and \textit{time-on-task}. These metrics allow
to measure the user experience, this offers so much more than just
simple observation. \\Usability metrics can help reveal patterns that
are hard or even impossible to see. Evaluating software with a small
sample size usually reveals the most obvious usability
problems\cite{albert2013measuring}.\\
Then, as a general rule of thumb, during the early stages of design,
we need fewer participants to identify the major usability issues. As
the design gets closer to completion, the tests should include more
<<<<<<< HEAD
participants to identify the remaining issues. Following this precept,
ten representative users were selected to test the system,
subsequently, it was realized an analysis about the system performance
and issues presented for users.\\
The results

%%%%%%%%%%%%%%%%%%%%%%%%%%%%%%%%%%%%%%%%%%%%%%%%%%%%%%%%%%%%%%%%%%%%%
=======
participants to identify the remaining issues\cite{albert2013measuring}.\\
Following this precept, ten representative users were selected to test 
the system, subsequently, it was realized an analysis about the 
system performance and issues presented in the system.\\
%%%%%%%%%%%%%%%%%%%%%%%%%%%%%%%%%%
>>>>>>> origin/master
%
% Menciona también las pruebas de error de recomendacion
% Profe, este comentario se refiere a las pruebas realizadas con los 
% algoritmos, los experimentos?
%
% Fijate como cambié lo de arquitectura por metodología
% ya que si te das cuenta este proceso se puede utilizar de manera
% generica para otros sistemas de recomendación
% La implementación especifica es la de recommet
% pero lo mismo puede usarse para Protoboard.
% Al principio debemos hablar de manera general.
<<<<<<< HEAD
%%%%%%%%%%%%%%%%%%%%%%%%%%%%%%%%%%%%%%%%%%%%%%%%%%%%%%%%%%%%%%%%%%%

\section{Aims}

The main contribution is to define a method of context-aware
recommender systems using different techniques of recommendation,
another goal is to provide a \textit{useful knowledge} 
to utilize context-aware recommender systems
using a hybrid method that is easily implemented in different domains
such as e-learning, movies, music, tourism, etc. In this particular
case, the restaurants domain is used as a case of study to test the
propose method. \\ Another important contribution is the use of fuzzy rules in the
proposed method, this allows the use of linguistic
information closer to the real context of the people, i.e., 
the method uses this technique to analyze the user preferences.
The particular aims are following:
=======
%FALTA HABLAR DE LO DIFUSO....COMO SE IMPLEMENTO...JUSTIFICACION...IMPORTANCIA.
%%%%%%%%%%%%%%%%%%%%%%%%%%%%%%%%%%
\section{Aims}

The contribution is to propose a method in the field of
context-aware systems, in order to provide a
\textit{useful knowledge} to utilize context in recommender systems
using a hybrid method that is easily implemented in different domains
such as e-learning, movies, music, tourism, etc. In this particular
case, the restaurants domain is used as a case of study to  test the
propose method. The particular aims are following:
>>>>>>> origin/master
\begin{itemize}  
\item To elaborate an analysis about state of the art in the field
of context-aware recommender systems through  the revision of
literature. 
<<<<<<< HEAD
\item To select the algorithms that represent alternatives of
solution for the problem to test their perfomance in different
domains.
\item Realize experiments with the proposed algorithms.
\item Based in previous experiments and results, to propose a hybrid method and apply it in a case of study.  
\item To define the fuzzy inference system that serves as recommender technique in the proposed method, as well as the variables and fuzzy rules involved.
\item To develop a prototype of context-aware recommender system 
using the proposed method.
=======
\item To select the algorithms that represent alternatives of 
solution for the problem to evaluate their perfomance in different domains.
\item To propose a hybrid method for context applied in a case of study.  
The method proposes several techniques for recommendations.
\item To develop a prototype of context-aware recommender system 
using the method in a specific domain.
>>>>>>> origin/master
\item To evaluate the performance of the method using 
different datasets in order to observe the system behaviour 
for each particular case.
\end{itemize} 

\section{Outline}

The rest of this thesis is organized as follows: 
\begin{itemize}  
\item Chapter 2 describes an in-depth study of current background and
related work is presented to give a general overview of recommender
systems and their evolution in recent years. This study includes the
traditional recommender systems, their methods and techniques to
improve recommendations, as so the problems to face up this systems.
Subsequently, the hybrid methods used in different applications, their
strenghts and weaknesses for each hibridation and the domains of
application. Finally, context-aware recommender systems are mentioned,
in the same way, we speak about the advantages and disadvantages of
the use of context in recommender systems.
\item Chapter 3 describes the fundamental concepts required to
understand the proposed method.
\item Chapter 4 presents a model of context-aware recommender system,
the proposed method  involves the paradigm of post-filtering in a restaurants
domain. This chapter include the overall explanation of data models and 
the method functionality, as well as its components for this case of study.
\item Chapter 5, the general results of different projects involved
are presented along with the validation of every experimentation. The
experiments were realized using different datasets and different
algorithms in order to find an optimal manner to reduce the error
level. This chapter also details the results for each experiment 
from a point of view of scientific results.
\item Chapter 6, after the development of context-aware recommender
system, it was evaluated the impact of context
in recommendation process. This chapter describes the usability tests
that were applied on-line in order to evaluate the satisfaction of
users. Details of the environment and the characteristics of the tests
are described, as well as the results of each one.
\item Chapter 7, this chapter concludes with a
summary of its contributions and  its limitations. It discuss final
conclusions and the proposals for the future work.
\end{itemize}  
At the end, this thesis includes several appendices describing
detailed technical aspects of the context-aware recommender system,
the pseudocode of algorithms, interfaces of the prototype of 
context-aware recommender system and experiment study
materials.



