\chapter{Introduction} \label{introduction} 

Recommender systems are a technique to provide suggestions of 
useful items for a user. The suggestion relate to various
decition-making processes, for instance, what items to buy, what music
to listen to or what on-line news to read. \textit{Item} is the general
term to denote what product or service  the system recomends for each
user. A recommender system normally focuses in a kind of item.
Recommender sistems arise to cover the lack of personal experience to
evaluate the overwhelming number of alternative items that a Web site
may offer\cite{resnick1997recommender}.
Recommender systems development initiated from a rather simple
observation: individuals often rely on recommendations provided by
others in making routine and daily decisions\cite{mahmood2009improving},
\cite{mcsherry2009differentially}. 
For example people commonly relies on suggestions when selecting a
book to read; for hire human resources in a company, the department
counts on recommendation letters in their recruiting decisions; and
when selecting a movie to watch, people tend to read and relies on the
movie reviews that a film critic has written.\\
Recommender systems tries to imitate this human behaviour, to
recommend a product in daily life is normally the most effective
manner to achieve in this systems that the user prefers an item in a
Web site. Over the time, the improvements for recommender systems are
focused in differents techniques that involves the context in the
recommendation process.  The importance of contextual information has
been recognized in different domains and disciplines. In recommender
systems, context implies the situations around of the user when it is
interacting with the system and all the information that this
situation represents.
% definir aqui Context?
The most formal definition is proposed by Annin K.
Dey\cite{dey2001understanding}: \textit{``Context is any information
that can be used to characterize the situation of an entity. An entity
is a person, place or object that is considered relevant to the
interaction between a user and an application, including the user and
applications themselves.''} This definition makes it easier to define
the important context in a specific recommender system.\\ 
However, some authors make an understandable comparison of different 
contexts such as Bazire et.al.\cite{bazire2005understanding} that compares 
different definitions of context in different fields and conclude that is
complicated makes a unifying definition of context because of the
nature of the concept in the disciplines. In computer sciences Fischer
G.\cite{fischer2012context} takes the context such as the interaction
between humans and computers in socio-technical systems that takes
place in a certain context referring to the physical and social
situation in which computational devices and environments are
embedded. Also identifies the important aspects to consider when the
context is used in its framework: how it is obtained the context, how
is represented the context and what objectives and purposes has the
context in a particular application is used. \\
Context-aware recommender systems are gaining ever more attention
because of its performance and adaptability for different domains, the
way to improve personalized recommendations based in contextual
factors is an important technique to increase the benefits in many
domains. Nowadays many companies are incorporating some kind of
context in their recommendation engines, the application covers fields
such as E-commerce\cite{schafer1999recommender}, \cite{bulander2005enabling};
music\cite{ricci2012context},\cite{baltrunas2011incarmusic}; 
places of interest\cite{baltrunas2012context},
movies\cite{eyjolfsdottir2010moviegen}; vacation packages\cite{liu2011personalized},
\cite{liu2014cocktail}; travel guides\cite{savage2012m}; e-learning\cite{ortigosa2010entornos} and restaurants\cite{chu2013chinese}.\\  
The majority of these recommender systems are focused on 
recommending relevant items for users without
additional contextual information such as time, location,
companion or place. However, recent systems incorporate the contextual
information to make recommendations in order to deliver items adjusted
to the users' current context. For instance, the site Sourcetone
interactive radio \cite{huq2010automated} when selecting a song for
the customer takes in consideration the listener's current mood in
order to provide a better recommendation. \\ The context is used for
increase important aspects such as user satisfaction, recommendation
quality, usefulness and more.\\
This research presents an architecture proposed for a context-aware
recommender system in the restaurant's domain. The objetive is to
demonstrate how the level of user satisfaction is increased through
the context when the system recommends restaurants in a specific
context. The architecture contains three techniques for this process:
\begin{enumerate}  

\item \textit{Fuzzy Inference System} to recommend such as an expert
in restaurants, it considers the inputs \textit{ratings average},
\textit{price of restaurant} and \textit{number of votes} to infer how
relevant a restaurant is for the user.  Generally, this recommendation
is based in the popularity of each restaurant into the user community.
\item \textit{Content-based technique} utilizes the restaurant profiles 
to compare how \textit{similar} is a restaurant with respect of another, i.e.  
the restaurants that are \textit{similar} to restaurants 
that the user rated with high rating. The idea is to find restaurants 
with similar features to recommend. 
\item \textit{Collaborative filtering
technique} is based in the user profile to identify the user
preferences and to find neighbors that have the same tastes. The
recommendation consist in the suggestions of other users with similar
tastes that rated restaurants that are not rated for the current user.
A top-N list of restaurants is obtained to recommend for the user. 
\end{enumerate}  
The results of the three techniques are the list of
recommendations for the user that  are adjusted in the user context.
This is the last step and is represented as \textit{context filter} in
the architecture, then the recommender system obtains a list of
contextualized recommendations. The architecture proposed works
simultaneously to obtain recommendations, the hybrid method allows the
recommender system to generate suggestions despite the scarse
of user information. When the system face up to the cold-start
problem or the over-specialization problem, the hybrid method
responses properly to show results for the user. An expert 
recommendation contributes to this goal, the rules specified in 
the Fuzzy Inference System helps to inferring the most suitable 
restaurant considering the users' opinions.\\
The context implementation makes better recommendations, the
restaurants matched in the same context of the user will be added
within the final recommendation list. The context-aware recommender
system tries to increase the level of user satisfaction. To meet this
goal two metrics were utilized to test the system in an on-line test.
Ten users were selected to interact with the system, subsequently an
analysis about the system performance and main issues highlited by the
users was realized. \\
The rest of this thesis is organized as follows: chapter 2 explains
the state of the art,  chapter 3 contains the background of the topics
related, chapter 4  describes the context-aware recommender system and
each one of its components, the chapter 5 describes the manner how the
experiments were done  and the results obtained, chapter 6 describes
the system evaluation in a restaurant domain, finally, chapter 7
presents the conclusions and future work proposed.




