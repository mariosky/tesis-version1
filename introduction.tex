\chapter{Introduction} \label{introduction} 


Esta tesis representa un estado del arte de los sistemas de
recomendación en el área de computación.  A través de un estudio de la
literatura se ha desarrollado un análisis de los diferentes enfoques
aplicados a los sistemas de recomendación sensibles al contexto
actuales, así como sus fortalezas y debilidades.  El estudio del
estado del arte se realizo con el fin de obtener una solida
contribución en el ámbito de los sistemas de recomendación sensibles
al contexto, esto se logró mediante las pruebas realizadas con el
método propuesto, la recolección de información contextual aplicada en
el sistema y los experimentos previos que marcaron el camino de la
investigación.

\section{Motivación}

En su rutina diaria,  las personas tienen la necesidad de tomar
decisiones frecuentemente, aunque no tengan la experiencia suficiente
para decidir entre las alternativas posibles. Generalmente, las
personas confían y toman en consideración las recomendaciones de
terceros para llevar a cabo decisiones de  interés personal.  Existe
una gran cantidad de información generada en una sociedad,  entonces,
la falta de experiencia de las personas hace que cada día sea mas
trascendente contar con métodos automáticos  de filtrado o selección
de información relevante para la toma de decisiones. Grandes
cantidades de información se generan en Internet diariamente y está
disponible para cualquier usuario, así surge el problema de la
saturación de información,  a tal grado que los usuarios son incapaces
de separar la información relevante de la irrelevante.  La necesidad
de herramientas que ofrezcan soluciones se ha vuelto imprescindible,
de manera que las  herramientas  que proveen un servicio de
recomendación proveen la ayuda necesaria para hacer un uso mas
provechoso de la información disponible. El crecimiento desmesurado de
información y la falta de experiencia de los usuarios evidencian la
necesidad de motores de búsquedas automáticos o semiautomáticos que
ayuden a la toma de decisiones mediante las recomendaciones de
productos o servicios personalizados.

\section{Objetivos}

El objetivo es la contribución al estado del arte de los sistemas de
recomendación sensibles al contexto, proveer conocimiento útil para la
creación de estos sistemas de recomendación híbridos basados en
contexto utilizando el método propuesto en un  dominio específico.  Se
definieron objetivos particulares logrados para el objetivo general: 
\begin{enumerate}
\item Hacer una análisis sobre el estado del arte de sistemas de
recomendación mediante el estudio de la literatura.  
\item Evaluar los algoritmos que representan alternativas de solución 
al problema.  
\item Proponer nuevas técnicas de recomendación basadas en contexto en un
dominio particular. 
\item Desarrollar un sistema de recomendación híbrido basado en contexto. 
\item Realizar experimentos para evaluar el rendimiento del algoritmo 
propuesto con diferentes conjuntos de datos contextuales con el fin 
de observar el comportamiento de los algoritmos propuestos. 
\item Desarrollar un prototipo de sistema de recomendación sensible 
al contexto en un dominio particular.
\end{enumerate}

\section{Hipotesis}

La hipótesis científica propuesta en este trabajo de investigación es
la siguiente:
H1: Can the algorithms based on contextual information improve the
limitations of the traditional recommender systems?
Ho: The algoritms based on contextual information not improve the
limitations of the traditional recommender systems.\\
Para validar la hipótesis se hicieron varios experimentos con los tres
paradigmas de recomendación contextual, los conjuntos de datos
utilizados fueron MovieLens, TripAdvisor,  FilmTrust  y  Restaurantes
de Tijuana.  Considerando los objetivos de la tesis, los resultados
obtenidos permiten afirmar que hay evidencia suficiente para rechazar
la hipótesis nula.\\
%%%%%%%%%%%%%%%%%%%%%%%%%%%%%%%%%%%%%%%%%%%%%%%%%%%%%%%%%%%%%%%%%%%
%%%%%%%%%%%%%%%%%%%%%%%%%%%%%%%%%%%%%%%%%%%%%%%%%%%%%%%%%%%%%%%%%%%%
Recommender systems are a technique to provide suggestions of 
useful items for a certain user. The suggestion relates to various
decision-making processes, for instance, what items to buy, what music
to listen to or what on-line news to read. \textit{Item} is the general
term to denote what product or service the system recommends for each
user. A recommender system normally focuses in a kind of item\cite{resnick1997recommender}.

% Recommender systems arise to cover the lack of personal experience to
% evaluate the overwhelming number of alternative items that a Web site
% may offer \cite{resnick1997recommender}.
Recommender systems development initiated from a rather simple
observation: individuals often rely on recommendations provided by
others in making routine and daily decisions \cite{mahmood2009improving},
\cite{mcsherry2009differentially}. 

% For example people commonly relies on suggestions when selecting a
% book to read; when hiring human resources in a company, the department
% counts on recommendation letters in their recruiting decisions; and
% when selecting a movie to watch, people tend to read and rely on 
% movie reviews that a film critic has written.\\
%% Debemos escoger uno o dos parrafos que expliquen lo de recomender systems
% si te fijas los tre parrafos anteriores son tres definiciones, como que empiezas
% el tema tres veces, en lugar de seguir describiendo, despues regresamos a que esta
% parte quede bien.

%%El siguiente parrafo no se entiende:
Recommender systems try to imitate this human behaviour, to
recommend a product in daily life is normally the most effective
manner to achieve in this systems that the user prefers an item in a
Web site.

Over the time, the improvements of recommender systems are
focused on differents techniques that involve the context in the
recommendation process. 

The importance of contextual information has
been recognized in different domains and disciplines. % Agregar referencias 
In recommender
systems, context implies the situations around of the user when it is % Creo que hay solo una situacion posible
interacting with the system and all the information that this
situation represents. 
% definir aqui Context?
In this work we use a definition proposed by Annin K.
Dey \cite{dey2001understanding}: \textit{``Context is any information
that can be used to characterize the situation of an entity. An entity
is a person, place or object that is considered relevant to the
interaction between a user and an application, including the user and
applications themselves.''} 
This definition makes it easier to define
the important context in a specific recommender system.\\  %% Aclara esto mejor definir el contexto importante?
% Mejor con un ejemplo?

However, some authors make an understandable comparison of different % A que te refieres con understandable?
contexts such as Bazire et.al. \cite{bazire2005understanding} that compares 
different definitions of context in different fields and conclude that is
complicated makes a unifying definition of context because of the
nature of the concept in the disciplines. In computer sciences Fischer
G. \cite{fischer2012context} defines context as the interaction
between humans and computers in socio-technical systems that takes
place in a certain context referring to the physical and social
situation in which computational devices and environments are
embedded. % Me gusta esta definicion tambien
Also identifies the important aspects to consider when the
context is used in that framework: how it is the contextual data obtained , how
the context is represented and what objectives and purposes the
context has when used in a particular application. \\

Context-aware recommender systems are gaining even more attention
because of their performance and adaptation for different domains, the % a que te refieres con adaptation?
way to improve personalized recommendations based in contextual 
factors is an important technique to increase the benefits in many
domains. % Da un ejemplo, For instance taking in to account the hour of the day, or the day of the week 
% when recommending restaurants could filter out restaurants that are currently closed or near closing time.  
Nowadays many companies are incorporating some kind of
context in their recommendation engines, application can be found in fields
such as e-commerce \cite{schafer1999recommender}, \cite{bulander2005enabling};
music \cite{ricci2012context},\cite{baltrunas2011incarmusic}; 
places of interest \cite{baltrunas2012context},
movies\cite{eyjolfsdottir2010moviegen}; vacation packages \cite{liu2011personalized},
\cite{liu2014cocktail}; travel guides \cite{savage2012m}; e-learning\cite{ortigosa2010entornos} and restaurants\cite{chu2013chinese}.\\  

The majority of these recommender systems are focused on 
recommending relevant items for users without
additional contextual information such as time, location or %location = place
companionship.  %!! Mas arriba dices que todos esos son 
% sistemas que incorporan al contexto??

However, recent systems incorporate contextual
information to make recommendations that deliver items adjusted
to the users' current context. For instance, the site Sourcetone % Referenca o url?
interactive radio \cite{huq2010automated} when selecting a song for
a customer it takes into consideration the listener's current mood in
order to provide a better recommendation. \\ Contextual information is used to
increase important characteristics such as user satisfaction, recommendation 
%aspects? no me gusta mucho esa palabra.  fetaures? characteristic 
quality, usefulness and more.\\
In this thesis an architecture and methodology are proposed for designing context-aware
recommender systems validated with a case study of a restaurant recommender system. The objetive is to
demonstrate how the level of user satisfaction is increased through
the use contextual information. The architecture is an hybrid of three techniques:

\begin{enumerate}  
\item \textit{Fuzzy Inference System} This a rule based recommender defined by an expert
in the domain, it considers as linguistic variables for instance for
the restaurant domain: \textit{ratings average:(low,medium,high)} ,
\textit{price of restaurant:(cheap,average,expensive)} and 
\textit{number of ratings of item:(few,several,many)} to infer how
relevant a restaurant is for the user. Generally, this recommendation
is based on the popularity of each item in the user community.
\item \textit{Content-based technique} utilizes the item profiles 
to compare how \textit{similar} is an item is with respect to another, i.e.  
restaurants that are \textit{similar} (same cuisine, ambient, price range)  to others 
that the user has rated favorably. The idea is to find items 
with similar features. 
\item \textit{Collaborative filtering
technique} is based on the user profile to identify user\'s
preferences and to find neighbors that have the same taste. The
recommendation consist in the suggestions of other users with similar
tastes that rated restaurants again in a similar way but
where have not been rated by the current user.
A Top-N list of restaurants is obtained to recommend for the user. 
\end{enumerate} 

The results of the three techniques are the list of
recommendations for the user and then in the next module are adjusted 
for user in context.
This is the last step and is represented as \textit{context filter} in
the method, then the recommender system obtains a list of
contextualized recommendations. The proposed methodology works
simultaneously to obtain recommendations, the hybrid method allows the
recommender system to generate suggestions despite having scarse
user information. When the system faces the the cold-start
problem or the over-specialization problem, the hybrid method
responses properly to show results for the user. These problems 
are described in a later sction. The expert\'s 
recommendation contributes to this goal, the rules specified in 
the Fuzzy Inference System help to infer the most suitable 
restaurant considering the only the input variables and the knowledge 
represented by the rules.\\
The context-based filter then makes recommendations based on those
items that are appropiate to the context of the user to produce
the final recommendation list. The context-aware recommender
system tries to increase the level of user satisfaction. To meet this
goal two metrics were utilized to test the system in an on-line test.

% Hablar del uso de pruebas de satisfaccion y usabilidad,
% se debe justificar en base a esto el por que son
% aparentemente pocos usuarios.

% Menciona también las pruebas de error de recomendacion
% Fijate como cambié lo de arquitectura por metodología
% ya que si te das cuenta este proceso se puede utilizar de manera
% generica para otros sistemas de recomendación
% La implementación especifica es la de recommet
% pero lo mismo puede usarse para Protoboard.
% Al principio debemos hablar de manera general.

Ten users were selected to interact with the system, subsequently an
analysis about the system performance and main issues highlited by the
users was realized. \\

The rest of this thesis is organized as follows: chapter 2 explains
the state of the art in context aware recommender systems,  
chapter 3 contains the background of the related topics, 
chapter 4  describes the proposed context-aware recommender methodology
describing each one of its components, then in chapter 5 the experimental set up
and the results are presented, chapter 6 describes
the evaluation of a case study a restaurant domain and finally, chapter 7
presents conclusions and a proposal for future work .




