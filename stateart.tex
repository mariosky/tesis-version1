\chapter{State of the art} \label{stateoftheart}

%Propongo una Sección de RS tradicionales y
%otra de Context Aware


In this section the state of the art in 
context aware recommender systems is presented. 
As a technology recommender systems have beem applied 
in many domains, and sometimes they represent the  key
technology for the success of web and mobile applications.

%Debemos presentar el SoA con cierto orden, ya sea como una linea
%del tiempo, empezando de los primeros trabajos a los más recientes
%o organizarlo por dominios, tecnologías etc. Voy a 
%clasificar los diferentes trabajos y después los organizamos
% 


%Tipo de información: Social, Aplicación Social 
%Tags
Some works utilize social information to recommend such
as Manca et.al.\cite{manca2014mining} where the friend recommender
system is applied in the social bookmarking domain, its goal was to
infer the interest of users from content selecting the available
information of the user behavior and analyzing the resources and the
tags bookmarked for each user, therefore the recommendations are
through mining user behavior in a tagging system, analyzing the
bookmarks tagged of the user and the frecuency for each used tag. 
%Keywords
J.Yao et.al.\cite{yao2012product} proposes a new product recommendation
approach for new users based on the implicit relationships between
search keywords and products. The relationships between keywords and
products are represented in a graph and relevance of keywords to
products is derived from attributes of the graph.
%Semantic
The relevance
information is utilized to predict preferences of new users. J.
Golbeck et.al.\cite{golbeck2006filmtrust} presents FilmTrust, a
website that integrates Semantic Web-based social networks, augmented
with trust, to create prediction movie recommendations. Trust takes on
the role of a recommender system forming the core of an algorithm to
predict a rating for recommendations of movies. This is an example of
how the Semantic Web, and Semantic trust networks in particular, can
be exploited to refine the user experience. \\  
%Eso de pros y cons me lo han criticado, me pedían algo más específico
%Este párrafo está bien por que se organiza por problema que atacan las
%propuestas
Traditional recommender techniques has its pros and cons, for
instance, the ability to handle data sparsity and cold-start problems
or considerable ramp-up efforts for knowledge acquisition and
engineering. Establish hybrid systems that combine the strengths of
algorithms and models to overcome some of the shortcomings and
problems has become the properly manner to improve the difficults for
each algorithm.
%Turistas
An example is presented by L.Castro et.al.\cite{castro2012prototype} 
a hybrid recommender system for the province of San Juan, Argentina, 
to recommend tourist packages  based on preferences and interest 
of each user, artificial intelligence
techniques are used to filter and customize the information. The
prototype of recommender system utilizes three techniques to
recommend: demographic, collaborative and content-based. The goal is
to recommend tourist packages that matches with the user profile.
%Restaurantes
L. Martinez et al.\cite{martinez2009reja} presents REJA, a hybrid
recommender system that involves collaborative filtering and
knowledge-based model, that is able to provide recommendations in some
situation for user; besides it provides georeferenced information
about the recommended restaurants.
%De los primeros
Balabanovic et.al.\cite{balabanovic1997fab} presents
Fab, a hybrid recommender system for automatic recognition of
emergent issues relevant to various groups of users. It also enables
two scaling problems, pertaining to the rising number of users and
documents, to be addressed. Claypool et.al.\cite{claypool1999combining} 
presents P-Tango system that utilizes content-based and collaborative
filtering techniques, it makes a prediction through the weighted
average that includes content-based prediction and collaborative
filtering prediction. The weights of predictions are determined on a
per-user basis, allowing the system to determine the optimum mixture
of content-based and collaborative recommendation for each user.
Pazzani M.\cite{pazzani1999framework} presents Entree as a hybrid
recommender system that it does not use numeric scores, but rather
treats the output of each recommender (collaborative, content-based
and demographic) as a set of votes, which are then combined in a
consensus scheme. The recommender system includes information such as
the content of the page, ratings of users and demographic data about
users. 
%Hibridos
Others works with hybrid recommender systems are ProfBuilder
\cite{al1999semantic}, PickAFlick\cite{burke1999integrating}  and
\cite{tran2000hybrid}, where are presented multiple recommendation
techniques. Usually, recommendation requires ranking of
items or selection of a single best recommendation, at this point some
technique must be employed to recommend. \\ 
Traditional recommender systems such as above mentioned, tend to use
simple user models. For example, user-based collaborative filtering
generally models the user as a vector of item ratings. As additional
observations are made about users’ preferences, the user models are
extended, and the user preferences is used to generate
recommendations. This approach, therefore, ignores the notion of any
specific situation, the fact that users interact with the system
within a particular context and  that preferences of items might 
change in another context.  
Overall, the context is able to make the recommender system be 
powerful that is adaptable to the changing user's situation.\\
The context is defined in the domain of the application and the system
has a context model that provides the information for the recommender
system. For instance Ricci et.al. \cite{baltrunas2011incarmusic} uses
the context in music domain using a model-based paradigm, in this
context-aware recommender system the context was defined as a set of
independent contextual factors(independent in order to get a
mathematical model) such as \textit{driving style, road type,
landscape, sleepiness, traffic conditions, mood weather and natural
phenomena} to specifies the relevant context for the music
recommendation. In order to estimate the relevance of selected
contextual factor, the users were requested to evaluate music tracks
in different contextual situations for each genre. The prediction
takes in account this relevance to recommend music tracks prefered by
the user according the genre and the contexts mentioned.In
restaurant domain Chung-Hua et al.\cite{chu2013chinese} presents a
context-aware recommender system for mobiles using a post-filtering
paradigm, the architecture involves a model client-server that works
with a request of data in the client side for the server side.
Subsequently, taking in account the contextual factors to filter the
properly restaurants to recommend. The context-aware recommender
system uses such as contextual factors \textit{location and season}, 
also utilize the user preferences to personalize the recommendations
in the user context.
Baltrunas et.al.\cite{baltrunas2011context} presents ReRex for tourism, 
a context-aware recommender system based in a model-based paradigm, the system
recommends and provides explanations about the why the places of
interest(PoI) are recommended. The proposed model computes a
personalized context dependent rating estimation. Subsequently, in
order to generates the explanation of recommendation the system uses
the factor that in the predictive model has the lasgest positive
effect on the rating prediction for the point of interest. The set of
contextual factors considered in ReRex are \textit{distance},
temperature, weather, season, companion, time day, weekday,
crowdedness, familiarity, mood, budget, travel length, transport and
travel goal. The main issue in ReRex system is the low user
satisfaction because of the explanations not able to be understood,
however the users recognize that the explanation is a very important
component that it influence the system acceptance. Noguera et. al.
\cite{noguera2012mobile} presents a context-aware recommender system
for tourism based in REJA that utilizes the location through a 3D-GIS
system, the application uses progressive downloading and rendering of
3D maps over mobiles networks. It is also in charge of tracking the
user’s location and speed based on GPS and the requesting. The system
utilizes pre-filtering and post-filtering paradigm. Pre-filtering is
used to reduce the number of items considered for the recommendation
according to the user’s location, and  post-filtering is applied to
re-rank the previous top-N list according to the physical distance
from the user for each recommended restaurant. The disadvantage in this
system is the lack of user reviews, because the recommendations are
based only in the location point without consider the experience of
other diners concerning the recommended restaurant. 
Cena et al.\cite{cena2006integrating} presents a tourist guide for
context in intelligent content adaptation. UbiquiTO system is a
tourist guide that integrates different forms of context-related
adaptation: for media device type, for user characteristics and
preferences, for the physical context of the interaction. UbiquiTO uses
a rule-based modeling approach to adapt the content of the provided
recommendation, such as the amount, type of information and features
associated with each recommendation. 
Bulander et.al\cite{bulander2005comparison} presents the MoMa-system that
offers proactive recommendations using a post-filtering approach for
matching order specifications with offers. When creating an order, the
client application will automatically fill in the appropriate physical
context and profile parameters, for example, \textit{location} and \textit{weather},
then, for example, the facility should not be too far away from the
current location of the user and beer should not be
recommended if it is raining. On the other side, advertisers’
suppliers put offers into the MoMa-system. These offers are also
formulated according to the catalogue. When the system detects a pair
of context matching order and offer, the end user is notified, in the
preferred manner (for example, SMS, email). At this point, the user
must decide whether to contact the advertiser to accept the offer.
Finally, Schifanella et al.\cite{schifanella2008mobhinter} develops
Mob-Hinter, a \textit{context-dependent} distributed model,where a user device
can directly connect to other mobile devices that are in \textit{physical
proximity} through ad-hoc connections, hence relying on a very limited
portion of the users’ community and just on a subset of all available
data (pre-filtering). The relationships between users are modeled with
a similarity graph. MobHinter allows a mobile device to identify the
affinity network neighbors from random ad-hoc communications. The
collected information is then used to incrementally refine locally
calculated predictions, with no need of interacting with a remote
server or accessing the Internet. The Recommendations are computed
using the availables rating of the user neighbors.
Abowd et. al. presents Cyberguide project \cite{abowd1997cyberguide},
which encompassed several tour guide prototypes for different handheld
platforms. Cyberguide provided tour guide services to mobile users,
exploiting the contextual knowledge of the user’s current and past
locations in the recommendation process. The PECITAS system
\cite{tumas2009personalized} presented by Thumas offers location-aware
recommendations for personalized point-to-point paths. The paths are
illustrated by listing the various connections that the user must take
to reach the destination using public transportation and walking. An
interesting aspect of PECITAS is that, although an optimal shortest 
path facility is incorporated, users may be recommended alternative 
routes that pass through several attractions, given that
their specified constraints (e.g. latest arrival time) and travel-related 
preferences (maximum walking time, maximal number of transport
transfers, sightseeing preferences, etc) are satisfied. Yu and Chang
presents LARS \cite{yu2009personalized} which supports personalized
tour planning using a rule-based recommendation process. This system
packages ‘where to stay’ and ‘where to eat’ features together with
‘typical’ tourist recommendations for sightseeing and activities. For
instance, recommended restaurants (selected based on their location,
menu, prices, customer rating score, etc) are integral part of the
tour and the time spent for lunch/dinner is taken into account to
schedule visits to attractions or to plan other activities.
Savage et. al. presents  "I'm feeling LoCo" system \cite{savage2012m}
that proposes a ubiquitous location­ based recommendation algorithm
that focuses on user experience by considering user preferences, time,
location and similarity measures automatically, having Foursquare as a
dataset. We also focus on user experience and aim that user input is
minimal. The information  om the user's social network, form of
transportation and phone's sensors is inferred to provide
recommendation of places  om the dataset.
Reddy et.al\cite{reddy2006lifetrak} presents LifeTrack system that
incorporates sensor information into song selection. The songs are
represented in terms of tags that the user assigns in order to link
the songs to the appropriate contexts in which they should be played.
User feedback is incorporated to make a song more or less likely to
play in a given context. Context considered relevant to song selection
includes location, time of operation, velocity of the user, weather,
traffic and sound. User locations and velocity are determined by GPS.
Location information includes tags based on zip code and whether the
user is inside or outside (inferred by the presence or absence of a
GPS signal). The times of the day are divided out into configurable
parts of the day (morning, evening, etc). The velocity is abstracted
into one of four states: static, walking, running and driving. Use of
accelerometers are planned to enable indoor velocity information. If
the user is driving, an RSS feed on traffic information is used to
typify the state as calm, moderate or chaotic. If the user is not
driving, a microphone reading is used for the same purpose.
Additionally, an RSS feed provides a meteorological condition (frigid,
cold, temperate, warm or hot).\\ The table \ref{tab:stateoftheart}  
describes examples of contextual factors in different domains 
of application, specifies the contextual factors considered 
such as  part of the context, the methodology for each 
application and  kind of devices.

\begin{sidewaystable}[]
  \caption{Comparison of context-aware recommender systems.}
    \label{tab:stateoftheart}
  \bigskip
    \centering\small\setlength\tabcolsep{2pt}
        \hspace*{-1cm}\begin{tabular}{p{3.5cm} p{6cm} p{4cm} p{3cm} p{3cm} }%{l l l l l}
           \toprule
             \textbf{Application} &\textbf{Contextual Factor} &\textbf{Domain} &\textbf{Paradigm} &\textbf{Device}  \\ \hline

           \midrule
             \textbf{CoMoLE} & \textbf{Time, available time, place, device, level of knowledge, learning style.} & \textbf{E-learning} & \textbf{Pre-filtering} & \textbf{Mobiles, PC, laptop.}   \\ \hline 

             \textbf{Moma-System} & \textbf{Location, time.} & \textbf{E-commerce} & \textbf{Post-filtering} & \textbf{PC, laptop.}  \\ \hline

             \textbf{UbiquITO} & \textbf{Season, time, temperature.} & \textbf{Tourism} & \textbf{Post-filtering} & \textbf{Mobiles} \\ \hline

             \textbf{ReRex} & \textbf{Distance of the point of interest,  temperature, weather, season, weekend, companion, travel goal, transport.} & \textbf{Tourism} & \textbf{Model-based} & \textbf{Mobiles} \\ \hline

             \textbf{LifeTrack} & \textbf{Location, time, day of the week, traifc noise(level), temperature, weather.} & \textbf{Music} & \textbf{ Post-filtering} & \textbf{PC, Mobiles.} \\ \hline

             \textbf{CARS} & \textbf{Location and season.} & \textbf{Restaurants} & \textbf{Post-filtering} & \textbf{PC, laptop.} \\ \hline

             \textbf{InCarMusic} & \textbf{Driving style, road type, landscape, sleepiness, traffic conditions, mood weather and natural phenomena.} & \textbf{Music} & \textbf{Model-based} & \textbf{Mobiles} \\ \hline

            \textbf{REJA} & \textbf{Location.} & \textbf{Restaurants} & \textbf{Pre-filtering and Post-filtering} & \textbf{PC, laptop, mobiles.} \\ \hline

            \textbf{CiberGuide} & \textbf{Location, time, weather.} & \textbf{Tourism} & \textbf{Post-filtering} & \textbf{Mobiles} \\ \hline

            \textbf{PECITAS} & \textbf{Location, routes.} & \textbf{Transport} & \textbf{Post-filtering} & \textbf{Mobiles} \\ \hline

            \textbf{LARS} & \textbf{Tourists’ location and time.} & \textbf{Tourist packages} & \textbf{Post-filtering} & \textbf{Mobiles} \\ \hline

            \textbf{I'm feeling LoCo} & \textbf{Location, transportation.} & \textbf{Tourism} & \textbf{Model-based} & \textbf{Mobiles} \\ \hline

            \textbf{MOPSI} & \textbf{Location} & \textbf{Tourism and transport} & \textbf{Post-filtering} & \textbf{Mobiles} \\ \hline

           \bottomrule
        \end{tabular}\hspace*{-1cm}
\end{sidewaystable}








