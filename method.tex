\chapter{Proposed method}\label{method}

The proposed method is based  in a previous architecture presented in
 \cite{garcia2009hybrid} used in an \textit{e-learning plataform (Protoboard)}
where the context  involves characteristics referring the student
enviroment as well as their performance in previous learning activities.  \\ 
After that, another recommender system called \textit{Recomet}  
 \cite{ramirez2013restaurant} implements the same method using 
only traditional recommender systems, i.e., it does not use the context 
in recommendation process  to suggest items to users.   \\ 
Chapter  \ref{stateoftheart} describes some the applications of both pre-filtering and 
post-filtering paradigms, each one is used in the appropiate domain 
and taking in account the system goals (see Table  \ref{tab:stateoftheart}).  
The propose method follows a \textit{post-filtering} approach, 
to contextualize the recommendations list
(see Figure  \ref{fig:paradigms}) previously generated using a hybrid
of three recommendation techniques, working simultaneously.
Once a user defines the current context, the post-filtering approach
adjusts the  recommendations for user's context, this process is called
``contextualization". \\ 
Each recommendation tecnique uses the collaborative filtering rating-matrix 
to obtain  predictions of items, considering variables such as
popularity and similarity among users and items. In this section
each technique used and components of the method are explained in
detail.\\  Overall, the 
post-filtering paradigm tries to cover the  user's needs in a wide range of
situations and it is based on the user's profile information. The
chapter describes  each component of the  method, as well as its
functionality and thier interaction with  another components.\\ 
In order to explain the method, a case of study in a
restaurant   domain will be used. Then, an brief description of
recommendation  techniques are also explained: 
\begin{enumerate} 
\item \textbf{Expert's fuzzy inference system}, 
this is a rule based recommender, these rules are defined by 
an expert in the domain. The
fuzzy inference system has the following input variables:
\textit{ratings average}:  \textit{low,medium and high}, \textit{price
of restaurant}:  \textit{cheap, average and expensive}, and
\textit{number of ratings of item}:  \textit{few, several and many},
these variables represent fuzzy values and are used to infer how
relevant a restaurant is for the  user. This recommendation is based
on the popularity of each  item in the user community. For other
domains, fuzzy variables must be identified by a domian expert and
expressed as a fuzzy inference system.
\item \textbf{Content-based technique} 
utilizes the item profiles (descriptions) to compare how
\textit{similar} is an item with respect to  another, i.e. restaurants
that are \textit{similar} have equal atributes (same cuisine,  atmosphere, price range,
etc.) to others that have been rated high by the user.  The idea is to find items
with similar features to the items that the  user prefers or it has
high rating in the user profile. 
\item \textbf{Collaborative filtering technique} 
is based in the user's preferences  to find neighbors that
have similars tastes and preferences. Similarity among users is
calculated using Pearson Correlation, a threshold upon of 0.8 is used
to determine the size of the neighbour. Then, recommendation process
considers the opinions of other users that rated restaurants with a
similar rating, but only recommends when the item have not been rated
by the current user. A threshold of 3.5 (in rating scale) determines
the items that will be added in the Top-N list that will be shows for
the user. 
\end{enumerate}  
The last step and is the \textit{contextualization} and is
represented as a \textit{context filter} in the method, as result the
list of contextualized recommendations is obtained. \\ As each
technique works simultaneously to obtain recommendations, the hybrid
method allows to generate suggestions even without  ratings of user,
it means that using content-based technique or the fuzzy inference system, 
the system faces the cold-start and overspecialization
problems using these thechniques, these problems were previously
described in section  \ref{coldstart} and  
section  \ref{overspecialization}, respectively.\\ 
For the case of study, the proposed method was tested in the 
restaurants domain for the city of Tijuana, a context-aware 
recommender system  was developed  in order to verify its 
performance with real users interacting with  the system and to
test  its functionality. \\  The application was developed using
technologies as \textit{Django Framework}, \textit{Python language}, 
\textit{Bootstrap 3.0}, \textit{JavaScript}, \textit{JQuery} and  
\textit{PostgreSQL database}. The complete description of the 
system will be show in a posterior section.\\ 
To evaluate the performance of the proposed method in
the case of study several experiments were conducted, the algorithms
were tested using  contextual datasets and the number of contextual
factors used varied  according the information provided in the
datasets. \\ The goal of the  experiments were to observe the role of
contextual data  in the performance of the algorithms and to
identify what contextual factors are more relevant for users in a
specific situation, how recommendations are improved using context
and, how improves the accuracy in recommendations.

\section{Data Models}

The data model defines an abstraction of the context. An effective on-
line  recommender system must be based upon an understanding of
consumer's  preferences and successfully mapping potential products
into these preferences  \cite{adomavicius2011context}. \\ Pan and
Fesenmaier  \cite{pan2006online} argued that this can be achieved
through the understanding of  how consumers describe in their own
language a \textit{product}, \textit{a place}, and the overall
\textit{experience}  when they are consuming the product or visiting
the place.  It is observed that  the language of the consumers is
often imprecise,  with attributes values often expressed in fuzzy
terms. 

\subsection{Restaurant Data Model}

Following the case of study mentioned in previous section,   the
restaurant model will be explained.\\  Traditionally, choosing a
restaurant has been considered as rational behavior where a number of
attributes contribute to the overall usefulness of a restaurant. For
example: food type, service quality, atmosphere of the restaurant, and
availability of information about a restaurant, plays an important
role at different stages in consumer's desition
making  \cite{auty1992consumer}. \\ 
While food quality and food type have
been perceived as the most important variables for consumers'
restaurant selection, situational and contextual factors have been
found to be important also. Due to this
Kivela  \cite{jack1997restaurant} identifies four types of restaurants:
1) \textit{fine dining/gourmet}, 2) \textit{theme/atmosphere}, 3)
\textit{family/popular}, and 4) \textit{convenience/fast-food}; on the
other hand Auty  \cite{auty1992consumer} identifies four types of dining
out occasions: 1) \textit{celebration}, 2) \textit{social occasion},
3) \textit{convenience/quick meal}, and 4) \textit{business meal}.\\
Taking in account the context, the model proposed was definded with 55
attributes about restaurants. \\ An exploration about contents of models
of others works were compared to define the suitable information into
the model. Therefore, the restaurant model is a binary vector with the
following contextual attributes: 1) \textit{price range}, 2)
\textit{payment type}, 3) \textit{alcohol type}, 
4) \textit{smoking area}, 5) \textit{dress code}, 6) \textit{parking type}, 7)
\textit{installations type}, 8) \textit{atmosphere type}, and 9)
\textit{cuisine type}. \\ An example of a restaurant model is depicted in
Figure  \ref{fig:restaurantmodel} with some domain values of the
context represented by a binary vector where \textit{1} means that the
restaurant has the property that corresponds to the position value.
Additionally, the restaurant model contains other information such as
\textit{users's reviews}, \textit{ratings average}, and
\textit{geographycal location.}
\begin{figure*}
\captionsetup{justification=centering,margin=2cm,font=footnotesize}
\centering
\setlength\fboxsep{0pt}
\setlength\fboxrule{0.7pt}
\fbox{\includegraphics[width=10cm,height=9cm,keepaspectratio]
{img/restaurant-model.png}}
\caption{User interface of the restaurant model.}
\label{fig:restaurantmodel}       
\end{figure*}

\subsubsection{Contextual Factors Data Model} \label{datamodelsection}   

The data model was physically implemented in a PostgreSQL database model, 
then the information in the context-aware recommender system was
managed in a schema of a relational database. \\ Technical support 
about installations of dependencies and the open source application are 
described in appendix  \ref{appendixa}.
The restaurant data model is depicted in the 
Figure  \ref{fig:restaurantmodeldata}, the model contains the restaurant
entity and its attributes. \\ The restaurant entity is related to
\textit{Item entity} in a \textit{``one-to-one"} relation that at the
same time is related to the \textit{RecommenderRule entity} which
specifies some restrictions for item recommendations. A large view of
all the entities related is depicted in the whole scheme refered in
Figure  \ref{fig:datamodel}.
\begin{figure*}
\captionsetup{justification=centering,margin=2cm,font=footnotesize}
\centering
\includegraphics[width=8cm,height=8cm,keepaspectratio]{img/data-resmodel.png}
\caption{The restaurant data model.}
\label{fig:restaurantmodeldata}     
\end{figure*}
\\ The entities related correspond to the proposed contextual factors, 
are defined as follows: 
\begin{itemize}
\item \textbf{Price:} \textit{cheap, regular, expensive, too expensive.}
\item \textbf{Payment:} \textit{credit/debit card, cash.}
\item \textbf{Alcohol:} \textit{no alcohol, wine-beer.}
\item \textbf{Smoking area:} \textit{yes, no.}
\item \textbf{Dresscode:} \textit{casual, informal, formal.}
\item \textbf{Installations:} \textit{garden, terrace, indoor, outdoor.}
\item \textbf{Atmosphere:} \textit{relax, familiar, friends, bussines, romantic.}
\item \textbf{Parking:} \textit{no parking, free parking, valet parking.}
\item \textbf{Cuisine:} \textit{japanese, chinese, italian, argentinean,
cantonese, mandarin, mongolian, arabic, greek, spanish, brasilian,
swiss, szechuan, asian, international, steak grill,vegetarian,
natural/healthy/light, traditional mexican, tacos, mediterranean,
middle eastern, american/fast food, gourmet, pizza, bar/beer, tapas
cafe/bar, french, birds, seafood.}
\end{itemize}
The \textit{cuisine types} were defined according the food variety 
of restaurants in Tijuana, there are 30 types of cuisines defined 
in the system. \\
The \textit{smoking area} is an attribute with boolean value that 
defines if a restaurant has a smoking area into its installations.
These are the fuzzy contextual factors proposed in this case study.

\subsection{User Model} 

The user's profile is derived from the ratings matrix. Let
$U=[u_1,u_2,...u_n]$ the set of all users and $ I=[i_1,i_2,$...$i_n] $
the set of all items, if $R$ represent the ratings matrix,  an element
$R_{u,i}$ represents a user’s rating $u \in U$  in a item $i \in I$.
The unknown ratings are denoted as $\neq $. \\ The matrix $R$ can be
decomposed into rows vectors, the row vector is denoted as $
\overrightarrow{r_u} $=$[R_{u,1}$...$R_{u,|I|}]$ for every $u \in U$.
\\ Therefore, each row vector represents the ratings of a particular user
over the items. Also denote a set of items rated by a certain user u
is denoted as $ I_u = i \in I | \forall  i: R_{u,i} \neq \emptyset $.
This set of items rated represents the user preferences where for each
domain element $R_{u,i} \in [1-5]$ represents the intensity of the
user interest for  the item.\\
\begin{figure*}
\captionsetup{justification=centering,margin=2cm,font=footnotesize}
\centering
\setlength\fboxsep{0pt}
\fbox{\includegraphics[width=10cm,height=9cm,keepaspectratio]
{img/user-profile.png}}
\caption{Example of user interface for user profile.}
\label{fig:user-profile}      
\end{figure*}

\subsubsection{User Data Model Implementation} 

<<<<<<< HEAD
The user data model in postgreSQL is represented in the 
Figure  \ref{fig:datausermodel}, the model involves the entities:
\textit{User, UserProfile, and Friends.} \textit{UserProfile entity}
provides the contextual information of user, \textit{User entity} is
the default model defined in the system and is related to userProfile
for suplies valuable information. \\
In the application, user profile has contextual
information such as: 1) price range, 2) current location, 3) cuisine
types, 4) attributes or characteristics of restaurants that the user want, 
<<<<<<< HEAD
5) list of  favorite restaurants.  \\ 
The user profile is stored in database and it is available for 
queries upon request, and it can be changed by users many times 
in a session. \\ The information used to make recommendations 
is the most recent. The user interface is represented in 
Figure  \ref{fig:user-profile}.
5) list of  favorite restaurants.  
The user profile is stored in database and it is available for 
queries upon request, and it can be changed by users many times 
in a session. The information used to make recommendations 
is the most recent. The user interface is represented in 
figure \ref{fig:user-profile}.
>>>>>>> origin/master
\begin{figure*}
%\captionsetup{justification=centering,margin=2cm}
\captionsetup{font=footnotesize}
\centering
\includegraphics[width=8cm,height=8cm,keepaspectratio]
{img/data-usermodel.png}
\caption{The user data model.}
\label{fig:datausermodel}     
\end{figure*}
The \textit{Friends entity}
represents the social aspect in the \textit{userProfile}, \textit{Friends} involves
the users related to the current user taking in account the
preferences of each other. \\The user profile entity is related with: 
price and cuisine are the same that in restaurant model,
<<<<<<< HEAD
attribute groups corresponds to restaurant model mentioned 
(section  \ref{datamodelsection}). A total of \textit{55 attributes}
(or characteristics)  could be contained in \textit{userProfile}. 
The domain values for price,  cuisine and attribute groups are following:
\begin{itemize}
\item \textbf{Price:} cheap, regular, expensive, too expensive.
\item \textbf{Cuisine:} japanese, chinese, italian, argentinean,
cantonese, mandarin, mongolian, arabic, greek, spanish, brasilian,
swiss, szechuan, asian, international, steak grill,vegetarian,
natural/healthy/light, traditional mexican, tacos, mediterranean,
middle eastern, american/fast food, gourmet, pizza, bar/beer, tapas
cafe/bar, french, birds, seafood.
\item  \textbf{Attribute groups:} Installations, atmosphere, 
parking, payment, smoking area, dresscode, alcohol.
\end{itemize}

\subsection{Relational data model} 

A complete database relational scheme is represented in the 
Figure  \ref{fig:datamodel}. This model involves the whole relational 
database for context-aware recommender system, as well as 
the entities and relations among them. \\
Contextual information is also stored in the following entities:
\textit{CurrentLocation, DistancePoi and Ratings.} \\ 
In the case of \textit{Reviews entity}  only describes the user's 
opinion about visited restaurants, this information contributes to 
have additional information about recent preferences of diners, 
however, this process is not implemented in the system still. \\
\textit{CurrentLocation entity} stores the geographical position of
user to get a \textit{"nearby recommendation"}, the system locates
restaurants around two kilometers from the user position, for instance
the system can takes a user position as \textit{latitude:32.529865}
and \textit{longitude: -116.986605}, this information is stored to
calculate distances from this point. \\ The geographical position is
changed frequently, in this manner, it allows the adaptation for each
<<<<<<< HEAD
particular user situation. \\ \textit{Distance PoI entity} stores the
distances (kilometers) between the user and restaurants, this
information is used to calculate a \textit{"nearby recommendation"},
where each recommended restaurant must not be over the threshold defined, 
then in this case, the system considers only nearby items, into two kilometers 
around the user position.\\
Finally, \textit{Rating entity} represents the user's preferences  
in a vector of ratings that could be increased in time and  
the user's preferences patterns thatcould be changed in time also.
\begin{landscape} 
\begin{figure}[!h] 
\captionsetup{font=footnotesize}
\centering
\includegraphics[width=1.3\textwidth]{img/recomet.png}
\caption{The data model of context-aware recommender system.}
\label{fig:datamodel}    
\end{figure}
\end{landscape}

\section{Expert recomendation} 

Fuzzy logic is a methodology that provides a simple way to obtain
conclusions of linguistic data. Is based on the traditional process of
how a person makes decisions based in linguistic information. \\ Fuzzy
logic is a computational intelligence technique that allows to use
information with a high degree of inaccuracy; this is the difference
with the conventional logic that only uses concrete and accurately
information  \cite{zedeh1989knowledge}.\\  In this work, fuzzy logic is
used to model fuzzy variables that highligh in the popularity of a
restaurant.\\   
The Expert module (fuzzy inference system) generates 
recommendations  when the other
recommendation techniques (collaborative filtering, content-based) are
not returning results because of the cold start problem.\\   The fuzzy
inference system proposed has three \textbf{input variables} (as 
in the previous work  \cite{garcia2009hybrid}): 1) \textit{rating} is
an average of ratings that has a particular restaurant inside the user
community; the domain of variable is from 0 to 5 and contains two membership
functions labeled as \textit{low} and \textit{high}(Figure  \ref{fig:mf:a}), 
2) \textit{price} represents the kind of price that
has a particular restaurant; the domain of variable is from 0 to 5 and
contains two membership functions labeled as \textit{low} and
\textit{high} (Figure  \ref{fig:mf:b}), and 3) \textit{votes} is used to
measure how many items have been rated by the current user, i.e., the
participation of the user, if the user has rated few items (less than
10) is not sufficient to make accurate predictions( Figure  \ref{fig:mf:c}), 
the domain of variable is from 0 to 10 and contains two
membership functions labeled as \textit{insufficient} and
\textit{sufficient}. \\ The \textbf{output variable} is
\textit{recommendation}, represents a weight for each restaurant
proposed by the expert considering the inputs mentioned above, the
domain of variable is from 0 to 5 and contains three membership functions
labeled as \textit{low}, \textit{medium} and \textit{high} (Figure  \ref{fig:mf:d}).
\begin{figure}[ht!]
   \captionsetup{font=footnotesize}
   \centering
   %%----primera subfigura----
   \subfloat[]{
        \label{fig:mf:a}
        \includegraphics[width=0.42\textwidth]{img/mf-rating.png}}
   \hspace{0.1\linewidth}
   %%----segunda subfigura----
   \subfloat[]{
        \label{fig:mf:b} 
        \includegraphics[width=0.42\textwidth]{img/mf-price.png}}\\[20pt]
   %%----tercera subfigura----
   \subfloat[]{
        \label{fig:mf:c} 
        \includegraphics[width=0.42\textwidth]{img/mf-votes.png}}
    \hspace{0.1\linewidth}
   %%----cuarta subfigura----
    \subfloat[]{
        \label{fig:mf:d} 
        \includegraphics[width=0.42\textwidth]{img/mf-recommendation.png}}
   \caption{The Gaussian membership functions of the expert system.
   }
   \label{fig:mfexpert} 
\end{figure}
The proposed fuzzy inference system (Figure  \ref{fig:expertfis})
represents the users' experience and their knowledge about restaurants.
This factors are considered important for users that visiting a
restaurant. This information is recovered of user profile and
restaurant profile; and the system uses this information to get
weights that influence in the final recommendations. The fuzzy
inference system uses five inference rules that involve the variables of
inputs and output. The input variables determine the recommendation
activation; each input variable contains labels as \textit{low} and
\textit{high} that also correspond to memberships functions of
Gaussian type. For the output variable \textit{recommendation} the
labels \textit{low}, \textit{medium}, and \textit{high} are used with
membership functions Gaussian type also. The rules are:
\begin{enumerate} 
\item \textit{If \textbf{rating} is high and \textbf{price} is low then 
\textbf{recommendation} is high.}
\item \textit{If \textbf{rating} is high and \textbf{votes} is sufficient then 
\textbf{recommendation} is high.}
\item \textit{If \textbf{rating} is high and \textbf{votes} is insufficient then 
\textbf{recommendation} is medium.}
\item \textit{If \textbf{rating} is low and \textbf{price} is high and then 
\textbf{recommendation} is low.} 
\item \textit{If \textbf{rating} is low and \textbf{votes} is insufficient then 
\textbf{recommendation} is low.}
\end{enumerate} 
\begin{figure*}
\captionsetup{justification=centering,margin=2cm,font=footnotesize}
\centering
\fbox{\includegraphics[width=0.70\textwidth]{img/expert.png}}
\caption{Fuzzy inference system of the expert.}
\label{fig:expertfis}      
\end{figure*}

\section{Fuzzy Weighted Average} 

The main goal of this fuzzy inference system is to assign weights for
each recommendation list.\\ The recommendation technique is based in the
amount of available information stored, because each technique uses
this information to provide a list of restaurants as well as a weight
for each one, therefore, these are used for  recommendations if its
weight is upper the threshold.  \\The fuzzy inference system has inputs
and outputs to infer each list's weight, its variables are depicted in
Figure  \ref{fig:fis-pesos}.  There are three membership functions for
inputs and three for outputs. \\The input variables are:
\textit{userSimilarity, restaurantSimilarity and Participation} and
are depicted in Figure  \ref{fig:fis-inputs}. The  Figure  \ref{fig:fis-inputs}.a
and  Figure  \ref{fig:fis-inputs}.b  are in a range from 0 to 1 to
define the similarity average among users and restaurants,
respectively. The Figure  \ref{fig:fis-inputs}.c has a range from 0
to 15  to represent the ratings of the user (participation). This
information is stored in the Popularity entity (see 
Figure  \ref{fig:datamodel}).
\begin{figure*}
\captionsetup{justification=centering,margin=2cm,font=footnotesize}
\centering
\fbox{\includegraphics[width=0.70\textwidth]{img/fis-pesos.png}}
\caption{Fuzzy Inference System to assign weights.}
\label{fig:fis-pesos}       
\end{figure*}
On the other side, the output variables are: 
\textit{Expert}, \textit{RestaurantProfile} and \textit{Correlation}, 
these are depicted in 
Figure  \ref{fig:fis-outputs}. \\The Figure  \ref{fig:fis-outputs}.a  represents
the weight for expert recommendation list, Figure  \ref{fig:fis-outputs}.b 
represents the weight of the content-based list and 
Figure  \ref{fig:fis-outputs}.c  represents the weight of collaborative
recommendation list, their membership functions are in a range from 0
to 1 to get the value.
\begin{figure}[ht!]
   \captionsetup{font=footnotesize}
   \centering
   %----primera subfigura----
   \subfloat[]{
        \label{fig:mf:a}
        \fbox{\includegraphics[width=0.42\textwidth]{img/usersimilarity.png}}}
   \hspace{0.1\linewidth}
   %%----segunda subfigura----
   \subfloat[]{
        \label{fig:mf:b} 
        \fbox{\includegraphics[width=0.42\textwidth]{img/restaurantsimilarity.png}}}\\% 
   %%----tercera subfigura----
   \subfloat[]{
        \label{fig:mf:c} 
        \fbox{\includegraphics[width=0.42\textwidth]{img/participation.png}}}
        %\hspace{0.1\linewidth}
   \caption{The Gaussian membership functions of input variables.
   }
   \label{fig:fis-inputs} 
\end{figure}
\begin{figure}[ht!]
   \captionsetup{font=footnotesize}
   \centering
   %%----primera subfigura----
    \subfloat[]{
        \label{fig:mf:d} 
        \fbox{\includegraphics[width=0.42\textwidth]{img/expert-mf.png}}}%\\ %[20pt]
        \hspace{0.1\linewidth}
      %%----segunda subfigura----
    \subfloat[]{
        \label{fig:mf:d} 
        \fbox{\includegraphics[width=0.42\textwidth]{img/restaurantprofile-mf.png}}}\\
     %%----tercera subfigura----
    \subfloat[]{
        \label{fig:mf:d} 
        \fbox{\includegraphics[width=0.42\textwidth]{img/correlation-mf.png}}}%\\ %[20pt]
   \caption{The Gaussian membership functions of output variables.
   }
   \label{fig:fis-outputs} 
\end{figure}
Taking in to account the input variables, the rules utilized to infer the 
output values are following:
\begin{enumerate} 
\item \textit{If \textbf{userSimilarity} is low and 
\textbf{restaurantSimilarity} is low and \textbf{participation} 
is insufficient then \textbf{expert} is high, \textbf{restaurantProfile} 
is low, \textbf{correlation} is low.}
\item \textit{If \textbf{userSimilarity} is low and 
\textbf{restaurantSimilarity} is low and \textbf{participation} 
is sufficient then \textbf{expert} is low, \textbf{restaurantProfile} 
is low, \textbf{correlation} is high.}
\item \textit{If \textbf{userSimilarity} is low and 
\textbf{restaurantSimilarity} is low and \textbf{participation} 
is minimun then \textbf{expert} is low, \textbf{restaurantProfile} 
is low, \textbf{correlation} is high.}
\item \textit{If \textbf{userSimilarity} is low and 
\textbf{restaurantSimilarity} is high and \textbf{participation} 
is insufficient then \textbf{expert} is low, \textbf{restaurantProfile} 
is high, \textbf{correlation} is low.}
\item \textit{If \textbf{userSimilarity} is low and 
\textbf{restaurantSimilarity} is high and \textbf{participation} 
is minimun then \textbf{expert} is low, \textbf{restaurantProfile} i
s high, \textbf{correlation} is low.}
\item \textit{If \textbf{userSimilarity} is low and 
\textbf{restaurantSimilarity} is high and \textbf{participation} 
is sufficient then \textbf{expert} is low, \textbf{restaurantProfile} 
is high, \textbf{correlation} is low.}
\item \textit{If \textbf{userSimilarity} is high and 
\textbf{restaurantSimilarity} is low and \textbf{participation} 
is insufficient then \textbf{expert} is low, \textbf{restaurantProfile} 
is low, \textbf{correlation} is high.}
\item \textit{If \textbf{userSimilarity} is high and 
\textbf{restaurantSimilarity} is low and \textbf{participation} 
is minimun then \textbf{expert} is low, \textbf{restaurantProfile} 
is low, \textbf{correlation} is high.}
\item \textit{If \textbf{userSimilarity} is high and 
\textbf{restaurantSimilarity} is low and \textbf{participation} 
is sufficient then \textbf{expert} is low, \textbf{restaurantProfile} 
is low, \textbf{correlation} is high.}
\item \textit{If \textbf{userSimilarity} is high and 
\textbf{restaurantSimilarity} is high and \textbf{participation} 
is insufficient then \textbf{expert} is low, \textbf{restaurantProfile} 
is low, \textbf{correlation} is high.}
\item \textit{If \textbf{userSimilarity} is high and 
\textbf{restaurantSimilarity} is high and \textbf{participation} 
is sufficient then \textbf{expert} is low, \textbf{restaurantProfile} 
is low, \textbf{correlation} is high.}
\item \textit{If \textbf{userSimilarity} is high and 
\textbf{restaurantSimilarity} is high and \textbf{participation} 
is minimun then \textbf{expert} is low, \textbf{restaurantProfile} 
is low, \textbf{correlation} is high.}
\end{enumerate} 
At the end, a \textit{weighted average} allows to get predictions for
each restaurant in the list of recommendations. In this way, for
instance if the \textbf{expert} has a weight of \textbf{0.5}, the
\textbf{restaurantProfile} is \textbf{0.8} and the
\textbf{correlation} is \textbf{0.6}, the system uses these weights to
calculate the final prediction for a particular restaurant using the
formula of the weighted average (Equation  \ref{eq:wa}):
\begin{equation}\label{eq:wa}
\displaystyle \bar{x} = {\sum_{i=1}^{n} x_{i} w_{i}
\over \sum_{i=1}^{n} w_{i} } = 
{x_{1}w_{1} + x_{2}w_{2} + x_{3}w_{3} + ... + x_{n}w_{n} 
\over w_{1} + w_{2} + w_{3} + .... + w_{n}}
\end{equation}
Where the set \textit{X=\{ $x_{1}, x_{2}, x_{3},...,x_{n}$ \}} represents the 
predictions and the set \textit{W = \{ $w_{1}, w_{2}, w_{3},...,w_{n}$ \} } 
represents the weights 
given for the fuzzy inference system for each recommendation technique. 
Then, applying the formula, the system can process it in the follow manner  
(Equation  \ref{eq:prediction}):
\begin{equation}\label{eq:prediction}
\displaystyle \bar{x}  = {(0.5 * 4.0) + ( 0.8 * 5) + (0.6 * 4.5)
\over (0.5 + 0.8 + 0.6)}
\end{equation}
The prediction corresponds the final value of recommendation for the
item, and is used to include it or exclude it of the list of
recommendation if is not over the threshold. \\ Then, for this case, the
prediction is \textbf{4.57}, it means that this restaurant will be in the
recommendation list of the user, subsequently, the recommendations
list will be contextualized.

\section{Contextual recommendation} 

The interface of the system (Figure  \ref{fig:context}) allows the collection 
of contextual information, such as type of price, restaurant's
attributes, type of cuisine, and geographical location. 
The context-aware recommender system uses post-filtering paradigm, 
then the contextual information is used as a filter for the final 
recommendations list. For example, geographical location is used 
to get restaurants around two kilometers of distance, next, 
the list of nearby restaurants is displayed for the user.\\  
\begin{figure*}
\captionsetup{justification=centering,margin=2cm,font=footnotesize}
\centering
\fbox{\includegraphics[width=0.70\textwidth]{img/context.png}}
\caption{System interface to collect contextual information.}
\label{fig:context}   
\end{figure*}
If the context-aware recommender system
considers other attributes, as type of price or the type of cuisine
preferred by the user, the system gets restaurants matched in the
context especified by the user at this time. 
\begin{figure*}
\captionsetup{font=footnotesize}
\centering
\fbox{\includegraphics[width=0.70\textwidth]{img/recom.png}}
\caption{System interface of recommendations for the user.}
\label{fig:recom}    
\end{figure*}
In the attributes box, the user can choose any preference about what
things are importants to select a restaurant. The features are
collected from the dataset of Tijuana restaurants. \\In the cuisine box,
the user choosen his/her favorite cuisine, it can be one or more
cuisines such as in attributes  also.
The context changes constantly, indeed, the users migh change 
it many times such as them wish.\\ 
After the post-filtering, the system displays the  recommended
restaurants according the information provides by the user. The
context-aware recommender system contains four techniques to display
recommendations. The interface in Figure  \ref{fig:recom} shows
recommendations: \textit{1) Expert, 2) Content-based, 3) Collaborative
filtering and 4) Nearby.} Each one was explained above, except the
nearby recommendations. \\ For nearby recommendations the system
calculates the approximate distance between the current geographical
location of the user and the available restaurants in the area.  The
threshold is two kilometers around the user position to determine what
restaurants will be recommended. The geographical position is
obtained through the Google Maps API.

\section{Method scheme} 

The general scheme of the proposed method is depicted in the 
Figure  \ref{fig:archit}. In the first part, the three techniques of
recommendations are suplied by the rating matrix: \\ 
1) \textbf{Fuzzy inference system of the expert}  obtains the input 
values  through rating matrix taking users ratings, so it can infer 
the output value.
2) \textbf{Content-based} uses the high ratings in the user profile 
and item profiles to calculate similarity between them and do that using Cosine
similarity measure.
\begin{figure*}
\captionsetup{font=footnotesize}
\centering 
\includegraphics[width=10cm,height=10cm,keepaspectratio]{img/archit.png}
\caption{Scheme of the proposed method.}
\label{fig:archit}  
\end{figure*}
3) \textbf{Collaborative filtering} is based in user profiles content in
ratings matrix, it calculates the similarity among  users using
Pearson correlation, subsequently,  a list of neighbors is obtained to
use their preferences to calculate predictions.\\
The second part shows three lists obtained, the fuzzy inference system
is used to assign weights for each one in order to determine the
prediction, these weights are important to calculate a
prediction because is obtained through the weighted average.\\
Later, the recommendation lists are reduced when the context filter is
applied, i.e., the contextualization process is applied to get
recommendations adjusted in the user context. 
Finally, the contextual recommendations list is displayed in the 
user interface (see Figure \ref{fig:recom}).


