%% Los cap'itulos inician con \chapter{T'itulo}, estos aparecen numerados y
%% se incluyen en el 'indice general.
%%
%% Recuerda que aqu'i ya puedes escribir acentos como: 'a, 'e, 'i, etc.
%% La letra n con tilde es: 'n.


% 
%MG: En el resumen debes hablar de todo el trabajo revisa: 
%% http://www.slideshare.net/jjmerelo/cmo-escribir-y-publicar-trabajos-cientficos-17948694

% Da una breve introducción al tema (problema)
% Los métodos que pleanteamos aquí son {metodología}
% para [mejorar, evitar, comprobar] {problema}  
% Se probó de la manera {x} que los objetivos se alcanzaron  
% con los resultados siguientes {x} 


\prefacesection{Resumen}

Los sistemas de recomendaci\'on tradicionales se han visto mejorados
notablemente en los ultimos a\~{n}os, el contexto ha sido utilizado para
implementar recomendaciones personalizadas y contextualizadas en este
tipo de sistemas con el fin de incrementar el grado de satisfacci\'on de
los usuarios con respecto de los productos recomendados. Actualmente
se conocen como sistemas de recomendaci\'on sensibles al contexto y
trabajan basados en los paradigmas de pre-filtrado y post-filtrado
principalmente. El contexto se define en cada aplicaci\'on debido a que
los factores contextuales son innumerables y dependen directamente del
dominio de aplicaci\'on y del objetivo del sistema de recomendaci\'on
implementado.  
%%%% Empieza aquí:
Los sistemas de recomendacion sensibles al contexto
usan metodos h\'ibridos para mejorar el desempe\~{n}o  a su vez, pueden
incluir sistemas difusos que mejoran el tratamiento y procesamiento
del lenguaje humano-m\'aquina en el sistema. En esta tesis se define una
arquitectura para un sistema de recomendaci\'on sensible al contexto,
que utiliza 
%%% Resume un poco esto:
t\'ecnicas de recomendaci\'on colaborativa y basada en
contenido,  Sistema de Inferencia Difuso y atributos difusos que se
integran en el proceso de recomendaci\'on contextual. 
%% El sistema se probó  {metodología}

%% Al integrar estas técnicas el desempeño mejoró en {x} respecto a {y}
%% de manera estadísticamente significativa.


%\chapter{Resumen}
\prefacesection{Abstract}

Traditional recommender systems has been improved in recent years,
context has been used to implement personalized and contextualized
recommendations for this kind of systems in order to increase the user
satisfaction with respect of the recommended items. These systems are
called context-aware recommender systems and are based in pre-filtering and post-filtering mainly. The context is defined  into the
application due to the large amount of contextual factors, these
depend of the domain and the system goals. Context-aware recommender
systems use hybrid methods and fuzzy techniques to improve the
processing of language human-machine. This thesis defines  an
architecture  for a context-aware recommender system that utilizes
recommendation techniques (collaborative and content-based), Fuzzy
Inference Systems and fuzzy attributes that are involved in the
contextual recommendation process.










