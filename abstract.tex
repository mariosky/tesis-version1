%% Los cap'itulos inician con \chapter{T'itulo}, estos aparecen numerados y
%% se incluyen en el 'indice general.
%%
%% Recuerda que aqu'i ya puedes escribir acentos como: 'a, 'e, 'i, etc.
%% La letra n con tilde es: 'n.

% 
%MG: En el resumen debes hablar de todo el trabajo revisa: 
%% http://www.slideshare.net/jjmerelo/cmo-escribir-y-publicar-trabajos-cientficos-17948694

% Da una breve introducción al tema (problema)
% Los métodos que pleanteamos aquí son {metodología}
% para [mejorar, evitar, comprobar] {problema}  
% Se probó de la manera {x} que los objetivos se alcanzaron  
% con los resultados siguientes {x} 

% Al integrar estas técnicas el desempeño mejoró en {x} respecto a {y}
% de manera estadísticamente significativa logrando para los datos de prueba 
% un error absoluto de XX.
% Es necesario comentar los resultados con el error?.. 

\prefacesection{Resumen}

Un sistema de recomendaci\'on sensible al contexto analiza  las
necesidades y preferencias de los  usuarios con el prop\'osito de
recomendar items utilizando la informaci\'on contextual que describe
la situaci\'on actual del usuario. En esta tesis se propone un
m\'etodo para sistemas de recomendaci\'on sensibles al contexto el cual 
puede ser aplicado en diferentes dominios, para mejorar las
recomendaciones utilizando informaci\'on contextual en un m\'etodo
de recomendaci\'on h\'ibrido. El m\'etodo involucra diferentes
t\'ecnicas que trabajan simult\'aneamente para obtener las
recomendaciones: filtrado colaborativo y basado en contenido adem\'as
de un sistema de inferencia para procesar tanto reglas como atributos difusos.
Posteriormente, las recomendaciones son filtradas por los factores de
contexto que representan la situaci\'on actual del usuario.
En esta tesis se analiza el trabajo relacionado y se destacan las
principales contribuciones del m\'etodo, presentando los resultados de un
extenso conjunto de experimentos que validan el m\'etodo propuesto.

\prefacesection{Abstract}

A context-aware recommender system analyzes the needs and
preferences of users in order to recommend items using contextual
information, that describes the current situation of the user. In this
thesis a method for context-aware recommender system is proposed. This
method can be applied in different domains to improve 
recommendations using contextual information as an hybrid recommender system.
The proposed method involves several techniques that work
simultaneously: collaborative and content-based filtering,
and a Fuzzy inference system to process rules and fuzzy
attributes. Subsequently, recommendations are filtered by using contextual
factors that represent the current situation of the user. An analysis 
of the related work is presented, and the main
contributions of the method highlighted showing ther esults of an 
extensive set of experiments, that validate the proposed method.











