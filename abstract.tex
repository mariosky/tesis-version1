%% Los cap'itulos inician con \chapter{T'itulo}, estos aparecen numerados y
%% se incluyen en el 'indice general.
%%
%% Recuerda que aqu'i ya puedes escribir acentos como: 'a, 'e, 'i, etc.
%% La letra n con tilde es: 'n.

% 
%MG: En el resumen debes hablar de todo el trabajo revisa: 
%% http://www.slideshare.net/jjmerelo/cmo-escribir-y-publicar-trabajos-cientficos-17948694

% Da una breve introducción al tema (problema)
% Los métodos que pleanteamos aquí son {metodología}
% para [mejorar, evitar, comprobar] {problema}  
% Se probó de la manera {x} que los objetivos se alcanzaron  
% con los resultados siguientes {x} 
 

\prefacesection{Resumen}

%%%% Empieza aquí:
% Los sistemas de recomendaci\'on sensibles al contexto
% consideran factores que dependen directamente del
% dominio de aplicaci\'on, el objetivo del sistema y la situación actual del usuario.  
% Normalmente en la implementaci\'on de este tipo de sistemas  
% se utilizan t\'ecnicas h\'ibridas con el objetivo de 
% mejorar el desempe\~{n}o, siguiendo estrat\'egias de pre y post
% filtrado. En esta tesis se propone una
% metodolog\'ia para el desarrollo de sistemas de recomendaci\'on sensibles al contexto,
% utilizando l\'ogica difusa para expresar las reglas y variables utilizadas
% al describir la situaci\'on actual y los atributos de las entidades que intervienen.
% El objetivo de esta metodolog\'ia es permitir tanto a usuarios como dise\~nadores
% el uso de variables ling\"uisticas para mejorar facilitarles la interacci\'on con el sistema.
% Para validar la propuesta se implementaron dos aplicaciones web con las 
% se hicieron pruebas de usabilidad y satisfacción. 

% Al integrar estas técnicas el desempeño mejoró en {x} respecto a {y}
% de manera estadísticamente significativa logrando para los datos de prueba 
% un error absoluto de XX.


En esta tesis se propone un m\'etodo para sistemas de recomendaci\'on
sensible al contexto que puede ser aplicado en diferentes dominios.
Los principales objetivos son: 1) mejorar las
recomendaciones utilizando informaci\'on contextual, y en consecuencia,
mejorar el grado de satisfacci\'on del usuario al recibir
recomendaciones del sistema y, 2) proponer un m\'etodo de
recomendaci\'on implementando el contexto en un sistema de recomendaci\'on
h\'ibrido. El m\'etodo propuesto involucra tres t\'ecnicas que trabajan
simult\'aneamente para obtener las recomendaciones: filtrado
colaborativo, basado en contenido  y un Sistema de Inferencia Difuso
para procesar reglas y atributos difusos. Posteriormente, las
recomendaciones son filtradas por el contexto del usuario para
adaptarlas a su situaci\'on actual. Este enfoque es llamado post-
filtrado. Al integrar estas t\'ecnicas el desempeño del sistema mejor\'o
el nivel de satisfacci\'on del usuario, con respecto al sistema de
recomendaci\'on no contextual, de manera significativa,
logrando para los datos de prueba un error porcentual de 16\%.



%\chapter{Resumen}
\prefacesection{Abstract}

In this thesis is proposed a method for context-aware recommender
system that can be applied in different domains. The goals are: 1)
improve recommendations using contextual information to increase the
degree of satisfaction of the user in the recommendations and, 2) to
propose a method of recommendation implementing context in a hybrid
recommender system. The proposed method involves three techniques that
work simultaneously to get recommendations: collaborative filtering,
content-based and a Fuzzy Inference System to process rules and fuzzy
attributes. Subsequently, recommendations are adjusted to the user's
context, this approach is called post-filtering. Through integration of
these techniques, the system improves the level of user satisfaction
with respect  the non-contextual recommender system significantly, 
the system achieves a percentage error of 16\% for the test data.


% Traditional recommender systems has been improved in recent years,
% context has been used to implement personalized and contextualized
% recommendations for this kind of systems in order to increase the user
% satisfaction with respect of the recommended items. These systems are
% called context-aware recommender systems and are based in pre-filtering 
% and post-filtering mainly. The context is defined  into the
% application due to the large amount of contextual factors, these
% depend of the domain and the system goals. Context-aware recommender
% systems use hybrid methods and fuzzy techniques to improve the
% processing of language human-machine. This thesis defines  an
% architecture  for a context-aware recommender system that utilizes
% recommendation techniques (collaborative and content-based), Fuzzy
% Inference Systems and fuzzy attributes that are involved in the
% contextual recommendation process.










