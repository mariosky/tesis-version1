\prefacesection{Resumen}

Los sistemas de recomendaci\'on surgen por la necesidad 
de los usuarios para seleccionar informaci\'on de su interes. 
Un sistema de recomendaci\'on propone una lista de items 
recomendados basados en las preferencias de los usuarios. 
La integraci\'on del contexto en estos sistemas lleva a 
un mayor acercamiento a las necesidades reales de los 
usuarios, el contexto permite una interacci\'on mas 
personalizada que refleja a mayor detalle los gustos y 
necesidades espec\'ificas del usuario. En esta tesis se 
propone un m\'etodo para sistemas de recomendaci\'on 
sensibles al contexto que utiliza t\'ecnicas de l\'ogica 
difusa y un sistema h\'ibrido en el motor de recomendaci\'on. 
En el m\'etodo propuesto, las t\'ecnicas implementadas 
trabajan simult\'aneamente para proveer recomendaciones 
al usuario con el objetivo de minimizar el efecto de 
los problemas que presenta cada t\'ecnica de 
recomendaci\'on. El sistema de inferencia difuso 
tiene el rol de experto para proponer recomendaciones 
a trav\'es del conocimiento basado en las reglas difusas. 
La implementaci\'on del paradigma de post-filtrado 
permite el ajuste de las recomendaciones en un 
contexto espec\'ifico. En esta tesis se analiza el 
trabajo relacionado y se destacan las principales 
contribuciones del m\'etodo, presentando los resultados 
de los experimentos realizados que validan el 
m\'etodo propuesto.

\prefacesection{Abstract}

Recommender systems arise from the user needs
to select information of their interest. A recommender system
proposes a list of suggested items which is based on
user preferences. The integration of context in these
systems leads closer to the real needs of
users, context allows a more personalized interaction
reflecting greater detail the specific tastes and needs.
In this thesis a method for recommender systems is 
proposed context-sensitive uses fuzzy logic techniques 
and system hybrid engine in the recommendation. 
In the proposed method, the recommender techniques 
are working simultaneously to provide recommendations 
to the user in order to minimize the effect of the 
problems that it presents each recommendation technique. 
The fuzzy inference system has the expert role to 
make recommendations through of knowledge based on 
the fuzzy rules. The implementation of post-filtering 
paradigm allows the adjustment of recommendations in a 
specific context. This research discusses related work 
and highlights the main contributions of the method, 
also it presents the result of experiments 
to validate the proposed method.












