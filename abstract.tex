%% Los cap'itulos inician con \chapter{T'itulo}, estos aparecen numerados y
%% se incluyen en el 'indice general.
%%
%% Recuerda que aqu'i ya puedes escribir acentos como: 'a, 'e, 'i, etc.
%% La letra n con tilde es: 'n.


% 
%MG: En el resumen debes hablar de todo el trabajo revisa: 
%% http://www.slideshare.net/jjmerelo/cmo-escribir-y-publicar-trabajos-cientficos-17948694

% Da una breve introducción al tema (problema)
% Los métodos que pleanteamos aquí son {metodología}
% para [mejorar, evitar, comprobar] {problema}  
% Se probó de la manera {x} que los objetivos se alcanzaron  
% con los resultados siguientes {x} 
 

\prefacesection{Resumen}

%%%% Empieza aquí:
Los sistemas de recomendaci\'on sensibles al contexto
consideran factores que dependen directamente del
dominio de aplicaci\'on, el objetivo del sistema y la situación actual del usuario.  
Normalmente en la implementaci\'on de este tipo de sistemas  
se utilizan t\'ecnicas h\'ibridas con el objetivo de 
mejorar el desempe\~{n}o, siguiendo estrat\'egias de pre y post
filtrado. En esta tesis se propone una
metodolog\'ia para el desarrollo de sistemas de recomendaci\'on sensibles al contexto,
utilizando l\'ogica difusa para expresar las reglas y variables utilizadas
al describir la situaci\'on actual y los atributos de las entidades que intervienen.
El objetivo de esta metodolog\'ia es permitir tanto a usuarios como dise\~nadores
el uso de variables ling\"uisticas para mejorar facilitarles la interacci\'on con el sistema.
Para validar la propuesta se implementaron dos aplicaciones web con las 
se hicieron pruebas de usabilidad y satisfacción. 

Al integrar estas técnicas el desempeño mejoró en {x} respecto a {y}
de manera estadísticamente significativa logrando para los datos de prueba 
un error absoluto de XX.


%\chapter{Resumen}
\prefacesection{Abstract}

Traditional recommender systems has been improved in recent years,
context has been used to implement personalized and contextualized
recommendations for this kind of systems in order to increase the user
satisfaction with respect of the recommended items. These systems are
called context-aware recommender systems and are based in pre-filtering and post-filtering mainly. The context is defined  into the
application due to the large amount of contextual factors, these
depend of the domain and the system goals. Context-aware recommender
systems use hybrid methods and fuzzy techniques to improve the
processing of language human-machine. This thesis defines  an
architecture  for a context-aware recommender system that utilizes
recommendation techniques (collaborative and content-based), Fuzzy
Inference Systems and fuzzy attributes that are involved in the
contextual recommendation process.










