\chapter{Conclusions and future work} \label{conclusions}

In the past, computer software was designed with little regard for the
user, so the user had to somehow adapt to the system. This approach to
system design is not at all appropriate today, the system must adapt
to the user. \\ This is why design principles  \cite{gong2004guidelines} 
are so important. The principles represent high-level concepts 
applied to the systems for the better design of user interfaces. 
We took this principles as a guideline to design the first prototype, 
however  when the system is in a pilot test, we hope find the 
common problems that the users face normally.
So, it was observed the user behavior in the test, in order to identify
the most frequently difficult and doubts about proposed tasks.\\ 
Later, it was done a brief interview with the users in order to 
understand their feelings or mood, their
experience, and overall, their opinion about the system prototype.  
The conclusions are based in user's comments, then the common errors 
in the system interface are summarized in three points:
\begin{enumerate}  
\item  Incomplete information for user, i.e., the system doesn't had
enough and clear information to be a friendly interface, and
therefore the user couldn't do easily a task.
\item Fails in design, because of unordered elements in the screen, 
in other words, the elements are not in the correct site into the 
screen to be easily identified per users.
\item Fails in the language and confusion, because of the English 
language is not the native language of the users.
\end{enumerate}
The three points mentioned represent the null values in data
Table (see Table  \ref{tab:datausers}), some users don't perform the task
because they were confused, so they decided to omit the task. The null
values weren't took in account when the median was calculated to obtain the
confidence interval (see Figure  \ref{fig:ci95}).\\  
By other hand, the USE questionnaire was useful to identify the
weaknesses in the system prototype. The percent in the factors is in
acceptable level (80\%), it allows to say that the system has a good
performance in the first test, but an increment in the number of users
to perform the next possible test could get more real information.\\
For the future work we proposed to improve the problems found 
in the user interface, so the proposals are the following:
\begin{enumerate} 
\item  Redesign the user interface could helps to be more friendly for
users. The redesign involves:
  \begin{enumerate}  
  \item Analyze the amount of information enough for a easy understanding, 
  i.e., how much information the user needs seeing without overload it.
  \item Modify the tasks descriptions in the most simple way to avoid 
  confusion.
  \item Add more language functionalities for to facilitate the tasks 
  for users.
  \end{enumerate}
\item To apply the usability test again with the changes in the interface 
in order to observe the level of improves and to compare the results. 
\item Apply an statistical test to analyze the results.
\item Add collaborative filtering based on model (matrix
factorization technique) in the system prototype in order to improve 
the level of user satisfaction in the context.
\item Add any contextual factors (such as companion, time of day, 
budget, etc.) in order to include more context information that contributes
to improve the recommendations.
\end{enumerate}
The proposed method was used in the system prototype in order to
validate its performance, the case of study shows acceptable results
but we consider that the method could be efficient in others domains.\\ 
The challenge is to apply the method in a e-learning environment
because the context involves more precise factors (for instance the
level of noise, the level of light, the level of knowledge, the location 
of the users, etc.) and the recommendation process considers more 
conditions (or fuzzy rules) to make a recommendations, as well as the information 
of the user profile that contains more characteristics of the user
preferences (as goals, level, learning style, activities, homework,
score, average, etc.). The base of this future work is the Protoboard
system  \cite{garcia2007simple} that is a e-learning platform
for students of Tijuana Institute of Technology.






