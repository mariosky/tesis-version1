\chapter{Conclusions and future work} \label{sec:5}

We observed the users behaviour to identify the most frecuently difficults and
doubts about tasks. We did a brief interview with users after the test in order
to understand their  feelings or mood, their ideas about the experience, and
overall, their opinion about the context-aware recommender system.  The
conclusions are based in user's comments, then the main errors in the system
interface are summarized in three points:
\begin{enumerate}  
\item  Incomplete information for user, i.e., the system doesn't had enough and clear information to be a friendly interface, and therefore the user couldn't do easily a task.
\item Fails in design, because of unordered elements in the screen, in other words, the elements are not in the correct site into the screen to be easily identified per users.
\item Fails in the language and confusion, because of the english language is not the native language of the users.
\end{enumerate}

The three points mentioned are related to the null values in data table (see
Table \ref{tab:datausers}), some users didn't the task because they were
confused, so they decided to omit the task. The null values weren't took in
account when the median was calculated (see Table \ref{tab:ic}).\\  The USE
questionnaire was useful to identify the weaknesses in the context-aware
recommender system.  The percent is upper of the acceptable (80\%), the results
allow to say that the system has a good performance. \\ For the future work we
proposed to improve the problems found in the user interface, so the proposals
are the following:

\begin{enumerate}  
\item  Redesign the user interface could helps to be more friendly for users. Due to the issues, the redesign involves: 
  \begin{enumerate}  
  \item Analyze the amount of information enough for a easy understanding, i.e., how much information the user needs seeing without overload it.
  \item Modify the tasks descriptions in the most simple way to avoid confusion.
  \item Add more language functionalities for to facilitate the tasks for users.
  \end{enumerate}
\item  To apply the usability test again with the changes in the interface in order to observe the level of improves and to compare the results. 
\item  Apply an statistical test to analize the results.
\item  Add collaborative filtering based on model (matrix factorization technique) within the context-aware recommender system in order to improve the level of user satisfaction in the context. 
\item  Add any contextual factors (such as companion, time of day, budget, etc.) in order to include more context information that could be relevant in the recommendations.
\end{enumerate}
